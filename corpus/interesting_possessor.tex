\documentclass{article}
\usepackage[T1]{fontenc}
\usepackage[utf8x]{inputenc}
\usepackage{longtable}
\usepackage[left=3cm, right=3cm]{geometry}\title{interesting\_possessor}
\date{}
\begin{document}
\maketitle
\noindent
\begin{longtable}{p{1cm}|p{1cm}|p{1cm}|p{13cm}}
ID&Lemma&Tag&Sentence\\
\hline
2&bróðir&nkfe&bjó að Hofi \textbf{bræðra} Höfðaströnd\\
\hline
3&bróðir&nkfe&« Eigi mun eg ágirnast föðurarf þinn en við hann mun eg vera héðan af sem við þig sjálfan og skal hann nú sitja hjá mér en eigi \textbf{bræðra} þér\\
\hline
4&bróðir&nkfe&« Ei varði mig \textbf{bræðra} að þú mundir vilja vera ræningi bræðra minna\\
\hline
5&bróðir&nkee&sonur \textbf{bróður} frá Kambi\\
\hline
6&bróðir&nkee&» Hann sagði að hann hefði drepið \textbf{bróður} og hefnt bróður síns « en eg gat það eigi með vottum sýnt\\
\hline
7&bróðir&nkfe&að tækjust góðar sættir með yður frændum því að mér er kunnigt skaplyndi þeirra manna er málum eiga að skipta við yður að þeir munu það allt vel halda er þeir verða á \textbf{bræðra} \\
\hline
8&bróðir&nkfe&Varð þar mikil orusta \textbf{bræðra} \\
\hline
9&bróðir&nkee&Þorfinni annan \textbf{bróður} \\
\hline
10&bróðir&nkfe&Munum vér eigi allir verr kunna um ráða þetta mál því að eg mun hvergi í móti standa að það gangi fram er ykkar sómi vaxi \textbf{bræðra} \\
\hline
11&bróðir&nkee&Máttu til þess ætla að eg mun mér engu af skipta um það er til \textbf{bróður} heyrir meðan eg spyr ekki til Ásbjarnar bróður þíns\\
\hline
12&bróðir&nkee&En er liðsmenn urðu varir við að þeir voru þar þá gripu þeir til vopna \textbf{bróður} og gengu að þeim Jóni en þeir bjuggust við og var við sjálft að þar mundi alþýða berjast\\
\hline
13&bróðir&nkfe&Stökk sinn veg hvor þeirra bræðra \textbf{bræðra} \\
\hline
14&bróðir&nkee&En eg sendi þig vestur \textbf{bróður} Barðaströnd til Gests bróður míns\\
\hline
15&bróðir&nkfe&Brotnað hafði lærleggur \textbf{bræðra} í glímu þeirra bræðra\\
\hline
16&bróðir&nkee&« Nú skiptir miklu hverju þú vilt til \textbf{bróður} snúa og ger nú minn sóma meira en eg er verð\\
\hline
17&bróðir&nkfe& \textbf{bræðra} átti dóttur þá er Ástríður hét\\
\hline
18&bróðir&nkee&Kærir Ormur þá um vígsmálið Klængs bróður síns og vill \textbf{bróður} ekki eiga undir biskupi\\
\hline
19&bróðir&nkee&» « Það kemur til \textbf{bróður} að mér þykir aldrei fyrir von komið að hefnt muni verða Helga bróður míns meðan eg lifi eftir\\
\hline
20&bróðir&nkfe&» En Þuríður mælti að þeim sonum hennar skyldi ekki deila dögurð og kvaðst hún deila \textbf{bræðra} \\
\hline
21&bróðir&nkfe&Ljótur sér nú hvar komið var og fleygir \textbf{bræðra} sér sverðinu til þeirra bræðra en ætlar að steypa sér útbyrðis og sér þá ei undanbragð annað\\
\hline
22&bróðir&nkfe&Rakki lá \textbf{bræðra} eldhúsinu\\
\hline
23&bróðir&nkee&« Eigi vil eg að þú virðir svo að eg óttist \textbf{bróður} að Helga\\
\hline
24&bróðir&nkfe&» Hann tekur þá það til \textbf{bræðra} sem þeir vildu og skildust að því\\
\hline
25&bróðir&nkfe&Spyrst þetta nú til \textbf{bræðra} og brá Víglundi mjög við þetta\\
\hline
26&bróðir&nkfe&En eigi langan tíma upp frá þessu tók Þorsteinn sótt og andaðist og þó að fyrr segi líflát \textbf{bræðra} þá lést Jökull fyrstur þeirra bræðra en Þórir lifði lengst\\
\hline
27&bróðir&nkee&Hann fékk sér skíð og kom um nóttina til Hofsár til Bjarnar og vakti hann af \textbf{bróður} \\
\hline
28&bróðir&nkfe&Þá gekk til Snorri goði og þreifaði um fótinn og fann að spjót stóð í gegnum fótinn milli hásinarinnar og fótleggsins og hafði níst allt \textbf{bræðra} \\
\hline
29&bróðir&nkfe&Varð hann feginn \textbf{bræðra} fundi\\
\hline
30&bróðir&nkfe&og samþykki alls almúga og fékk sér skip og fór til \textbf{bræðra} að sækja lík þeirra bræðra\\
\hline
31&bróðir&nkfeg&Björn hafði krókaspjót í hendi og hjálm á höfði og var gyrður sverði og skjöldur \textbf{bræðranna} hlið en Arnór þvari hafði sverð í hendi og hendi um öxl til og hélt um miðjan meðalkaflann\\
\hline
32&bróðir&nkee&« Þér fel eg \textbf{bróður} hendi Skarphéðinn að hefna bróður þíns og vænti eg að þér muni vel fara þó að hann sé eigi skilgetinn og þú munir mest eftir ganga\\
\hline
33&bróðir&nkee&» Moldi sagði \textbf{bróður} \\
\hline
34&bróðir&nkfe&« Það er öllum \textbf{bræðra} kunnigt að hér í vetur hafa verið miklar tökur\\
\hline
35&bróðir&nkee&« Búið arki að auðnu til \textbf{bróður} sem draga vill\\
\hline
36&bróðir&nkee&Hafði Hákon konungur þar mest varaðan Sturlu við að hann skyldi eigi auka manndráp \textbf{bróður} landinu og reka menn heldur utan\\
\hline
37&bróðir&nkee&Er þér það skjótt af að segja að eg hefi svo mikinn ástarhug til \textbf{bróður} fellt að eg fæ það eigi úr hug mér gert\\
\hline
38&bróðir&nkee&Nú er \textbf{bróður} því að segja að þeir bræður komu heim og sögðu fall Sigurðar konungs og Klypps bróður síns\\
\hline
39&bróðir&nkfe&Ófeigur mælti \textbf{bræðra} \\
\hline
40&bróðir&nkee&Nú er þar til \textbf{bróður} að taka að Þórarinn Ragabróðir spyr lát Glúms bróður síns\\
\hline
41&bróðir&nkfe&Nú skal segja hvað Þorgeir hefst að \textbf{bræðra} \\
\hline
42&bróðir&nkee&Ólafur varð skjótt í brottu af fundi \textbf{bróður} og reið aftur til Hrafns bróður síns\\
\hline
43&bróðir&nkee&Kvað þess þykja meiri von að Þorgils mundi grið hafa og bíð eg þunga daga ef eg skal \textbf{bróður} bíða að Þorgils væri drepinn\\
\hline
44&bróðir&nkee&því að við eigum málum að skipta við Hrafnkel goða um víg Einars Þorbjarnarsonar en við megum vel hlíta okkrum flutningi með þínu \textbf{bróður} \\
\hline
45&bróðir&nkee&dóttur \textbf{bróður} \\
\hline
46&bróðir&nkee&Þorvaldur hafði mart manna um sig og hafði það við orð að sitja fyrir þeim á \textbf{bróður} \\
\hline
47&bróðir&nkfe&Um vorið eftir víg Jóns var það tíðinda á Íslandi að Sighvatur Sturluson lét af höndum Hrafnssonu og tók þeim fari að \textbf{bræðra} \\
\hline
48&bær&nkee&En er hann kom út til bæjar fór hann brátt á konungsfund og tóku þeir tal \textbf{bæjar} \\
\hline
49&bær&nkee&Konungur lét sér það vel þokkast og bundu það við hann einkamálum sín í milli meðan sú stefna \textbf{bæjar} \\
\hline
50&bær&nkee&En þegar er þeir komu til bæjar Atla að \textbf{bæjar} þá tók sveinninn að biðja að hann væri skírður\\
\hline
51&bær&nkee-s&Klængs \textbf{Bæjar} var geymt eftir í Reykjaholti\\
\hline
52&bær&nkee&Margir hans menn löttu hann þess en eigi að síður fór hann með mikla sveit \textbf{bæjar} og kom til þess bæjar er drottning réð fyrir\\
\hline
53&bær&nkeeg&Kann hann vel að vera með góðum mönnum \textbf{bæjarins} \\
\hline
54&bær&nkee&» Kormákur kvað lítils mundu við þurfa og litast hann um og sér hvergi hrossin en þau höfðu vafist í einu lækjarfari skammt \textbf{bæjar} því er þau sátu\\
\hline
55&bær&nkeeg&Menn Ólafs konungs voru út á Gaularási og héldu \textbf{bæjarins} \\
\hline
56&bær&nkee&Fóru þeir svo allan dag fyrir bæ Rauðs þá flaut þar fyrir landi dreki hans sá hinn \textbf{bæjar} \\
\hline
57&bær&nkeeg&En þá báru þeir hann til skips og fóru til bæjar þess er heitir \textbf{bæjarins} Sæheimruð og lentu þar\\
\hline
58&bær&nkee&Reið hann alla þá nótt og eftir um daginn \textbf{bæjar} \\
\hline
59&bær&nkee-s&Þótti hann því ráðið hafa \textbf{Bæjar} Ketill gekk treglega af staðnum í Hítardal\\
\hline
60&bær&nkeeg&Og er þeir voru á burtu fór Gísli heim og býr þegar ferð sína og fær sér skip og flytur þangað á mikinn fjárhlut og fer Auður kona hans með honum og Guðríður fóstra hans og út til \textbf{bæjarins} og koma þar við land\\
\hline
61&bær&nkee&Fengu þeir sér róðrarskip það er við \textbf{bæjar} hæfi var\\
\hline
62&bær&nkee&« Því em eg vanur að geyma þar undir skipreiða minn eða hví fer eigi nokkur yðar inn undir skipið og rannsakið enn \textbf{bæjar} \\
\hline
63&bær&nkee&Það bréf skaltu láta koma á morgun fyrir brjóst þér og vefja dúkinum að utan og um búk þér svo sem hann \textbf{bæjar} \\
\hline
64&bær&nkfe&Og er þau koma þar og drepa á dyr þá mælti Björn við húskarl þann er \textbf{bæja} stakkgarðinum hafði verið að hann gengi út og byði Þorsteini þar að vera ef hann væri kominn » en eg get\\
\hline
65&bær&nkee&Reyndu Ránar dætur drengina og buðu þeim \textbf{bæjar} faðmlög\\
\hline
66&bær&nkeeg&» Glúmur reis upp þá og mælti að hún skyldi gleipa \textbf{bæjarins} örmust\\
\hline
67&bær&nkee&» Gengu þeir þá þrír bræður á eitt skip \textbf{bæjar} \\
\hline
68&bær&nkee&En er konungur þóttist sannfróður um \textbf{bæjar} sendir konungur menn og orðsending inn í Þrándheim og stefndi bóndum út til bæjar\\
\hline
69&bær&nkee&Ætla eg sjaldan að kveðja þig til ferðar með mér og óþökk skaltu af mér hafa fyrir þessa \textbf{bæjar} \\
\hline
70&bær&nkee&Dvaldist hann þar margar nætur og lá þar í skógum og svaf og \textbf{bæjar} ekki að sér\\
\hline
71&bær&nkeeg&En er á leið vorið og þeir sóttu út í Víkina þá hvarf Finnur í brott \textbf{bæjarins} liðinu nokkura daga\\
\hline
72&bær&nkee&En þá báru þeir hann til \textbf{bæjar} og fóru til bæjar þess er heitir á Sæheimruð og lentu þar\\
\hline
73&bær&nkeeg&Og er spor \textbf{bæjarins} lágu til bæjarins þá mælti Bjarni að þeir þrír skyldu ganga jafnframt og þar eftir aðrir þrír og síðan hinir þriðju þrír « og munu þá sýnast þriggja manna spor\\
\hline
74&bær&nkee-s&Þá riðu þeir til Kambs og þá Magnús grið af orðum \textbf{Bæjar} af Möðruvöllum\\
\hline
75&bær&nkee&Þeir finna eigi Gísla þar og fara nú um alla skóga að leita Gísla og finna hann \textbf{bæjar} \\
\hline
76&bær&nkee&Síðan sendi biskup eftir Einari þambarskelfi og kom \textbf{bæjar} til bæjar\\
\hline
77&bær&nkee&Auðun settist niður við árbakkann og kvaðst þyrsta \textbf{bæjar} \\
\hline
78&bær&nkee&lét þá búa skip sín og fór um vorið út til \textbf{bæjar} og sat þar um vorið þá er þar var fjölmennast og þungi var fluttur til bæjar af öðrum löndum\\
\hline
79&bær&nkee&En er hann kom til bæjar gekk hann \textbf{bæjar} með lið það\\
\hline
80&bær&nkee&Svo bar til ferð þeirra að þeir komu aftan dags til Atlaeyjar og lögðu þar að landi en þar var í eyjunni skammt upp bú mikið er átti Eiríkur \textbf{bæjar} \\
\hline
81&bær&nkeeg&En um kveldið eftir mat var biskup inni en menn hans margir úti í \textbf{bæjarins} \\
\hline
82&bær&nkee&Steinn fór einn til bæjar og kom sér brátt í tal við þá Sóta og drakk með \textbf{bæjar} um kveldið\\
\hline
83&bær&nkfeg&» Síðan sóttu þeir fast eftir þeim og fundust á hrísum upp \textbf{bæjanna} Dalsbæ milli bæjanna og Hellu er Narfi bjó\\
\hline
84&bóndi&nkee&Annarri ör skaut Gunnar að Úlfhéðni ráðamanni \textbf{bónda} og kom sú á hann miðjan og féll hann fyrir fætur bónda einum en bóndinn féll um hann þveran\\
\hline
85&bóndi&nkee&Svo bar hér til að það var einn dag \textbf{bónda} jólunum að komu til Einars bónda illvirkjar margir saman\\
\hline
86&bóndi&nkee&En þá síðan er menn þóttust verða ósjálfráðir fyrir ríki hans þá leituðu sumir í brott úr \textbf{bónda} \\
\hline
87&bóndi&nkee&Þú skalt nefna votta \textbf{bónda} rekkjustokki bónda þíns og segja skilið við hann lagaskilnaði sem þú mátt framast að alþingismáli réttu og allsherjarlögum\\
\hline
88&bóndi&nkee&» Helgi kvaðst ætla að eigi mundi skjótt hrinda mega ást þeirra Helgu « og er þér engi óvirðing í bóndi ef eg bið konunnar með réttum landslögum þeim sem hér ganga og með slíku fé sem þér \textbf{bónda} \\
\hline
89&bóndi&nkee&Þaðan sigldi hann vestur um Bretland og svo norður með Bretlandi og norður um Skotlandsfjörðu og létti eigi fyrr en hann kom norður í Þrasvík \textbf{bónda} Katanesi til Skeggja bónda\\
\hline
90&bóndi&nkfe&Steinvör kom og til með biskupi \textbf{bænda} \\
\hline
91&bóndi&nkfe&Suður \textbf{bænda} Auðsholti kom biskup í móti Þórði og bauð allt hið sama af bænda hendi sem fyrr\\
\hline
92&bóndi&nkee&Páll og Magnús Magnússynir \textbf{bónda} \\
\hline
93&bóndi&nkee&Þá tók Þorsteinn bóndi Guðríði af stólinum í fang sér og settist í bekkinn annan með hana gegnt líki Þorsteins og taldi um fyrir henni marga vega og huggaði hana og hét henni því að hann mundi fara með henni til \textbf{bónda} með líki Þorsteins bónda hennar og förunauta hans\\
\hline
94&bóndi&nkee&Svartur hét \textbf{bónda} Þorgríms bónda\\
\hline
95&bóndi&nkfe&« Skil eg þetta gjörla hvað þú segir mér \textbf{bænda} því hversu hverjum var farið bænda þinna en hitt verður enn ekki sagt hverjum þú unnir mest\\
\hline
96&bóndi&nkee&Var hvort \textbf{bónda} öðru vel hugþokkað\\
\hline
97&bóndi&nkee&Annar Nikulás var sonur Skratta-Bjarnar \textbf{bónda} á Griðli\\
\hline
98&bóndi&nkee&Bar þá brátt \textbf{bónda} eyjunni\\
\hline
99&bóndi&nkee&Verður oft þeirra í millum að standa um \textbf{bónda} \\
\hline
100&bóndi&nkee&Í þenna tíma þóttust menn þess verða varir að úthlaupsmenn eða illvirkjar mundu vera á leið þeirri er liggur á milli Jamtalands og Raumsdals því að engir komu aftur þeir er fóru og þótt saman væru fimmtán eða tuttugu þá höfðu þó engir aftur komið og þóttust menn því vita að frágerðarmaður mundi úti \textbf{bónda} \\
\hline
101&bóndi&nkee&Egill bað þá suma úti vera og gæta að \textbf{bónda} kæmist út\\
\hline
102&bóndi&nkee&Svo er sagt að Helga Granadóttir hljóp nú \textbf{bónda} búi Háls bónda síns og heim til föður síns og hitti ekki Áskel\\
\hline
103&bóndi&nkee&En hann reið eigi lengra en \textbf{bónda} Auðkúlustaði til Bjarna bónda og reið skjótt aftur til héraðs\\
\hline
104&bóndi&nkee&hann var \textbf{bónda} Stiklastöðum í orustu þá er hinn helgi Ólafur konungur féll\\
\hline
105&bóndi&nkee&Selfljót gengur fyrir austan úr heiðinni milli \textbf{bónda} og svo fellur það ofan í Lagarfljót\\
\hline
106&bóndi&nkee&» Ingjaldur leiddi þá til \textbf{bónda} og komust þeir þar í brott svo að víkingar urðu ekki varir við fyrir gný og eldsgangi og það er þeir voru eigi feigir\\
\hline
107&bóndi&nkee&Voru þá bundin sár \textbf{bónda} Egils\\
\hline
108&bóndi&nkee&Hann átti að ráða fyrir einu þorpi í Danmörk þar er \textbf{bónda} Vendilskaga heitir\\
\hline
109&bóndi&nkee&Báðu þeir Þorgils vægja til fyrir Brynjólfi bónda og ráði þeirra manna er þar voru mest \textbf{bónda} \\
\hline
110&bóndi&nkee&lík Ólafs konungs og voru um það mjög hugsjúkir hvernug þeir fengju til gætt að eigi næðu óvinir \textbf{bónda} að misfara með líkinu því að þeir heyrðu þær ræður bónda að það ráð mundi til liggja ef lík konungs fyndist að brenna það eða flytja út á sæ og sökkva niður\\
\hline
111&bóndi&nkee&Sæmundar og Guð mundar \textbf{bónda} og Guð mundar\\
\hline
112&bóndi&nkee&Það var tveim vetrum eftir víg Bolla \textbf{bónda} \\
\hline
113&bóndi&nkee&Þeir riðu að durum og gekk Sturla inn en hinir sátu \textbf{bónda} baki úti\\
\hline
114&bóndi&nkfe&Má eg vel sæma við þann \textbf{bænda} en best að engi sé\\
\hline
115&bóndi&nkee&Konungur bauð til sín stórmenni \textbf{bónda} \\
\hline
116&bú&nhee&Grettir reið heim til \textbf{bús} en Barði til bús síns\\
\hline
117&bú&nheeg&Sturla Sveinsson \textbf{búsins} \\
\hline
118&bú&nhee&fóru þá aftur er \textbf{bús} leið veturinn upp á mörkina\\
\hline
119&bú&nhee&Njáll reið heim af þingi og synir hans og voru þeir heima allir um \textbf{bús} \\
\hline
120&bú&nhee&settu eftir hann bautasteina \textbf{bús} \\
\hline
121&bú&nhee&Reið hann til þings um sumarið og fann þar þá menn er fram fluttu kristilegan boðskap sem brátt mun sagt \textbf{bús} \\
\hline
122&bú&nhee&Skyldi brullaup það vera \textbf{bús} Höskuldsstöðum\\
\hline
123&bú&nhee&kunni stórilla því er Kálfur hafði verið í bardaga í \textbf{bús} konungi\\
\hline
124&bú&nhee&Taka þeir nú upp frændsemi sína góða héðan í \textbf{bús} \\
\hline
125&bú&nhee&Og þar er eg drap Helga bróður ykkarn þá vil eg það bæta og gefa vil eg þér Einar sverðið Jarðhússnaut því að þú ert verður að \textbf{bús} \\
\hline
126&bú&nhee&sagðist þó eigi vilja langvistir \textbf{bús} þar um Ísafjörð\\
\hline
127&bú&nhee&Höskuldur færði fé allt til skips það sem Hrútur \textbf{bús} \\
\hline
128&bú&nhee&» Hallgerður hældist jafnan um víg Svarts en Bergþóru líkaði það \textbf{bús} \\
\hline
129&bú&nhee& \textbf{bús} getur Bjarni skáld\\
\hline
130&bú&nhee&Er það hér skjótast af að segja að þeir Snorri og Þorvaldur bundu vináttu sína með því móti að Gissur son Þorvalds skyldi fá Ingibjargar dóttur \textbf{bús} en Þorvaldur skyldi eiga hlut að við Hallveigu Ormsdóttur að hún gerði félag við Snorra og fara til bús með honum\\
\hline
131&bú&nhee&En Sámur sendi Þorbjörn ofan til Leikskála og skyldi hann þar búa en kona \textbf{bús} fór til bús með honum á Aðalból og sitja þar um nokkra hríð\\
\hline
132&bú&nheeg&En Oddur oremus mágur Þorgríms fór þá norður til Sighvats með börn sín Einar og \textbf{búsins} \\
\hline
133&bú&nheeg&sagði að þeir ætluðu að hann \textbf{búsins} Þórði bústað\\
\hline
134&bú&nhee&Hrafns Nú er þar til máls að taka er fyrr var frá horfið að þá er þeir Guðmundur biskup og Hrafn Sveinbjarnarson komu út og höfðu einn vetur verið í Noregi fór Hrafn vestur í Arnarfjörð \textbf{bús} Eyri til bús síns\\
\hline
135&bú&nhee&En er Snorri spurði austur í Skál að fátt var með þeim sendi Snorri menn ofan til Þorleifs en sumir fóru allt til \textbf{bús} og á Eyri\\
\hline
136&bú&nhee&Þorfinnur fór heim til \textbf{bús} og sat heima mjög til jóla sem fyrr er sagt\\
\hline
137&bú&nhee&Nú hljóp Skúta þegar til sinna manna og fara þeir á bak og riðu brott af \textbf{bús} \\
\hline
138&bú&nhee&Alls drápu þeir nær hundrað manna og tóku þar ógrynni fjár og komu aftur um vorið við svo \textbf{bús} \\
\hline
139&bú&nhee&» Þráinn var skamma stund í hafi og kom til \textbf{bús} og fór heim til bús síns\\
\hline
140&bú&nhee&Eftir það reið Þórður vestur í sveitir og sættist á víg Sörla við Ásbjörn mág sinn og \textbf{bús} \\
\hline
141&dagur&nkee&» En er þeir heyrðu þetta þeystu þeir út á mýrarnar \textbf{dags} \\
\hline
142&dagur&nkee&Grettir kom nú það í hug að hann þóttist hafa orðið varhluta fyrir Auðuni að knattleiknum sem áður er sagt og vildi hann prófa hvor \textbf{dags} meira hefði við gengist síðan\\
\hline
143&dagur&nkee&Skulum vér fyrir því heldur hafa hinna höfðingja dæmi er oss eru kunnari og betra er eftir að líkja að berjast um ljósa daga og með fylking en stelast eigi um nætur á sofandi \textbf{dags} \\
\hline
144&dagur&nkee&Þá er þeir voru búnir fluttu þeir Ásbjörn og Þorgeir til \textbf{dags} það er Finnbogi átti\\
\hline
145&dagur&nkee&Eg vil og ráða fyrir hversu miklu slátrað er í haust á hverju búi allra minna landseta og mun þá vel \textbf{dags} \\
\hline
146&dagur&nkee&Kom spjótið á hann miðjan og í gegnum hann og svo í brjóst á þeim manni er stóð \textbf{dags} baki honum og féllu þeir báðir dauðir\\
\hline
147&dagur&nkee& \textbf{dags} sterka\\
\hline
148&dagur&nkee&Og er nón var dags þá sneru þeir \textbf{dags} og gerði á veður hart\\
\hline
149&dagur&nkeeg&Grís settist á tal við Ingvildi og Þórhildi móður \textbf{dagsins} en þeir Karl tala við Ásgeir bónda og báðu þau öll samt koma suður síðar dagsins og fóru þeir heim fyrir og mættu þeim manni fyrir heima er Þórður gapa hét\\
\hline
150&dagur&nkee&Sváfu þeir þá enn um \textbf{dags} eftir\\
\hline
151&dagur&nkee&Því næst hvetur hann það svo það stóð \textbf{dags} kampi\\
\hline
152&dagur&nkee&» Þorkell lét sem hann heyrði eigi og bjóst vel heiman að klæðum og vopnum því að hann var skartsmaður hinn mesti og kom í síðasta lagi \textbf{dags} \\
\hline
153&dagur&nkee&Hörður ríður nú til \textbf{dags} við Hróar\\
\hline
154&dagur&nkfe&» « Þykir þér svo frændi \textbf{daga} \\
\hline
155&dagur&nkee&Þá snýr hann til móts við Víglund og taka þeir til og berjast og var \textbf{dags} atgangur lengi dags bæði harður og langur svo að þar mátti eigi í millum sjá hvor að sigrast mundi\\
\hline
156&dagur&nkee&« Nú skal eg ganga til \textbf{dags} en þér bíðið mín hér til hins þriðja dags ef þess þarf við en þér farið brottu leið yðra ef eg kem eigi aftur um það\\
\hline
157&dagur&nkee&Þetta er þó mitt ráð \textbf{dags} \\
\hline
158&dóttir&nvee&Þá bað Hallur Gissurarson Ingibjargar dóttur Sturlu og réðst það á þingi \textbf{dóttur} \\
\hline
159&dóttir&nvee&Ávaldi fór að \textbf{dóttur} og sagði honum hvern ójafnað Hallfreður gerði honum\\
\hline
160&dóttir&nvee&Hólmkell spurði foringja þeirra að nafni en hann kveðst Þórður heita og eiga heima í Austfjörðum en kvað það erindi sitt að biðja \textbf{dóttur} \\
\hline
161&dóttir&nvee&er átti Þorlaugu Hrólfsdóttur \textbf{dóttur} Ballará og Þuríðar dóttur Valþjófs Örlygssonar frá Esjubergi\\
\hline
162&dóttir&nvee&Synir þeirra voru þeir Þorgils og Narfi faðir Snorra prests er þar bjó \textbf{dóttur} \\
\hline
163&dóttir&nvee&Nær þessu var það tíðinda eitt sumar á þingi að búðir \textbf{dóttur} stóðu hið næsta og Allsherjarbúð er Magnús goði átti son Guðmundar gríss og Solveigar dóttur Jóns Loftssonar\\
\hline
164&dóttir&nvee&Búi bjó \textbf{dóttur} Esjubergi tólf vetur og átti mikið rausnarbú\\
\hline
165&dóttir&nvee&» Ekki varð af boðum \textbf{dóttur} við þá\\
\hline
166&dóttir&nvee&systir \textbf{dóttur} frá Keldum\\
\hline
167&dóttir&nvee&Þorkell hét maður er bjó í \textbf{dóttur} \\
\hline
168&dóttir&nvee&Og þá er Einar faðir Ingimundar andaðist þá gaf Ingimundur Þorgilsi frænda sínum Reyknesingagoðorð sem fyrr var ritað og var þeirra frændsemi allar stundir góð meðan þeir lifðu \textbf{dóttur} \\
\hline
169&dóttir&nvee&Fór Þórarinn á fund Guðmundar og fékk hann þar góðar \textbf{dóttur} \\
\hline
170&dóttir&nvee&Og réð þá til fulltings við Inga konung \textbf{dóttur} sonur Dags Eilífssonar og Ragnhildar dóttur Skopta Ögmundarsonar\\
\hline
171&dóttir&nvee&» Dofri tók vel orðum \textbf{dóttur} og mælti til dóttur sinnar\\
\hline
172&dóttir&nvee&Hún spyr hann hvers son hann \textbf{dóttur} \\
\hline
173&dóttir&nvee&Óleifur hinn hvíti herjaði í vesturvíking og vann Dyflinni \textbf{dóttur} Írlandi og Dyflinnarskíði og gerðist þar konungur yfir\\
\hline
174&dóttir&nvee&faðir \textbf{dóttur} er átti Helgu dóttur Þorgeirs á Víðimýri\\
\hline
175&dóttir&nvee&Bauð Sæmundur á gerð \textbf{dóttur} er þá átti Þóru\\
\hline
176&dóttir&nvee&« Eg er kominn hingað með Glúmi bróður mínum \textbf{dóttur} að biðja Hallgerðar dóttur þinnar\\
\hline
177&dóttir&nvee&Þá mælti Hallfreður \textbf{dóttur} \\
\hline
178&faðir&nkee&« að þú þóttir ei hafa \textbf{föður} vit\\
\hline
179&faðir&nkee&Hann var faðir \textbf{föður} \\
\hline
180&faðir&nkee&faðir \textbf{föður} í Reykjaholti\\
\hline
181&faðir&nkee&Og Skúta fer til \textbf{föður} og vó hann þann fyrstan mann í hefnd föður síns\\
\hline
182&faðir&nkee&móðir \textbf{föður} \\
\hline
183&faðir&nkee&Hann var faðir \textbf{föður} \\
\hline
184&faðir&nkee&móðir \textbf{föður} \\
\hline
185&faðir&nkee&« Ekki em eg því vanur að veita það er eg veit eigi \textbf{föður} beðið er\\
\hline
186&faðir&nkee&leyst og sent heim frjálsa til föður síns \textbf{föður} \\
\hline
187&faðir&nkee&faðir \textbf{föður} \\
\hline
188&faðir&nkee&» Hann kvaðst hvergi fara mundu « og skal eg hér drepinn þér til \textbf{föður} og man eg það enn að faðir minn féll í liði föður þíns og Ingimundar og hefir það af þér hlotist og þínum mönnum\\
\hline
189&faðir&nkee&bróðir Sigurðar \textbf{föður} \\
\hline
190&faðir&nkee&Krafði þá Gull-Haraldur Harald konung að hann skipti ríki við hann í helminga svo sem burðir hans voru til og ætt þar í \textbf{föður} \\
\hline
191&faðir&nkee&Hann var faðir \textbf{föður} \\
\hline
192&faðir&nkee&móðir \textbf{föður} \\
\hline
193&faðir&nkee&» En þótt það fyndist \textbf{föður} Háreki að honum þótti þetta móti skapi þá lét hann Ásmund við sýslu taka sem konungur hafði orð til send\\
\hline
194&faðir&nkee&Hann var á fóstri með Miðfjarðar-Skeggja móðurbróður sínum að Reykjum í Miðfirði þar til er hann var átján \textbf{föður} gamall\\
\hline
195&faðir&nkee&faðir \textbf{föður} \\
\hline
196&faðir&nkee&Björn varð mikill maður og kom aftur til \textbf{föður} og drap marga menn í hefnd föður síns og varð hinn röskvasti maður\\
\hline
197&faðir&nkee&Hann tók í hafi skipsögn þeirra manna er þá voru ófærir og lágu á skipsflaki og þá fann hann Vínland hið góða og kom um sumarið til Grænlands og hafði þannug með sér prest og kennimenn og fór til \textbf{föður} í Brattahlíð til Eiríks föður síns\\
\hline
198&faðir&nkee&móðir \textbf{föður} \\
\hline
199&faðir&nkee&Valþjófur var faðir \textbf{föður} \\
\hline
200&faðir&nkee&« Eg stökkti í brott Steinari syni Önundar sjóna og þótti það heldur \textbf{föður} \\
\hline
201&faðir&nkee&faðir \textbf{föður} \\
\hline
202&faðir&nkee&Hann kom á Víðimýri til Kolbeins unga og var þar um \textbf{föður} \\
\hline
203&faðir&nkee&Nú er að segja \textbf{föður} Hrafni að hann bjó skip sitt í Leiruvogum\\
\hline
204&faðir&nkee&Ásta fór þegar til Upplanda til föður síns \textbf{föður} \\
\hline
205&faðir&nkee&faðir \textbf{föður} \\
\hline
206&faðir&nkee&faðir \textbf{föður} \\
\hline
207&faðir&nkee&faðir \textbf{föður} \\
\hline
208&faðir&nkee&« Dirfð mun \textbf{föður} þykja herra er bandinginn biður yður ásjá\\
\hline
209&faðir&nkee&« Svo með því að þú hést mér eigi því þá er eg fór með þér frá Íslandi \textbf{föður} búi föður míns\\
\hline
210&faðir&nkee&« Eigi mundir þú á katlinum hafa haldið ef eg hefði ráðið og mun þetta illa gefast við ofsa \textbf{föður} og ríðið eftir honum og biðjið hann hafa ketilinn\\
\hline
211&faðir&nkee&faðir \textbf{föður} \\
\hline
212&faðir&nkee& \textbf{föður} Geira\\
\hline
213&faðir&nkee& \textbf{föður} son var Þorsteinn\\
\hline
214&faðir&nkee&faðir \textbf{föður} \\
\hline
215&faðir&nkee&» Þar til ræða þeir um þetta er \textbf{föður} reiddust og veðjuðu\\
\hline
216&faðir&nkee&móðir \textbf{föður} \\
\hline
217&faðir&nkee&Það þiggja þeir og er Þorfinnur fenginn konum til geymslu og er honum mjólk \textbf{föður} \\
\hline
218&faðir&nkee&móður Þorgils \textbf{föður} Þorgils\\
\hline
219&faðir&nkee&var faðir \textbf{föður} \\
\hline
220&faðir&nkee&Björn sonur Ketils flatnefs var faðir \textbf{föður} \\
\hline
221&faðir&nkee&faðir Óttars \textbf{föður} \\
\hline
222&faðir&nkee&til \textbf{föður} og bjó að Rauðalæk\\
\hline
223&faðir&nkee&Eyvindur fýstist til \textbf{föður} eftir andlát föður síns\\
\hline
224&faðir&nkee&Arnkell var úti staddur \textbf{föður} \\
\hline
225&faðir&nkee&Úlfur aurgoði var faðir \textbf{föður} \\
\hline
226&faðir&nkee&faðir \textbf{föður} \\
\hline
227&faðir&nkee&dóttur Gunnlaugs úr Þverárhlíð og Vélaugar \textbf{föður} frá Esjubergi\\
\hline
228&faðir&nkee&Jörundur var faðir \textbf{föður} \\
\hline
229&faðir&nkee&Þau fóru á fund hans til Skagafjarðar og sögðu honum deili á sér og sögðu hann frænda \textbf{föður} \\
\hline
230&faðir&nkee&faðir \textbf{föður} \\
\hline
231&faðir&nkee&Svo sýndist Þorsteini sem það væri hin mesta gersemi og alllíklegt til \textbf{föður} og gerði sér það í hug að duga mundi ef hann næði saxinu\\
\hline
232&faðir&nkee&Þetta haust drap Grímur hersir Öndótt kráku fyrir það er hann náði eigi fénu til handa konungi en Signý kona Öndótts bar á skip allt lausafé þeirra þegar hina sömu nótt og fór með sonu \textbf{föður} \\
\hline
233&faðir&nkee&móðir \textbf{föður} \\
\hline
234&faðir&nkee&Hálfdan og Steinvör tóku við honum vel \textbf{föður} \\
\hline
235&faðir&nkee&móður Þorgils \textbf{föður} Þorgils\\
\hline
236&faðir&nkee&Það var eitt sinn í tali þeirra bræðra að Eyvindur kvaðst heyra gott af Íslandi sagt og fýsti bróður sinn Ketil til \textbf{föður} með sér eftir andlát föður síns\\
\hline
237&faðir&nkee&Eftir það fór hann heim fagnandi til bús síns og dýrkaði alla daga lífs síns með hreinni þjónustu \textbf{föður} guð í þeirri kirkju er hann hafði honum helgað og fyrst var ger í Fljótum í nafni föður og sonar og heilags anda þeim er vegur og dýrð eilíflega einum guði í þrenningu\\
\hline
238&faðir&nkee&faðir \textbf{föður} \\
\hline
239&faðir&nkee&Þorkell sonur Geitis fór utan og jafnan landa í millum þegar er hann hafði aldur til \textbf{föður} og varð hann lítt við riðinn mál þeirra Brodd-Helga og Geitis föður síns\\
\hline
240&faðir&nkee&móðir Snæris \textbf{föður} \\
\hline
241&faðir&nkee&Um haustið fóru þeir norðan af Ströndum til Ísafjarðar og settu upp skip sitt þar er þeim þótti vel komið og búa \textbf{föður} \\
\hline
242&faðir&nkee&Hún var dóttir \textbf{föður} úr Vatnsfirði\\
\hline
243&faðir&nkee&faðir \textbf{föður} \\
\hline
244&faðir&nkee&Móðir hennar hét \textbf{föður} \\
\hline
245&faðir&nkee&segir að þá skip \textbf{föður} en Orm hinn langa\\
\hline
246&faðir&nkee&En er það lið er flúið hafði af Írlandi kom til \textbf{föður} og Sigurður spurði fall Magnúss konungs föður síns þá réðst hann þegar til ferðar með þeim og fóru þeir um haustið austur til Noregs\\
\hline
247&faðir&nkee&móðir Þorsteins \textbf{föður} \\
\hline
248&faðir&nkee&Þuríður dóttir Þorgeirs Galtasonar var móðir Styrmis en Styrmir Þorgeirsson var faðir \textbf{föður} \\
\hline
249&faðir&nkee&Fóru þar utan Sigurður seli og Kolfinna Þorvaldsdóttir utan \textbf{föður} ráð\\
\hline
250&faðir&nkee&Hún var gift \textbf{föður} \\
\hline
251&faðir&nkee&« Var eg \textbf{föður} nokkurum orustum með Haraldi konungi föður yðrum\\
\hline
252&faðir&nkee&þá kaupir hann engið \textbf{föður} því sem Ljótur kvað á og gaf fyrir tuttugu hundruð þegar í stað og skilja að því\\
\hline
253&faðir&nkee&« Það er líkast að því komir þú á leið að eg verði héraðflótta en synir mínir óbættir en því skal eg þér launa að þú skalt Steingerðar aldrei \textbf{föður} \\
\hline
254&faðir&nkee&faðir \textbf{föður} \\
\hline
255&faðir&nkee&faðir \textbf{föður} \\
\hline
256&faðir&nkee&móðir \textbf{föður} \\
\hline
257&faðir&nkee&faðir \textbf{föður} \\
\hline
258&faðir&nkee&faðir Hrafnkels \textbf{föður} \\
\hline
259&ferð&nvee&Hann tók þá það ráð að leita \textbf{ferðar} \\
\hline
260&ferð&nveeg&Bændur kærðu þetta fyrir konungi og báðu hann frelsa sig af þessum ófriði \textbf{ferðarinnar} \\
\hline
261&ferð&nvfe&skilvísa og skjóta í viðbragði og kunni vel fyrir mönnum að sjá og \textbf{ferða} ferða að skipa\\
\hline
262&ferð&nvee&» Helgi bauðst til ferðar þessarar og gera að \textbf{ferðar} áður var ætlað\\
\hline
263&ferð&nvee&En flestir menn lögðu þungt til \textbf{ferðar} ef hann hefði verið vitandi ferðar þessar\\
\hline
264&ferð&nvee&» Sauðamaður ríður ofan til \textbf{ferðar} og ofan í Gönguskarð\\
\hline
265&ferð&nvee&» Þorsteinn svarar og kvað hann lítt hafa fyrir séð hvort hann kæmi nokkurn tíma aftur eða aldrei en kvað hamingjuna hafa styrkt nú svo sitt mál að hann hafði heill aftur \textbf{ferðar} \\
\hline
266&ferð&nvee&Hann var einhleypingur og \textbf{ferðar} \\
\hline
267&ferð&nvee&» Snorri var að skipi \textbf{ferðar} nætur\\
\hline
268&ferð&nvee&en ósýn líst mér ferð yður \textbf{ferðar} Grettir er ósjúkur og heill\\
\hline
269&ferð&nvee&Eyjólfur var \textbf{ferðar} Hólum þar til er leið páskavika mjög\\
\hline
270&ferð&nvee&Þann vetur seldu þeir Skúfur og Bjarni bæinn \textbf{ferðar} Stokkanesi og aðrar jarðir þær sem þeir áttu og svo kvikfé og ætluðu að ráðast í brott af Grænlandi\\
\hline
271&ferð&nvfe&Óttar vandaði um við Ingólf og bað hann létta af komum \textbf{ferða} \\
\hline
272&ferð&nvee&Þó að við nokkurn liðsmun sé að eiga þá er betra að láta líf sitt við sæmd ef þess verður auðið heldur en þola aðgerðalaust þvílíka \textbf{ferðar} \\
\hline
273&ferð&nvee&svo og ef þér viljið nokkura menn hafa héðan til þessar \textbf{ferðar} þá mun yður það heimult og allan farargreiða þann er þér viljið Þorsteini til segja\\
\hline
274&ferð&nvee&» « Svo skal vera \textbf{ferðar} \\
\hline
275&ferð&nveeg&Þorgils kvaðst eigi til \textbf{ferðarinnar} að leggja sig í hættu við illmenni\\
\hline
276&ferð&nvee&Ólafs konungs Gilli lögsögumaður \textbf{ferðar} Gilli lögsögumaður\\
\hline
277&ferð&nvee&og fékk honum umboð sitt að skipa jarðir þær er Egill átti í Sogni og \textbf{ferðar} Hörðalandi og bað hann selja ef kaupendur væru til\\
\hline
278&ferð&nvfe&þá er Gellir Þorkelsson fékk leyfi að fara til \textbf{ferða} svo sem fyrr var ritið og var hann þá með Ólafi konungi og undi illa ófrelsi því er hann skyldi eigi fara ferða sinna þannug er hann vildi\\
\hline
279&ferð&nveeg&Nú fór hann \textbf{ferðarinnar} uns hann kemur suður í Rómaborg\\
\hline
280&ferð&nveeg&» Bolli svarar \textbf{ferðarinnar} \\
\hline
281&ferð&nvee&Bjóst \textbf{ferðar} ferðar\\
\hline
282&ferð&nvee&hvort hann ætlar að bændur muni vilja taka við honum að konungi \textbf{ferðar} \\
\hline
283&ferð&nvee&Og þann sama dag er til boðsins skyldi koma búast þeir \textbf{ferðar} \\
\hline
284&ferð&nvee&Álfur tók þakksamlega við gjöfinni « og má hér gera mér af loðkápu » og bað Egil þar koma til \textbf{ferðar} er hann færi aftur\\
\hline
285&ferð&nvee&Kvað hún Kotkel og konu hans og sonu gera sér óvært í fjárránum og fjölkynngi en hafa mikið traust af Hallsteini \textbf{ferðar} \\
\hline
286&ferð&nvee&Er það helst við orði manna að gylfrum gangi \textbf{ferðar} \\
\hline
287&ferð&nvee&þeirra tveggja skipsagna er engi maður hafði af \textbf{ferðar} \\
\hline
288&ferð&nvee&og fyrir það annað að eg sé reyndar er reiði gengur af að eigi byrjar mér að gera Tófa líftjón þá er nú svo komið að eg mun fara með yður ef þér viljið og með öngum fjölda liðs ef þér kallist þá lausir vera og sýknir ef konungur sér mig og eg kem á \textbf{ferðar} fund\\
\hline
289&ferð&nvee&« Ekki þykir mér það \textbf{ferðar} en mín að vera foringinn þessarar ferðar\\
\hline
290&ferð&nvee&Honum þótti mikið er þeir skildu og öllum þótti mikil \textbf{ferðar} fráför\\
\hline
291&ferð&nvee&En þó mun eg því orði \textbf{ferðar} lúka að mér þykir þú Bolli hafa komið merkilegastur maður af Íslandi um mína daga\\
\hline
292&ferð&nvee&Það er að segja frá för Finns að hann hafði skútu og á nær þremur tigum \textbf{ferðar} en er hann var búinn fór hann ferðar sinnar til þess er hann kom á Hálogaland\\
\hline
293&ferð&nvee&« Lítið ætla eg þeim um það bræðrum að gera þetta til \textbf{ferðar} sér\\
\hline
294&ferð&nvee&Gísli kvaðst eldur vera mjög \textbf{ferðar} ófriði og væri sér mál af að láta en kvað þó ærna nauðsyn á vera að allir drýgðu dáð og veittu honum\\
\hline
295&ferð&nvee&hinn háva \textbf{ferðar} og réðst Þorkell til ferðar með honum því að hann var þá búinn áður að fara í hernað\\
\hline
296&ferð&nvee&» Gunnar býr sig og konu sína og höfðu með sér lausafé slíkt er þau máttu með komast \textbf{ferðar} \\
\hline
297&ferð&nvee&En þó gerist nú það miklu meira vandmæli en fyrr hefir verið því að vér höfum hér til náð í friði að sitja af útlendum höfðingjum en nú spyrjum vér hitt að Noregskonungur ætli að herja á hendur oss og er mönnum þó grunur \textbf{ferðar} að Svíakonungur muni og til þeirrar ferðar ráðast\\
\hline
298&ferð&nvee&» « Hér munum við \textbf{ferðar} leita\\
\hline
299&ferð&nvee&Bjóst \textbf{ferðar} þá til ferðar með honum\\
\hline
300&ferð&nvee&Á öndverðan vetur átti Guðmundur för \textbf{ferðar} Hvítárvöllu\\
\hline
301&ferð&nvee&Þorbjörn selur lendur sínar og kaupir skip er stóð uppi í Hraunhafnarósi \textbf{ferðar} \\
\hline
302&ferð&nvee&bauð fyrst Þóri og Arinbirni og húskörlum \textbf{ferðar} og ríkum búendum og var til þeirrar ferðar fjölmennt og góðmennt\\
\hline
303&ferð&nvee&níundi Án hrísmagi \textbf{ferðar} \\
\hline
304&ferð&nvee&Þætti mér eigi fjarri að vér færum til \textbf{ferðar} við hann\\
\hline
305&ferð&nvee&Hví sætir það að þú kveður enga menn aðra til ferðar \textbf{ferðar} en bændur\\
\hline
306&ferð&nvee&» Slíkum orðum mælti hver þeirra og svo Sigfússynir og tóku það ráð allir að ríða í \textbf{ferðar} \\
\hline
307&ferð&nvee&Hrafn reið þaðan til \textbf{ferðar} og spurði að Þorgils var að Stað en Sturla að Staðarhóli\\
\hline
308&ferð&nvee&En um morguninn var Oddur snemma \textbf{ferðar} fótum og bjuggust þeir til ferðar\\
\hline
309&ferð&nvee&En þó að hún svari á annan veg þeirra erindum en þeir mundu vilja þá sjá þeir engan sinn kost til þess að sinni að þeir mundu hana í brott hafa nema hennar vilji væri til \textbf{ferðar} og búast þeir þá ferðar sinnar\\
\hline
310&ferð&nvee&Og er höfðingjar ræddu þetta sín í milli þá þótti þeim bændur mart satt sagt hafa í sinni \textbf{ferðar} \\
\hline
311&ferð&nvee&Þórólfur son \textbf{ferðar} réðst til ferðar með Birni og fékk Skalla-Grímur honum fararefni\\
\hline
312&ferð&nveeg&Skyldi hann hafa hálft eða þriðjung en stundum minna \textbf{ferðarinnar} \\
\hline
313&frændi&nkee&Þeirra \textbf{frænda} getur Þormóður í erfidrápu þeirri er hann orti um Þorgeir\\
\hline
314&frændi&nkee&Hann kom út hingað til Íslands þetta sumar í Blönduósi í \textbf{frænda} \\
\hline
315&frændi&nkee&Það var um haustið þá er skip höfðu komið til Íslands af Noregi að sá kvittur kom yfir að Björn mundi hafa hlaupist á brott með \textbf{frænda} og ekki að ráði frænda hennar\\
\hline
316&frændi&nkee&« Nú vil eg \textbf{frænda} \\
\hline
317&frændi&nkee&» Þá mælti jarl \textbf{frænda} \\
\hline
318&frændi&nkee&sagði það illa reifa mundu \textbf{frænda} \\
\hline
319&frændi&nkee&» Uni fór á fund Höfða-Þórðar og sagði honum sín vandræði um víg Odds sonar síns « og vildi eg hafa þitt liðsinni að rétta mitt \textbf{frænda} \\
\hline
320&frændi&nkee&Það var einn dag er Órækja hafði róið yfir fjörð til Haukadals að skemmta sér að Jón Ófeigsson hafði farið í Arnarfjörð \textbf{frænda} Kúlu til Ísleifs frænda síns\\
\hline
321&frændi&nkee&en \textbf{frænda} er helmingsmunur að meiri er sá er hann vildi hafa\\
\hline
322&frændi&nkee&Það þekktist Þórður \textbf{frænda} \\
\hline
323&frændi&nkee&nema sjálft gengi \textbf{frænda} braut\\
\hline
324&frændi&nkfe&Með því að Hrútur átti að vitja til \textbf{frænda} fjárhlutar mikils og göfugra frænda þá fýsist hann að vitja þess\\
\hline
325&frændi&nkee&« Bræður Kala munu þar sjá til sæmdar er eg em um eftirmál og vígsbætur fyrir \textbf{frænda} \\
\hline
326&frændi&nkee&Þá var liðið frá hingaðburð \textbf{frænda} herra Jesú Kristi tólf hundruð þrír tigir og sjö ár\\
\hline
327&frændi&nkee&Hann reið í Höfða til Teits \textbf{frænda} frænda síns við þriðja mann\\
\hline
328&frændi&nkee&Litlu síðar fer Hrútur á fund Ólafs \textbf{frænda} síns og segir honum að hann vill eigi hafa svo búið við Þorleik og bað hann fá sér menn til að sækja heim Þorleik\\
\hline
329&frændi&nkee&« Þar er þú spurðir eftir um lögmenn þá mun eg þér því skjótt svara að sá er engi í vorum flokki og \textbf{frænda} veit eg von nema Þorkels Geitissonar frænda þíns\\
\hline
330&frændi&nkfe&En er \textbf{frænda} leið sumarið sendi Þórður Andrésson mann með bréfi sunnan til þeirra Brandssona frænda sinna þann er Kraki hét\\
\hline
331&frændi&nkfe&Fór hann með þeim \textbf{frænda} Staðarhól\\
\hline
332&frændi&nkee&Þar bjó Már prestur Þormóðsson frændi hans og Hafliða Mássonar og sagði Grímur honum hvað títt \textbf{frænda} \\
\hline
333&frændi&nkee&mikill \textbf{frænda} og sterkur\\
\hline
334&frændi&nkfe&Eftir það fann hún grafsilfur \textbf{frænda} undir viðarrótum\\
\hline
335&frændi&nkee&« að mér mundi stoða orð mín um einn mann og þann er eigi hefir meira saka til að bera en þessi maður því að hvorki hefir hann drepið frændur yðra né vini og ekki hefir hann neina svívirðing gert yður svo að þér þurfið neinn heiftarhug \textbf{frænda} honum að hafa fyrir það\\
\hline
336&frændi&nkfe&sem þeir Sveinbjörn og Krákur \textbf{frænda} \\
\hline
337&frændi&nkee&« Fékk Skarphéðinn mér í hendur höfuð Sigmundar og bað mig færa þér en eg þorði eigi að gera \textbf{frænda} \\
\hline
338&fundur&nkeeg&» Fara þeir nú og lögðu margir fram hjá honum hesta sína og þótti gaman að reyna fráleik hans svo gropasamlega sem hann sjálfur tók \textbf{fundarins} \\
\hline
339&fundur&nkee&En er Ólafur fór af Staðarhóli og suður til Eyra þá tók hann hest fyrir Snorra \textbf{fundar} Múla því að hann nennti illa að ganga og ríður uns hann kemur til fundar við Árna og tekur hann við honum og lætur hann vera þar á laun\\
\hline
340&fundur&nkee& \textbf{fundar} Þórdísarstaði\\
\hline
341&fundur&nkee&Mun þér það hæfa að hafa eigi alla oss í fangi \textbf{fundar} \\
\hline
342&fundur&nkeeg&þar vildu hvorirtveggju sá \textbf{fundarins} \\
\hline
343&fundur&nkeeg&Hafði Þorsteinn þar grið er hann skyldi selja Þorgilsi til \textbf{fundarins} þessa\\
\hline
344&fundur&nkee&hvort þeir \textbf{fundar} Starkaðarson eða Þorgeir Otkelsson hefðu farið með þann hug til fundar að vinna á Gunnari ef þeir mættu\\
\hline
345&fundur&nkee&Þá skuld mun eg gjalda að sál Þorgils mætti fyrir \textbf{fundar} sakir eigi hart hafa\\
\hline
346&fundur&nkee&Hélt hann þegar til fundar við Arinbjörn sem fyrst mátti hann \textbf{fundar} \\
\hline
347&fundur&nkee&Þeirra fundar getur Þormóður í erfidrápu þeirri \textbf{fundar} hann orti um Þorgeir\\
\hline
348&fundur&nkee&En þaðan í frá kölluðu menn að dvínaði liðveisla \textbf{fundar} við Þorgrím\\
\hline
349&fundur&nkee&Vildi Kálfur og synir Erlings að þeir færu öllu liðinu inn til bæjarins og létu þá skeika að sköpuðu en Þorbergur vildi að fyrst væri með vægð farið og láta boð \textbf{fundar} \\
\hline
350&fundur&nkeeg&Héraðshöfðingjum voru orð send að þeir kæmu til þessarar stefnu og öllum bændum og \textbf{fundarins} \\
\hline
351&fundur&nkee&Betri þótti föður \textbf{fundar} sá kostur að falla í bardaga í konungdómi sínum en ganga sjálfkrafa í þjónustu við Harald konung eða þola eigi vopn sem Naumdælakonungar gerðu\\
\hline
352&fundur&nkee&Var þá Gissur í Noregi og settust þeir þar undir \textbf{fundar} dóm\\
\hline
353&fundur&nkee&Svo gerir hann \textbf{fundar} \\
\hline
354&fundur&nkee&En þeir sýstu það eitt að þeir sættust á víg Snorra og var hann fé bættur og fóru við það í \textbf{fundar} \\
\hline
355&fundur&nkee&stefnið þá til \textbf{fundar} fólkinu\\
\hline
356&fundur&nkee&« Engi er eg íþróttamaður en mikið traust \textbf{fundar} eg undir fótum mínum og brjóstheill er eg og því fær engi tekið mig á rás\\
\hline
357&fundur&nkee&Eftir þetta ríða þeir heim til \textbf{fundar} við biskup\\
\hline
358&fundur&nkee&Aðra skóggangssök sel eg þér á hendur Starkaði er hann hefir höggvið í skógi mínum \textbf{fundar} Þríhyrningshálsum og skalt þú sækja þær sakar báðar\\
\hline
359&fundur&nkee&Mærin var ókát og grét einart um \textbf{fundar} \\
\hline
360&fundur&nkeeg&» Þá sögðu menn að Hrafn gætti eigi miður skógarins en fundarins og \textbf{fundarins} \\
\hline
361&fundur&nkee&Talaði Bjarni þá við Lýting og Blæng og Hróa \textbf{fundar} fékk hann skjótt af þeim slíkt er hann beiddi\\
\hline
362&fundur&nkee&En Áslákur og Árni og nokkurir menn með þeim komust á \textbf{fundar} upp og fóru til fundar við Eystein konung og sögðu honum hvernug Ingi konungur hefði fagnað þeim\\
\hline
363&fundur&nkee&Tjáði síðan hver nauðsyn það mig til þessar \textbf{fundar} dregur ef hann skyldi eigi þykja athlægi eða ómannan\\
\hline
364&fundur&nkee&Það er sagt einn dag er þeir feðgar Höskuldur og Ólafur gengu \textbf{fundar} búð og til fundar við Egil\\
\hline
365&fundur&nkee&Þótti þá sýnt að annar hvor \textbf{fundar} mundi hníga verða fyrir öðrum\\
\hline
366&fundur&nkee&Þau urðu endalok þessa fundar að Þorbrandur féll fyrir Þorgeiri en \textbf{fundar} fyrir Þormóði\\
\hline
367&fundur&nkeeg&En því er Skútu Áskelssonar ekki hér við getið að hann var þá ekki \textbf{fundarins} Íslandi og hafði hann verið utan um hríð\\
\hline
368&fundur&nkee&Böðvar sendi menn til fundar við Pál um veturinn og beiddist lands þess er tæki fjóra tigu hundraða fyrir hennar hönd \textbf{fundar} betur sóma að hún væri nokkurn veginn frá leyst\\
\hline
369&fundur&nkee&Með mér var skógarmaður hans um vetur og reið eg í tún hans um vorið með hann og gekk hann út og reið eg með þann sýknan í burt svo sem verða \textbf{fundar} \\
\hline
370&fundur&nkee&« Þormóður skógarmaður vor er á firðinum einn \textbf{fundar} skipi og skulum vér fara til fundar við hann\\
\hline
371&fundur&nkee&Það var presturinn \textbf{fundar} \\
\hline
372&fundur&nkee&En er sveitarmenn verða \textbf{fundar} varir fara þeir til fundar við Brand biskup og biðja að hann banni Guðmundi presti í brott að ráðast\\
\hline
373&fundur&nkee&Komu þar hvorirtveggju og voru sínumegin \textbf{fundar} hvorir og trúðu hvorigir öðrum vel\\
\hline
374&fundur&nkeeg&Böðvar af Stað kom til \textbf{fundarins} við Sturlu og bauð að fara með honum til fundarins\\
\hline
375&fundur&nkeeg&Vinir Kolbeins fýstu hann mjög að sættast við Sighvat og fóru þá enn menn í milli þeirra og kom svo að fundur var lagiður með þeim í \textbf{fundarins} \\
\hline
376&fundur&nkee&Allmjög eru þér þá mislagðar hendur ef þú varðar mér Ljósavatnsskarð svo að eg megi þar eigi fara með förunautum mínum en þú varðar það eigi hið litla skarðið sem er í milli þjóa þér svo að ámælislaust \textbf{fundar} \\
\hline
377&fundur&nkee&Nú fara þeir til Stokka tuttugu saman og er þeir komu á bæinn mælir Skeggi til mægða við Þorbjörn « en til samfara við Þórdísi dóttur \textbf{fundar} \\
\hline
378&fundur&nkee&Sagði hann að hún \textbf{fundar} ráða ferðum sínum en hvergi lést hann fara mundu\\
\hline
379&fundur&nkee&« Vel er mér við Blund-Ketil því að einn tíma er við Tungu-Oddur deildum á alþingi um þrælsgjöld er dæmdust á hendur honum og fór eg að heimta í foraðsillu veðri og vér þrír saman og komum um nótt til Blund-Ketils og var oss þar allvel fagnað og þar vorum vér \textbf{fundar} \\
\hline
380&fundur&nkee&Bað Þórarinn Orm sjá fyrir hlut þeirra þannig sem hann \textbf{fundar} \\
\hline
381&fundur&nkee&Sel heldur gullhring minn til skulda en eigi \textbf{fundar} \\
\hline
382&fundur&nkee&En hann fór sjálfur til \textbf{fundar} og sat þar um jólin\\
\hline
383&fundur&nkee&Hittu þeir Ásgrím Gilsson er kallaður var baulufótur \textbf{fundar} \\
\hline
384&fundur&nkee&gerðu þá ferð sína til \textbf{fundar} við konung\\
\hline
385&fundur&nkee&» Leifur hljóp þegar upp og \textbf{fundar} og gekk á brott til fundar við Gilla\\
\hline
386&fundur&nkee&Helga dóttir Fróða jarls átti sér fóstru framsýna og fór hún með \textbf{fundar} \\
\hline
387&fé&nhee&Þrjú ár eða fjögur voru þeir í siglingum til \textbf{fjár} og áttu þá auð fjár\\
\hline
388&fé&nhee&« Vildi eg \textbf{fjár} \\
\hline
389&fé&nhee&Hún var systir \textbf{fjár} að Hlíðarenda\\
\hline
390&fé&nhee&Hún var gift þeim manni er Þorvarður \textbf{fjár} og bjuggu þau í Görðum þar sem nú er biskupsstóll\\
\hline
391&fé&nhee&Því máli var eigi fjarri tekið en þó sagði Ósvífur að það mundi \textbf{fjár} kostum finna að þau Guðrún voru eigi jafnmenni\\
\hline
392&fé&nhee&En Barði sagði að hann var engi auðmaður né þeir bræður eða frændur \textbf{fjár} « og eigi munum vér biðja oss fjár til bóta\\
\hline
393&fé&nhee&Muntu heyrt hafa þar fyrr um rætt að eg kallaði mér arf Bjarnar hölds er Berg-Önundur bróðir þinn hélt fyrir \textbf{fjár} \\
\hline
394&fé&nhee&Og er hún frá það að Þorsteinn var látinn en faðir hennar andaður þá þóttist hún þar enga uppreist fá \textbf{fjár} \\
\hline
395&fé&nhee&Að vísu ætla eg að þessir sveinar séu minnar ættar en þó hefi eg eigi séð slík heljarskinn sem sveinar þessir \textbf{fjár} \\
\hline
396&fé&nhee&kvað það maklegast að það fé færi þeim til sáluhjálpar er aflað höfðu og til þeirrar kirkju er bein þeirra voru að \textbf{fjár} \\
\hline
397&fé&nhee& \textbf{fjár} er honum þótti nokkuð mannsmót að og vildi hann þýðast\\
\hline
398&fé&nhee&Vil eg biðja \textbf{fjár} ef þú vilt gifta mér hana\\
\hline
399&fé&nhee&Eg vildi að þú keyptir okkur \textbf{fjár} tvö en þú héldir knerri okkrum í kaupstaði sem þú ert vanur\\
\hline
400&fé&nhee&Stóð Þráinn í miðjum dyrum en þeir stóðu til \textbf{fjár} hvor Víga-Hrappur og Grani Gunnarsson\\
\hline
401&fé&nhee&Hún sýndist honum ríða grám hesti og býður honum með sér að fara til síns innis og það þekkist \textbf{fjár} \\
\hline
402&fé&nhee&Spyr Þórir ef mönnum sé nokkur hugur \textbf{fjár} að ganga upp á land og fá sér fjár\\
\hline
403&fé&nhee&Hann spurði hvort þeir þættust ekki þurfa \textbf{fjár} er starfaði fyrir þeim « vildi eg gjarna fara með ykkur\\
\hline
404&fé&nhee&Og fyrr en Þorgils kæmi heim af þingi hafði hann eigi minna fé þegið en átta tigi hundraða af vinum sínum og frændum en margir buðu honum heim úr öllum sveitum bæði norðan og \textbf{fjár} \\
\hline
405&fé&nhee&Gustur hét sauðamaður \textbf{fjár} í Tungu\\
\hline
406&fé&nhee&lést honum mundu kaup fá svo honum hugnaði \textbf{fjár} \\
\hline
407&fé&nhee&Þessir víkingar drápu menn \textbf{fjár} voru nálega berserkir\\
\hline
408&fé&nhee& \textbf{fjár} helming fjár þess og Ásgerði konu sinni\\
\hline
409&fé&nhee&Flosi tók þá fésjóð af \textbf{fjár} sér og kvaðst vildu gefa Hallbirni\\
\hline
410&fé&nhee&Hann var systrungur \textbf{fjár} að frændsemi og var hraustur maður og skáld gott og átti góða kosti fjár og gildur maður fyrir sér\\
\hline
411&fé&nhee&Græðist nú svo skjótt féið að hann á einn ferjuna og heldur nú milli Miðfjarðar og Stranda hvert \textbf{fjár} \\
\hline
412&fé&nhee&Vigdís var meir gefin til fjár \textbf{fjár} brautargengis\\
\hline
413&fé&nhee&Nú sigla þeir í haf og komu til Eiríksfjarðar skipi sínu heilu og voru þar um \textbf{fjár} \\
\hline
414&fé&nhee&Hann var þá hrymdur mjög og var hún til fjár \textbf{fjár} \\
\hline
415&fé&nhee&Þótti Gunnar unnið hafa mörgum manni hið mesta frelsi í drápi blámannsins en jarl þóttist hafa fengið hina mestu sneypu er hann missti blámann sinn en fékk ekki \textbf{fjár} \\
\hline
416&fé&nhee&Og er þeir voru búnir til ferðar þá þakka þeir henni þarvist sína og allan velgerning þann er hún hafði þeim \textbf{fjár} \\
\hline
417&fé&nhee&Þórður átti systur \textbf{fjár} er Auður hét\\
\hline
418&fé&nhee&Þetta sumar var og heitið Þorgrími Helgasyni Arndísi dóttur Þórðar Skeggjasonar af Skeggjastöðum og voru brullaupin bæði saman að Hofi og var veitt með hinu mesta \textbf{fjár} \\
\hline
419&fé&nhee&Alls drápu þeir nær hundrað \textbf{fjár} og tóku þar ógrynni fjár og komu aftur um vorið við svo búið\\
\hline
420&fé&nhee&þá var þeim nú hálfu meira kapp \textbf{fjár} því að vera sem firrst Væringjum því að óvígður\\
\hline
421&fé&nhee&Þeir bræður í \textbf{fjár} eigi öll mál með sér um fjárfar\\
\hline
422&fé&nhee&Liðu veturnætur og fannst ekki \textbf{fjár} \\
\hline
423&fé&nhee&Þú ert kallaður maður fégjarn en mér er lítið \textbf{fjár} að missa og verður margur maður meira að vinna til minna fjár\\
\hline
424&fé&nhee&og sagði að þeir mundu fá virðing mikla af konungi ef þeir vildu hann þýðast \textbf{fjár} \\
\hline
425&fé&nhee&Hann lét fara með sér Ingvildi fagurkinn og gerði hann það til skapraunar við hana en eigi fyrir ræktar \textbf{fjár} \\
\hline
426&fé&nhee&Lauk þar svo er skotið var til \textbf{fjár} úrskurðar en Þorsteinn tók við varðveislu fjár þess er faðir hans hafði átt\\
\hline
427&fé&nhee&Kom þá þar til Böðvar Þórðarson og Sturla mágur hans og sátu menn úti á velli fyrir sunnan hús og var rætt um \textbf{fjár} \\
\hline
428&fé&nhee&Síðan var gert brúðlaup \textbf{fjár} og fer hún heim á Mel og á hún þar hjá\\
\hline
429&fé&nhee&Fór hann síðan þangað er skip þau voru er til Noregs ætluðu og voru öll burtu nema eitt og var þar ráðinn fjöldi manna með miklum \textbf{fjár} \\
\hline
430&fótur&nkfe&Þá sáu heimamenn \textbf{fóta} að í óefni var komið með þeim en þeir þorðu eigi til að fara vopnlausir\\
\hline
431&fótur&nkfe&Hallfreður var enn með hirðinni og var konungur þó færri til hans en áður en þó bætti hann þetta víg fyrir \textbf{fóta} \\
\hline
432&fótur&nkfe&Skyldi þar setja niður hæla þá er höfuð var \textbf{fóta} öðrum enda\\
\hline
433&fótur&nkfe&Og er þeir komu norður fyrir Dumbshaf kom maður af landi ofan og \textbf{fóta} í ferð með þeim\\
\hline
434&hestur&nkeeg&gengu einn dag til hrossa sinna og er þeir komu í afréttina til \textbf{hestsins} þá sakna þeir hestsins og leituðu víða og fundu hann um síðir undir einum klett stórum\\
\hline
435&hestur&nkeeg&Guðmundur bað leita hestsins \textbf{hestsins} \\
\hline
436&hestur&nkfe&« Ekki bar eg það fram er ei mættir þú vel neyta og ætla eg ei því verr að þau undur beri fyrir þig að þú sért brátt í helju og víst mun þetta \textbf{hesta} furða vera\\
\hline
437&hestur&nkee&Hleypur hann heim í Flatatungu \textbf{hests} sagði bónda fundinn og biður hann skunda að veita lið Þórði\\
\hline
438&hestur&nkee&Þenna dag er Sighvatur reið úr Reykjaholti reið Valgarður út á Mýrar að biðja hesta norður til \textbf{hests} eftir viðum\\
\hline
439&hestur&nkfe&Reið Sturla \textbf{hesta} lötum hesti er Álftarleggur var kallaður\\
\hline
440&hestur&nkee&Síðan gekk Auður brott og til hests og hljóp á \textbf{hests} og reið heim eftir það\\
\hline
441&hestur&nkfe&» Fór hann þá upp í Þórólfsfell og synir hans og bundu þar hey \textbf{hesta} fimmtán hesta en á fimm hestum höfðu þeir mat\\
\hline
442&hestur&nkee&Rindill svaraði \textbf{hests} \\
\hline
443&hestur&nkfeg&Hann hafði \textbf{hestanna} höfði sem biskupsmítur væri\\
\hline
444&hestur&nkee&Rindill svaraði \textbf{hests} \\
\hline
445&hestur&nkfeg&Hún reið suður yfir Sælingsdalsheiði og nam eigi \textbf{hestanna} fyrr en undir túngarði að Laugum\\
\hline
446&hestur&nkee&Vildi hann þá leita til hests síns \textbf{hests} \\
\hline
447&hestur&nkeeg&Í því hljóp að Oddur skeiðkollur og höggur til \textbf{hestsins} og stefnir á fótinn en Höskuldur brá undan fætinum og fram á hestshálsinn og kom á síðu hestsins\\
\hline
448&hlutur&nkfe& \textbf{hluta} kann eg fýsa\\
\hline
449&hlutur&nkfe&En er hann þóttist fullkominn til ríkis af styrk landsmanna þá gerðist hann svo harður og frekur við landsfólkið að menn þoldu honum eigi og drápu Þrændir sjálfir hann og hófu þá til \textbf{hluta} Tryggvason er óðalborinn var til konungdóms og fyrir allra hluta sakir vel til höfðingja fallinn\\
\hline
450&hlutur&nkfe&En þaðan frá var honum svo mikil hræðsla í brjósti að hann mátti hvorki með haldkvæmd njóta \textbf{hluta} fyrir ugg og ótta slíkra hluta\\
\hline
451&hlutur&nkfe&Ábóti hét að leggja til \textbf{hluta} með þeim en bað Ögmund eigi halda vini sína til rangra hluta með ofkappi því að þess er von að Sæmundur vilji það eigi hafa\\
\hline
452&hlutur&nkeeg&og munu þeir báðir verða gildir bændur \textbf{hlutarins} munu þeir báðir verða gildir bændur\\
\hline
453&hlutur&nkfe&Hjarðarinnar hafði gætt ánauðigt fólk \textbf{hluta} hafði gætt ánauðigt fólk\\
\hline
454&hlutur&nkfe&Hann var faðir Döllu er átti Ísleifur er síðan var biskup í \textbf{hluta} \\
\hline
455&höfuð&nhee&Þorgerður svarar máli hans og sagði eigi spara þurfa að vinna ógrunsamlega að við \textbf{höfuðs} \\
\hline
456&höfuð&nheeg&Barði var í skóginum og snertu eina \textbf{höfuðsins} þeim\\
\hline
457&höfuð&nhee&Hljóp hann þá í hamraskarð eitt og vildi eigi renna því að hann \textbf{höfuðs} liðið allt\\
\hline
458&höfuð&nhee&Þorgeir var fyrir búi þeirra bræðra í Reykjarfirði og reri jafnan til fiska því að þá voru firðirnir fullir af \textbf{höfuðs} \\
\hline
459&höfuð&nhee&að hann hefði því komið á \textbf{höfuðs} \\
\hline
460&höfuð&nhee&og varð það aldrei að þeir fengju færi \textbf{höfuðs} honum\\
\hline
461&höfuð&nhee&Höfuð Grettis lögðu þeir í salt í útihúsi því er Grettisbúr var kallað þar í \textbf{höfuðs} \\
\hline
462&höfuð&nheeg&Menn \textbf{höfuðsins} fóru til og vildu hjálpa honum en er þeir tóku uppi til höfuðsins þá trað hún fótleggina svo að nær brotnuðu\\
\hline
463&höfuð&nhee&Þessi tíðindi spurði Hákon jarl og lét dæma Kol útlagan um allt ríki sitt og lagði fé \textbf{höfuðs} höfuðs honum\\
\hline
464&höfuð&nhee&Nú er það sagt að Þórður hjó til \textbf{höfuðs} og beit af honum þjóhnappana og fellur Björn þá\\
\hline
465&hönd&nvee&Tók hann við Haraldi feginsamlega og \textbf{handar} föruneyti\\
\hline
466&hönd&nvfe&Liðsmunur var allmikill og þó féll meir lið \textbf{handa} Aðils\\
\hline
467&hönd&nveeg&» Gunnsteinn svarar \textbf{handarinnar} öngvu\\
\hline
468&hönd&nvfe&Sá hét Ásmundur er fyrir þeim var og keypti til \textbf{handa} og kvaðst allt mundu út gjalda\\
\hline
469&hönd&nvfe&var tilkomin \textbf{handa} og óðalborin í allar ættir en tiginborin fram í kyn\\
\hline
470&hönd&nvfe&Skúfur gerði Þorkatli í kunnleika að hirðmaður Ólafs konungs var þar á skipi sá er Þormóður \textbf{handa} \\
\hline
471&hönd&nvfe&Þá hljóp konungurinn fram með brugðnu sverði og þegar hann kemur í höggfæri við Hlégunni þá höggur hann til hennar með sverðinu og kemur höggið á hálsinn og hjó hann af henni höfuðið og féll það \textbf{handa} \\
\hline
472&hönd&nvfe&Og mun hans reiði á liggja og muntu hana hafa ef þú vilt svo margs manns blóði út hella um þessa sök en líkast ef þú lætur fyrirfarast þetta \textbf{handa} þessari hátíðinni um friðinn að guð og hinn heilagi Jóhannes sjái þér hlut til handa um þetta mál\\
\hline
473&hönd&nvfe&Þar kom til \textbf{handa} Óttar svarti og beiddist að ganga til handa Ólafi konungi\\
\hline
474&hönd&nvfe&En það var sem víða finnast dæmi til að þá er landsfólkið verður fyrir hernaði og fær eigi styrk til \textbf{handa} þá játa flestir öllum þeim álögum er sér kaupa frið í\\
\hline
475&hönd&nvfe&nefni eg fyrst til þess Jón Loftsson er dýrstur maður er á landi þessu og allir skjóta sínum málaferlum \textbf{handa} \\
\hline
476&hönd&nvfe&Skyldu þeir biðja \textbf{handa} til handa konungi en Haraldur synjaði\\
\hline
477&hönd&nvee&Þeir Ljótur og Skafti fundust vinir og töluðust við og segir Ljótur honum allan atburðinn um viðskipti þeirra Guðmundar « og ef við megum semja mál okkar mun eg ekki til þess taka og förum við að ræða við \textbf{handar} \\
\hline
478&hönd&nvfe&» Gekk Eyvindur þá fram og Álfur bróðir \textbf{handa} með honum og hjuggu til beggja handa og létu sem óðir eða galnir væru\\
\hline
479&hönd&nvfe&Árni Auðunarson gekk fram með Sighvati og hjó til beggja handa \textbf{handa} \\
\hline
480&hönd&nvfe&og \textbf{handa} \\
\hline
481&hönd&nvfe&« Efna vil eg Steinar liðsemd við þig þá er eg hét að veita þér til \textbf{handa} er þú vilt taka þér til handa\\
\hline
482&hönd&nvee&Kom hann til móts við þá og festi trúnað sinn með eiðum við Kolbein og var til þess trauður í \textbf{handar} \\
\hline
483&hönd&nveeg&En Eyjólfur bauðst nú til að sjá skírsluna og kvað auðvelt að þeir vildu enn tefja málið « og skal því meira hug \textbf{handarinnar} leggja eftir að sjá\\
\hline
484&hönd&nvfe&segir hann hafa numið sig í burt af Grænlandi undan Sólarfjöllum « \textbf{handa} Bárði föður mínum með fjölkynngi og ætlar mig sér til handa og frillu\\
\hline
485&hönd&nvee&Sér hann þá til annarrar handar Þorkel standa í höggfæri við \textbf{handar} \\
\hline
486&hönd&nvfe&Á þetta skal hætta ef þú vilt eigi biðja \textbf{handa} mér til handa\\
\hline
487&hönd&nvfe&Hefi eg nú látið skip búa og flutt þangað til mikið fé en jarðir þessar er eg \textbf{handa} hér skaltu varðveita til handa Þorleifi syni mínum\\
\hline
488&hönd&nvfe& \textbf{handa} vildi eg að þú bæðir mér til handa\\
\hline
489&hönd&nvee&Þá drápu þeir hross eitt er laust hljóp með þeim og flógu af skinn og þöndu um Klaufa og bundu hann um þvert bak \textbf{handar} hrossi og snúa við það ofan eftir dalnum og fundu Ásgeir fyrir neðan Vatnsdalsá\\
\hline
490&hönd&nvfe&» Konungur brosti að og kvað \textbf{handa} \\
\hline
491&hönd&nvfe&Hann undi þar eigi og beiddi að Þorgils léti fylgja honum til Guðmundar og segir að hann þóttist það spurt hafa að með Guðmundi var rausn mest \textbf{handa} Íslandi og væri hann honum til handa sendur\\
\hline
492&hönd&nvfe&Síðan reið hann til Bergþórshvols og sagði Bergþóru vígið og öðrum \textbf{handa} \\
\hline
493&hús&nhfe&Þeir fóru ofan eftir Sælingsdal og út fyrir Hvamm og allt \textbf{húsa} Skarfsstaði og fóru sumir til húsa heim og leystu út naut úr fjósi þrettán\\
\hline
494&hús&nhee&En er skammt var af hádegi \textbf{húss} hún við og hleypur heim til húss\\
\hline
495&hús&nhee&Og er þetta er \textbf{húss} heyra þau mannamál og er Eyjólfur þar kominn við hinn fimmtánda mann og hafa áður komið til húss og sjá döggslóðina sem vísað væri til\\
\hline
496&hús&nheeg&Síðan bjuggust þeir til \textbf{hússins} og lögðust niður í seti þar við eldinn\\
\hline
497&hús&nhfe&Þá voru fallnir úr liði Bjarna fjórir menn en þeir margir sárir er eftir \textbf{húsa} \\
\hline
498&hús&nhfe&Þeir höfðu með sér alls konar fénað því að þeir ætluðu að byggja landið ef þeir mættu \textbf{húsa} \\
\hline
499&hús&nhee&Hún settist þá niður utan undir túngarði en sendir Grímkel \textbf{húss} húss að biðja Þorbjörgu þeim griða\\
\hline
500&hús&nheeg&« Nú munum vér eiga þrenn verk fyrir höndum og skaltu Önundur viða heim til \textbf{hússins} oss\\
\hline
501&hús&nhfe&Þar lét hann húsa konungsgarð og gera Maríukirkju \textbf{húsa} \\
\hline
502&hús&nhee&Þá sté hún af baki en bað smalasveininn gæta \textbf{húss} meðan hún gengi til húss\\
\hline
503&hús&nhee&Síðan mælti Sveinn við menn sína \textbf{húss} \\
\hline
504&hús&nheeg&reið um bý þann er ríkur maður átti er Símon \textbf{hússins} \\
\hline
505&hús&nhfe&Þar kom þá að honum entist ekki gæfa þar til og elskaði þá aðra konu er Ásgerður hét og var hún að vistum \textbf{húsa} Kálfskinni og fór Ingimundur þangað oft og varð þeim Þorgerði það að sundurþykki og stökk Þorgerður í burt stundum af því og ofan í Árskóg\\
\hline
506&kona&nvee&er \textbf{konu} fékk vansa í sárafari\\
\hline
507&kona&nvfe&Guðrún gerðist trúkona mikil \textbf{kvenna} \\
\hline
508&kona&nvee&Aron stökk þá til Barðastrandar og var í helli í Arnarbælisdal \textbf{konu} kosti konu þeirrar er bjó í Tungumúla\\
\hline
509&kona&nvfe&Hún er nú \textbf{kvenna} \\
\hline
510&kona&nvee&Guðmundur átti fjölda þingmanna út um Svarfaðardal og náfrændur og fór hann þannig að heimboði haust og \textbf{konu} \\
\hline
511&kona&nvfe&Dóttir hans hét \textbf{kvenna} \\
\hline
512&kona&nvee&Þykir mér Þorkell ekki lítt hafa dregist til \textbf{konu} við mig\\
\hline
513&kona&nvfe&Dóttir Þóris jarls þegjanda heitir \textbf{kvenna} \\
\hline
514&kona&nvfe&Þau áttu dóttur eina \textbf{kvenna} er Helga hét og var allra kvenna fríðust og þótti sá kostur bestur í Fljótsdalshéraði\\
\hline
515&kona&nvfe&Systir \textbf{kvenna} óx upp með honum er Oddný hét\\
\hline
516&kona&nvee&kölluðu þeir hann trételgju og þótti hæðilegt \textbf{konu} ráð\\
\hline
517&kona&nvee&« Gunnhildur kona mín er dóttir Bjarnar og \textbf{konu} \\
\hline
518&kona&nvfe&Þykir mér og þann veg að \textbf{kvenna} verk þetta er þú hefir unnið að eg kalla þig ekki að verra dreng\\
\hline
519&kona&nvee&Brátt er Vermundur kom heim vakti Halli berserkur til \textbf{konu} við Vermund að hann mundi fá honum kvonfang mjög sæmilegt\\
\hline
520&kona&nvfe&Hann var kvongaður og hét Dalla kona hans og var \textbf{kvenna} vænst og kynstór og kvenna högust á alla hluti\\
\hline
521&kona&nvfe&Gunnar spurði hvað \textbf{kvenna} sjá hefði verið\\
\hline
522&kona&nvfe&Og eigi miklu síðar komu þeir sömu menn á Hítarnes til \textbf{kvenna} og segja þetta\\
\hline
523&kona&nvee&En er hann var mettur \textbf{konu} móðir konu hans hann verða í brottu\\
\hline
524&kona&nvee&Hún kenndi hann því að hún var þá með \textbf{konu} Tryggvasyni bróður sínum er Hjalti var þar\\
\hline
525&kona&nveeg&Son minn hefir þú drepa látið og látið koma ógervilegan mér til handa og fyrir þá sök skaltu eiga að sjá þinn son alblóðgan af mínu \textbf{konunnar} \\
\hline
526&kona&nvee&« að hann sé þér engi skapbætir en þú þarft \textbf{konu} heldur að bætt sé um með þér\\
\hline
527&kona&nvfe&» Bjarni mælti \textbf{kvenna} \\
\hline
528&kona&nvee&föður \textbf{konu} að Vatnsleysu\\
\hline
529&kona&nvee&Ólafur lét vel yfir því \textbf{konu} \\
\hline
530&kona&nvee&Þá berjast þeir \textbf{konu} lýkur svo að Ari fellur og lætur líf sitt\\
\hline
531&kona&nvfe&Grímur hét og sonur Ásgríms en Þórhalla \textbf{kvenna} \\
\hline
532&kona&nvfe&og í þeirri ferð fékk hann Gunnhildar dóttur Össurar tota og hafði hana heim með \textbf{kvenna} \\
\hline
533&kona&nvfe&Hann \textbf{kvenna} mesti klerkur\\
\hline
534&kona&nvee&» Þórir kvaðst minnst háttar af þeim « fyrir það að á mig kemur berserksgangur jafnan þá er eg vildi síst og vildi eg bróðir að þú gerðir \textbf{konu} \\
\hline
535&kona&nvee&En er voraði lýsti Egill yfir því fyrir konungi að hann ætlaði í brott um sumarið og til \textbf{konu} og vita hvað títt er um hag Ásgerðar « konu þeirrar er átt hefir Þórólfur bróðir minn\\
\hline
536&kona&nvfe&En vér þykjumst hitt skilja að konungur vill fyrir engan mun þig lausan láta en höfum það fyrir satt að þú munir fátt það er á Íslandi er til skemmtanar þá er þú situr á tali við Ingibjörgu \textbf{kvenna} \\
\hline
537&kona&nveeg&Dregur Þóroddur fram með Þorgilsi en Skafti með Ásgrími \textbf{konunnar} \\
\hline
538&kona&nvee&Þorleifur sagði konungi að hann fýstist út til Íslands og beiddi konung \textbf{konu} að fara þegar að vori\\
\hline
539&kona&nvfe&Og þá er mjög var ráðin sættarstefna með þeim Karli og Ljótólfi þá sagði Ingvildur að seint mundi verða fyllt skarð í vör \textbf{kvenna} ef sjá sætt skyldi takast\\
\hline
540&kona&nvfe&Guðrún var kurteis kona svo að í þann tíma þóttu allt barnavípur það er aðrar konur höfðu í skarti \textbf{kvenna} henni\\
\hline
541&kona&nvfe&Hún var þegar á unga aldri \textbf{kvenna} kurteis\\
\hline
542&kona&nvfe&Mun hann hafa farið um \textbf{kvenna} að leita sér kvenna\\
\hline
543&kona&nvfe&Sú hét Hlégunnur \textbf{kvenna} \\
\hline
544&kona&nvfe&« Veit eg ákafa \textbf{kvenna} en gott mun að hefta vandræði þetta\\
\hline
545&kona&nvfe&Önnur dóttir Ásgeirs hét \textbf{kvenna} \\
\hline
546&kona&nvfe&Hrappur spratt á fætur sem skjótast og þreif öxi sína \textbf{kvenna} \\
\hline
547&kona&nvee&er þú telur til \textbf{konu} fyrir hönd konu þinnar\\
\hline
548&kona&nveeg&« að á skal gera kostinn fyrir bænastaðinn þinn og flutning Skeggja en engan mundi eg \textbf{konunnar} gera ef Ormur hefði sjálfur beðið konunnar\\
\hline
549&kona&nvfe&« Þá viljum vér þverlega þessar \textbf{kvenna} synja ef þér viljið aflaga eftir leita og upp hefja\\
\hline
550&kona&nveeg&« Vita mun eg fyrst hversu mikið silfur er í sjóð þeim er eg hefi \textbf{konunnar} belti mér\\
\hline
551&kona&nvfe&Hún var \textbf{kvenna} fríðust og nær allra kvenna stærst þeirra sem mennskar voru\\
\hline
552&kona&nvfe&Og áður en þeir bræður skildust mælti Ívar að Þorfinnur skyldi þau orð bera Oddnýju Jóansdóttur að hún biði hans og giftist \textbf{kvenna} \\
\hline
553&kona&nvee&Síðan ríður hann á fund Ólafs Höskuldssonar og segir honum þau tíðindi er þar höfðu \textbf{konu} \\
\hline
554&konungur&nkee&« þá vænti eg þess að um þetta ráð spyrjum vér ekki Svíakonung \textbf{konungs} \\
\hline
555&konungur&nkee&Þá sá af landi í bug allra seglanna og bar hvergi í milli svo sem einn garður \textbf{konungs} \\
\hline
556&konungur&nkee&Svo mun eg gera hvort sem þér viljið nokkurn hug á leggja orðsending \textbf{konungs} eða engan\\
\hline
557&konungur&nkfe&Þegar eftir jólin byrjaði Ólafur konungur ferð sína til Upplanda því að hann hafði fjölmenni mikið en tekjur norðan úr landi höfðu engar til hans komið þá um haustið því að leiðangur hafði úti verið um sumarið og hafði þar konungur allan kostnað til \textbf{konunga} \\
\hline
558&konungur&nkee&Hann lét þá og þar sem annars staðar lög þau upp lesa sem hann bauð mönnum þar í landi kristni að halda og lagði við líf og limar eða aleigusök hverjum manni er eigi vildi undirganga kristin \textbf{konungs} \\
\hline
559&konungur&nkee&Hann var í býnum í Björgyn með presti nokkurum og gildraði til ef hann mætti verða skaðamaður Haralds konungs og voru mjög margir menn að þessum ráðum með honum og þeir sumir er þá voru hirðmenn og herbergismenn \textbf{konungs} og þeir höfðu fyrr verið hirðmenn Magnúss konungs\\
\hline
560&konungur&nkee& \textbf{konungs} saga rauða Óleifur hét herkonungur er kallaður var Óleifur hvíti\\
\hline
561&konungur&nkee&Hann gerði verkmenn í byggðina og tók sér þrjá tigu \textbf{konungs} vel búna er riðu heim með honum\\
\hline
562&konungur&nkee&Var hann þar til \textbf{konungs} tekinn sem í öðrum stöðum\\
\hline
563&konungur&nkee&Kærðu þeir það fyrir Þórði og setti hann þær greinir þá niður er voru á milli \textbf{konungs} og þeim bar á\\
\hline
564&konungur&nkee&Fóru þessi einkamál þá aftur til Magnúss konungs \textbf{konungs} \\
\hline
565&konungur&nkee&« er þú níðist \textbf{konungs} drykkju við gamalmenni og hleypur að vændiskonum um síðkveldum en fylgir eigi konungi þínum\\
\hline
566&konungur&nkee&Var hann þá í förum um hríð og varð aldrei glaður \textbf{konungs} \\
\hline
567&konungur&nkee&ráði \textbf{konungs} konungs hárfagra og ætlaði að leggja undir sig landið\\
\hline
568&konungur&nkee&Voru það einkamál þeirra höfðingjanna \textbf{konungs} \\
\hline
569&konungur&nkee&Hálfdan var bróðir Ólafs \textbf{konungs} og Haralds konungs\\
\hline
570&konungur&nkee&» Og af fortölum \textbf{konungs} var það ráð konungs að rjúfa leiðangurinn og gaf þá hverjum leyfi heim að fara og lýsti því að annað sumar skyldi hann leiðangur úti hafa af öllu landi og halda þá til móts við Svíakonung og hefna þessa lausmælis\\
\hline
571&konungur&nkee&upplenskur maður að \textbf{konungs} \\
\hline
572&konungur&nkee&Þú slóst á þig skrópasótt til þess að hellt var í þig mjólk á imbrudögum út \textbf{konungs} Íslandi\\
\hline
573&konungur&nkee&Og eitt sinn er þeir voru á galeiðum við her og vörðu enn konungs ríki þá kom að þeim \textbf{konungs} \\
\hline
574&konungur&nkee&Það er nú kallað Ólafshlið er heilagur dómur konungs var borinn upp af skipi og er það nú í miðjum \textbf{konungs} \\
\hline
575&konungur&nkee&Nú er það vilji vor og samþykki bóndanna að halda þau lög sem þú settir oss hér á Frostaþingi og vér játuðum \textbf{konungs} \\
\hline
576&konungur&nkee&Þar varð hann sóttdauður og er hann heygður \textbf{konungs} Borró\\
\hline
577&konungur&nkee&Var þá enn kistan komin upp mjög úr jörðu og var þá kistan Ólafs konungs spánósa svo sem nýskafin \textbf{konungs} \\
\hline
578&konungur&nkee&En svo mikið hafði Haraldur konungur aukið álög og landsskyldir að jarlar hans höfðu meira ríki en konungar höfðu \textbf{konungs} \\
\hline
579&konungur&nkee&Varð þá ekki fleira til tíðinda það \textbf{konungs} \\
\hline
580&konungur&nkee&Þá er Loftur biskupsson kom utan fór hann til Hákonar konungs og Haraldur Sæmundarson \textbf{konungs} \\
\hline
581&konungur&nkee&er Haraldur konungur hafði haft réðu það að hleypiskip var gert og sent norður til Þrándheims að segja fall \textbf{konungs} og það með að Þrændir skyldu taka til konungs son Haralds konungs\\
\hline
582&konungur&nkee&Þér mæltuð illa og ómaklega í gærkveld til Halldórs vinar yðvars er þér kennduð honum að hann drykki sleitilega því að það var horn Þóris og hafði hann unnið og ætlaði að bera til \textbf{konungs} ef eigi drykki Halldór fyrir hann\\
\hline
583&konungur&nkee&Ívar hét sonur hans er síðan varð göfugur \textbf{konungs} \\
\hline
584&konungur&nkee&« að hásetar Þórðar hafi svarið til \textbf{konungs} í mínu umdæmi en Þórði til friðar\\
\hline
585&konungur&nkee&Hann fylgdi sjálfur því boði og veitti þar styrk og refsing að þar er eigi gengi við \textbf{konungs} \\
\hline
586&konungur&nkee&Þótti honum sér óvænt til undankomu þótt hann freistaði \textbf{konungs} að leynast og fara huldu höfði leið svo langa sem vera mundi áður hann kæmi úr ríki Eiríks konungs\\
\hline
587&konungur&nkee&Þórir bróðir Magnúss konungs kom til \textbf{konungs} um haustið með orðsendingum Magnúss konungs svo sem fyrr var ritað\\
\hline
588&konungur&nkee&er þetta gerðist \textbf{konungs} \\
\hline
589&konungur&nkee&Drottning var til \textbf{konungs} vel og svo gerðu aðrir eftir\\
\hline
590&konungur&nkee&Sendi Vigfús þá menn til Þorvarðs að segja honum svo \textbf{konungs} \\
\hline
591&konungur&nkee&að hann hafði \textbf{konungs} veturgestur verið\\
\hline
592&konungur&nkee&Haraldur konungur flýði þá austur í Vík til skipa sinna og fór síðan til \textbf{konungs} á fund Eiríks konungs eimuna og sótti hann að trausti\\
\hline
593&konungur&nkee&Lýkur svo þinginu að Hákon var til \textbf{konungs} tekinn\\
\hline
594&konungur&nkee&Fær Ólafur sér hesta og sækir nú á fund Haralds konungs með sínu \textbf{konungs} \\
\hline
595&konungur&nkee&er Einar kom til jarls þá segir jarl honum allt um skipti \textbf{konungs} og svo það að hann vill liði safna og fara á fund Ólafs konungs og berjast við hann\\
\hline
596&konungur&nkee&Geirmundur \textbf{konungs} við kenndir Hjörleifssonar konungs\\
\hline
597&konungur&nkee&Tjáðu menn þá fyrir jarli hver ófæra honum var í að gera svo mikið hervirki á konungs þegnum og í \textbf{konungs} landi\\
\hline
598&konungur&nkee&Hann hafði átt tuttugu fólkorustur \textbf{konungs} \\
\hline
599&konungur&nkee&Var festur \textbf{konungs} konungs dómur\\
\hline
600&konungur&nkee&« Illa gerir \textbf{konungs} er þú kastar mjög svo kristni þinni\\
\hline
601&konungur&nkee& \textbf{konungs} Gissur byskup\\
\hline
602&konungur&nkee&þá hóf Ólafur konungur upp sama ákall til ríkis í Orkneyjum sem hann hafði haft við Brúsa jarl og beiddi Þorfinn þess hins sama að hann skyldi játa konungi þeim hluta landa er hann átti \textbf{konungs} \\
\hline
603&konungur&nkee&Þeim byrjaði vel og tóku Noreg og er \textbf{konungs} för allfræg\\
\hline
604&konungur&nkee&Héldu þeir liði því út eftir Þrándheimi \textbf{konungs} \\
\hline
605&konungur&nkee&Engi var för skörulegri en sú að máli manna er Gregoríus fór fyrir því að Hákon hafði meir en fjóra tigu hundraða manna en Gregoríus eigi öll fjögur \textbf{konungs} \\
\hline
606&konungur&nkee&Þá hjó Gestur höfuð af Raknari og lagði það við þjó honum \textbf{konungs} \\
\hline
607&konungur&nkee&höggur nú bæði stórt og tíðum og drepur þrjá menn \textbf{konungs} lítilli stundu en hinn er fyrir stóð áður drap einn\\
\hline
608&konungur&nkee&Vænti eg að vér höfum svo um búið í sumar að þorparinn viti hvað hann skal vinna bæði á Skáni og \textbf{konungs} Sjólandi\\
\hline
609&konungur&nkee&Steinn bað hann fá sér hest og sleða með \textbf{konungs} \\
\hline
610&konungur&nkee&Arinbjörn segir að honum var ekki \textbf{konungs} af von um skipti þeirra Eiríks konungs « en ekki mun þig fé skorta Egill\\
\hline
611&konungur&nkee&Ingi konungur gaf grið Nikulási Skjaldvararsyni þá er skip hans var hroðið og gekk hann þá til Inga konungs og var með honum síðan meðan hann \textbf{konungs} \\
\hline
612&konungur&nkee&Og er Egill var á brottu þá kallaði jarl til sín bræður tvo er hvortveggi hét \textbf{konungs} \\
\hline
613&konungur&nkee&Móðir \textbf{konungs} var Ingiríður dóttir Sigurðar konungs sýr og Ástu\\
\hline
614&konungur&nkee&Svo er sagt að eitt sinni er Þorsteinn Hallsson kom úr kaupferð af Dyflinni og var það ekki að konungs vilja né leyfi er hann hafði \textbf{konungs} \\
\hline
615&konungur&nkee&Hrútur er hirðmaður Haralds konungs Gunnhildarsonar og hafði af honum mikla virðing \textbf{konungs} \\
\hline
616&konungur&nkee&Kvistir og limar trésins boðaði afkvæmi hans er um allt land dreifðist og af hans ætt hafa verið jafnan síðan konungar í \textbf{konungs} \\
\hline
617&konungur&nkee&Hann fer nú síðan suður með landi og í Vík austur og þá til Danmerkur og er þá uppi hver peningur fjárins og verður hann þá biðja matar bæði fyrir sig og fyrir \textbf{konungs} \\
\hline
618&konungur&nkee&Veittu Þorsteini margir lið að þessu \textbf{konungs} \\
\hline
619&konungur&nkee&« Það er öllum auðsýnt mönnum að konungur má gera af ráði Óttars það sem hann vill þó eð hann skýrði um kvæði \textbf{konungs} \\
\hline
620&konungur&nkee&Því næst dróst að her \textbf{konungs} öllum megin\\
\hline
621&konungur&nkee&Þá var liðið \textbf{konungs} falli Ólafs konungs Tryggvasonar þrettán vetur\\
\hline
622&konungur&nkee&Fór hann austan um Víkina og \textbf{konungs} frið öllum mönnum nema mönnum Magnúss konungs\\
\hline
623&konungur&nkfe&Þetta var þó raundar Grikkjakonungs eiga og auður sem allir menn segja að þar sé rautt gull húsum \textbf{konunga} \\
\hline
624&konungur&nkee&Þá flaug ör ein er fleinn er kallaður og kom í hönd Hákoni Og er það margra manna sögn að skósveinn Gunnhildar sá er Kispingur er nefndur hljóp fram í þysinum og \textbf{konungs} \\
\hline
625&konungur&nkee&Þessi tíðindi fréttir Haraldur konungur skjótlega \textbf{konungs} \\
\hline
626&konungur&nkee&En \textbf{konungs} skip komust undan og reru inn í fjörðu og hjálpu svo lífi sínu\\
\hline
627&konungur&nkee&Hún var þá enn ekki meir en fertug kona að aldri og þótti kosturinn vera hinn merkilegasti fyrir allra \textbf{konungs} sakir\\
\hline
628&konungur&nkee&Tófi var á vist með Agli Hallssyni um veturinn \textbf{konungs} Íslandi áður Egill fór utan með honum\\
\hline
629&konungur&nkee&þeirra er í jörðu hafa legið lengur en þessi \textbf{konungs} \\
\hline
630&konungur&nkee&Grjótgarður gerði þá njósn til Haralds konungs að eigi mundi í annað sinn vænna að fara að jarli \textbf{konungs} \\
\hline
631&konungur&nkee&Var þá sú veisla aukin og var þá drukkið brullaup Ólafs konungs og Ástríðar \textbf{konungs} með mikilli vegsemd\\
\hline
632&konungur&nkee&Mikið má konungs gæfa \textbf{konungs} \\
\hline
633&konungur&nkee&En ef til \textbf{konungs} kæmu tveir eða þrír þá gaf eg mér ekki um því að eg þóttist eigi uppnæmur fyrir þeim\\
\hline
634&konungur&nkee&Finnurinn hét þannug mjög til hjálpar er Haraldur var sonur hans og Haraldur bað honum eirðar og fékk \textbf{konungs} \\
\hline
635&konungur&nkee&Eru \textbf{konungs} véttinu lamar á bak en hespur fyrir og þar læst með lukli\\
\hline
636&konungur&nkee&síns þá fór hann með það lið er honum vildi fylgja því að allur múgur \textbf{konungs} hljóp upp með einu samþykki að rækja ætt Ingjalds konungs og alla hans vini\\
\hline
637&konungur&nkeeg&Voru margir menn sárir mjög en margir svo mjög mæddir að til einskis voru \textbf{konungsins} \\
\hline
638&konungur&nkee&» Svo kom að Þórir hét ferðinni til konungs og bað þá freista ef Eiríkur konungsson vildi fara með \textbf{konungs} \\
\hline
639&konungur&nkee&Setjum hér síðan frið í millum \textbf{konungs} vorra og herji hvorigir á aðra\\
\hline
640&konungur&nkee&» Síðan fór Ölvir á fund Kveld-Úlfs og sagði honum að konungur var reiður og eigi mundi duga nema annar hvor \textbf{konungs} færi til konungs\\
\hline
641&konungur&nkfe&svo að hver hefir eftir annan tekið \textbf{konunga} frænda og verið einvaldskonungur yfir Svíaveldi og yfir mörgum öðrum stórum löndum og verið allir yfirkonungar annarra konunga á Norðurlöndum\\
\hline
642&konungur&nkee& \textbf{konungs} honum slíkt er hann spurði af viðræðum þeirra Ólafs konungs og svo frá erindislokum\\
\hline
643&konungur&nkee&Bjarnar saga Hítdælakappa Nú \textbf{konungs} mönnum sem uppi voru um daga Ólafs konungs Haraldssonar og hans urðu heimulegir vinir\\
\hline
644&konungur&nkee&spurðust þá fyrir um ferðir Ólafs konungs \textbf{konungs} \\
\hline
645&konungur&nkee&En er Ólafur konungur kemur nær býnum þá hljópu þar fyrir þjónustusveinar til \textbf{konungs} og inn í stofuna\\
\hline
646&konungur&nkee&Þar féll Skopti en Eiríkur gaf grið þeim mönnum er þá stóðu upp \textbf{konungs} \\
\hline
647&konungur&nkee&Fór hann sem skyndilegast norður til \textbf{konungs} því að honum þótti þar vera allt megin En er Ólafur konungur kom í Þrándheim þá varð þar engi uppreist í móti honum og var hann þar til konungs tekinn og settist þar um haustið í Niðarósi og bjó þar til veturvistar og lét þar húsa konungsgarð og reisa þar Klemenskirkju í þeim stað sem nú stendur hún\\
\hline
648&konungur&nkee&Og lýkur hér frásögn þeirri er vér kunnum að segja \textbf{konungs} Þormóði kappa hins helga Ólafs konungs\\
\hline
649&konungur&nkee&Drifu þá til frændur Sveins og tóku Harald höndum og ætluðu að hengja \textbf{konungs} \\
\hline
650&konungur&nkee&« Eigi veit eg \textbf{konungs} auðið verður um það\\
\hline
651&konungur&nkee&Bar það og mjög til er hann fór að biskup trúði að það mundi með sannindum er sagt var \textbf{konungs} jartegnagerð og helgi Ólafs konungs\\
\hline
652&konungur&nkee&En vinir Sigurðar fóru síðan eftir líkinu úr Danmörk sunnan með skip og færðu til Álaborgar og grófu að Maríukirkju þar í \textbf{konungs} \\
\hline
653&konungur&nkee&En konungur brást reiður við þeim sendiboðum og bar hinn sami maður reiði konungs út sem honum hafði borið inn boðin \textbf{konungs} \\
\hline
654&konungur&nkee&Um morguninn gekk Jökull til tals við Búa og \textbf{konungs} \\
\hline
655&konungur&nkee&En þá síðan er menn þóttust verða ósjálfráðir fyrir ríki hans þá leituðu sumir í brott úr \textbf{konungs} \\
\hline
656&konungur&nkee&Þeir fóru síðan inn eftir firði og komu einn dag síðarla inn til \textbf{konungs} og báru þessi erindi jarli og segja allt um ferð Ólafs konungs\\
\hline
657&konungur&nkee&Var Gunnhildur systir \textbf{konungs} \\
\hline
658&konungur&nkee&Erlingur var vitur maður og var \textbf{konungs} og með hans ráði fékk Erlingur Kristínar dóttur þeirra Sigurðar konungs og Málmfríðar drottningar\\
\hline
659&konungur&nkee&Hann var fósturfaðir \textbf{konungs} ráðamaður fyrir liði hans því að konungur var þá á barns aldri fyrst er hann kom til ríkis\\
\hline
660&konungur&nkee&» Sendimenn sneru aftur leið sína þegar um kveldið og komu til Ólafs konungs nær miðri \textbf{konungs} \\
\hline
661&konungur&nkee&sá er \textbf{konungs} sundi var\\
\hline
662&konungur&nkee&En með því að vandamenn hans áttu hlut í þá sefaðist hann og lét vera \textbf{konungs} \\
\hline
663&konungur&nkee&leitaði ef það væri ráð höfðingja eða \textbf{konungs} að tekinn væri til konungs sonur Símonar skálps\\
\hline
664&konungur&nkee& \textbf{konungs} Gissur byskup\\
\hline
665&konungur&nkee&frænda síns og það með að þá var til konungs tekinn \textbf{konungs} og það með að þá var til konungs tekinn í Englandi Haraldur Guðinason og hafði tekið konungsvígslu\\
\hline
666&konungur&nkee&er sett var \textbf{konungs} alþingi\\
\hline
667&konungur&nkee&Þá var Magnús konungur átta \textbf{konungs} \\
\hline
668&konungur&nkee&Þeir Skúfur og Bjarni sakna Þormóðar og þykir þeim eigi örvænt að hann muni vera valdur \textbf{konungs} því að Skúfur hafði heyrt í Noregi konungs átekjur um hefnd eftir Þorgeir Hávarsson\\
\hline
669&konungur&nkee&« feginn orðinn er slíkur drengur hefir komið hingað til \textbf{konungs} með erindi konungs vors er vér erum allir skyldir undir að standa\\
\hline
670&konungur&nkee&sumir \textbf{konungs} markir\\
\hline
671&konungur&nkee&vildu gera frið milli sín og sætt og báðu konunga til þess og fóru þær orðsendingar heldur líklega til \textbf{konungs} \\
\hline
672&konungur&nkee&En er þessi orðsending kom til Jarisleifs konungs þá tók hann ráðagerð við drottningina og aðra höfðingja \textbf{konungs} \\
\hline
673&konungur&nkee&Skildust þeir með kærleik hinum mesta \textbf{konungs} \\
\hline
674&konungur&nkee&Skildust þeir vinir og fór Egill ferðar sinnar og kom aftan dags til hirðar jarlsins Arnviðar og fékk þar allgóðar \textbf{konungs} \\
\hline
675&konungur&nkee&sótti þangað að er merki konungs var \textbf{konungs} \\
\hline
676&konungur&nkee&» Nú fer Hreiðar í brott uns hann kemur \textbf{konungs} Upplönd og tekur Eyvindur við honum eftir orðsending konungs\\
\hline
677&konungur&nkee&Voru það einkamál þeirra höfðingjanna \textbf{konungs} \\
\hline
678&konungur&nkee&En er þessi orðsending kom til Ólafs konungs varð hann reiður mjög og hugsjúkur og var það nokkura daga er engi maður fékk orð af honum \textbf{konungs} \\
\hline
679&konungur&nkee&Um vorið eftir páska byrjaði Brynjólfur ferð sína suður til Björgvinjar á móts við Hákon \textbf{konungs} \\
\hline
680&konungur&nkee&Það var \textbf{konungs} einum misserum og fall Inga konungs og það er brenndur var bær Sturlu í Hvammi\\
\hline
681&konungur&nkee&Þá kvað hann \textbf{konungs} \\
\hline
682&konungur&nkee&Og er hún sér að klæði \textbf{konungs} eru brunnin þá gaf hún honum vel sex alnar af skarlati\\
\hline
683&konungur&nkee&Fjöldi dreif liðs til konungs \textbf{konungs} \\
\hline
684&konungur&nkee&Þá hafði hann verið konungur að Noregi sex vetur og tuttugu en hann var til \textbf{konungs} tekinn einum vetri eftir fall Haralds konungs\\
\hline
685&konungur&nkee&Orminum langa og Orminum skamma og Trananum færðu akkeri og stafnljá í skip Sveins konungs en áttu vopnin að bera á þá niður undir fætur \textbf{konungs} \\
\hline
686&konungur&nkee&létu skírast og veittu konungi svardaga til þess að halda rétta trú en leggja niður blótskap \textbf{konungs} \\
\hline
687&konungur&nkee&En er þeir höfðu tjaldað og um búist þá gengu þeir til \textbf{konungs} við Karl mærska\\
\hline
688&konungur&nkee&ætlandi að vitja með \textbf{konungs} konungs\\
\hline
689&konungur&nkee&það sem margtítt er að þeir skyldu til hins helga Ólafs konungs til vöku \textbf{konungs} \\
\hline
690&konungur&nkee&Álfur sonur \textbf{konungs} og Ingjaldur sonur Önundar konungs\\
\hline
691&konungur&nkee&Hann orti þá Ólafsdrápu og er þetta stef í \textbf{konungs} \\
\hline
692&konungur&nkee&Hann nam land frá Sandeyrará til \textbf{konungs} í Hrafnsfirði\\
\hline
693&konungur&nkee&Var \textbf{konungs} því þingi Knútur til konungs tekinn um allan Noreg\\
\hline
694&konungur&nkee&Og verður þar orusta mikil og að lyktum fær keisari sigur en Danakonungur flýði undan til \textbf{konungs} og fór út í Mársey\\
\hline
695&konungur&nkee&Gauka-Þórir og Afra-Fasti og þeirra sveit öll og hafði hver þeirra mann fyrir sig eða tvo eða sumir \textbf{konungs} \\
\hline
696&konungur&nkee&Þau voru systkin Úlfur jarl faðir Sveins konungs og Gyða móðir \textbf{konungs} \\
\hline
697&konungur&nkee&« Eigi viljum vér Uppsvíarnir að konungdómur gangi úr langfeðgaætt hinna fornu konunga á vorum dögum meðan svo góð föng eru til sem nú \textbf{konungs} \\
\hline
698&konungur&nkee&Og víst máttu vita það að eg tel mér misboðið í vígi Þorgeirs hirðmanns míns og þökk kynni eg \textbf{konungs} yrði hefnt\\
\hline
699&konungur&nkee&Bróðir hans var Úlfur \textbf{konungs} \\
\hline
700&konungur&nkee&» Eyjólfur kvaðst þakka konungi fyrir gjafar sínar og vinmæli « en það álag er hann vill að eg geri þá mun eg til stýra með konungs hamingju \textbf{konungs} \\
\hline
701&konungur&nkee&sagði að hann var þá gamall svo að hann var þá ekki til fær að vera úti \textbf{konungs} herskipum « mun eg nú heima sitja og láta af að þjóna konungum\\
\hline
702&konungur&nkee&« Lítil gersemi er sverð mitt en þó er það mart í \textbf{konungs} húsi að ei er í konungs garði\\
\hline
703&konungur&nkee&bar kveðju \textbf{konungs} Birni og enn fleiri höfðingja « og það með\\
\hline
704&konungur&nkee&Ólafur konungur ræddi þetta mál við Ingigerði og segir að þetta var hans vilji að hún giftist Jarisleifi \textbf{konungs} \\
\hline
705&konungur&nkee&Sigurður til \textbf{konungs} og fór á fund Haralds konungs bróður síns\\
\hline
706&konungur&nkee&Hann kallaðist sonur Ólafs Tryggvasonar og \textbf{konungs} ensku\\
\hline
707&konungur&nkee&Í Haugasundi stendur kirkja en við sjálfan kirkjugarðinn í útnorður er haugur \textbf{konungs} hárfagra\\
\hline
708&konungur&nkee&Og er Þórður óx upp fýsti hann til hirðar Gamla konungs \textbf{konungs} er allra manna var vinsælastur í Noregi af öllum konungum þegar leið Hákon Aðalsteinsfóstra\\
\hline
709&konungur&nkee&Þeir sáu hvar sveit manna fór ofan úr Veradal og höfðu þeir \textbf{konungs} njósn verið og fóru nær því sem lið konungs var og fundu eigi fyrr en skammt var í milli þeirra svo að menn máttu kennast\\
\hline
710&konungur&nkee&Síðan sendir hún menn á fund Geirviðar konungs og bað svo segja honum sín orð að hann \textbf{konungs} annaðhvort gera að unna henni hálfs ríkis og landráða við sjálfan sig eða ella skyldi hann búa sig og sína menn og koma til móts við hana með sinn her í sund þau er heita Síldasund og berjast við hana þar og hefði það þeirra sigur og gagn er meiri gæfu stýrði\\
\hline
711&konungur&nkee&Hún var spök að viti \textbf{konungs} \\
\hline
712&konungur&nkee&Og er það mál manna að yfir einskis manns líki hafi svo margur maður í Noregi jafnhryggur staðið sem Eysteins konungs síðan er andaður var Magnús \textbf{konungs} \\
\hline
713&konungur&nkee&Sveinninn Taðkur brá upp við hendinni og tók hana af honum og höfuðið af konunginum en blóðið konungsins kom á handarstúf sveininum og greri þegar fyrir \textbf{konungs} \\
\hline
714&konungur&nkee&Magnúsi konungi líkar þetta þungt er hann fréttir þetta og mæla margir fyrir honum að eigi sé allsæmilegt þeirra \textbf{konungs} að gera slíkt\\
\hline
715&konungur&nkee&hitti Þórð föður sinn og dvaldist þar um hríð \textbf{konungs} \\
\hline
716&konungur&nkee&Fer Finnbogi hljóðlega og tekur sér herbergi skammt \textbf{konungs} konungs aðsetu\\
\hline
717&konungur&nkee&fluttu þeir sína frændur til kirkna og veittu \textbf{konungs} \\
\hline
718&konungur&nkee&lagður í steinvegginn utar \textbf{konungs} kórinum hinum syðra megin\\
\hline
719&konungur&nkee&» En er jarl hafði því upp lokið að hann mundi fylgja þeim að þessu máli og leggja til þess sinn styrk þá þakkaði Björn honum vel og kvaðst hans ráðum vilja fram \textbf{konungs} \\
\hline
720&konungur&nkee&Voru það lög að þann mann skyldi drepa er \textbf{konungs} mann í konungs herbergi\\
\hline
721&konungur&nkee&Þá fóru menn í milli Ólafs konungs og Eilífs og báðu bændur \textbf{konungs} lengi að þeir legðu þingstefnu milli sín og réðu frið með nokkuru móti\\
\hline
722&konungur&nkee&þá fór hann til Íslands og nam fyrir sunnan \textbf{konungs} \\
\hline
723&konungur&nkee&Hann tók því vænlega \textbf{konungs} \\
\hline
724&konungur&nkee&En er þau komu í víkina var stefnt Borgarþing \textbf{konungs} \\
\hline
725&konungur&nkee&að guð þakki honum góða kveðju og svo starf og torveldi er þú hefir og þínir vinir í ríki þessu að beggja okkarra \textbf{konungs} \\
\hline
726&konungur&nkee&En er hann bauð til ferðar tvo sonu sína þá segir konungur að þeir skulu eigi fara með honum en sveinar vildu þó \textbf{konungs} \\
\hline
727&konungur&nkee&» « Ekki er eg \textbf{konungs} fús\\
\hline
728&konungur&nkee&en Ásmundur stóð í loftsvölum \textbf{konungs} konungi\\
\hline
729&konungur&nkee&Veit eg að vera munu hér með yður þeir menn er ekki munu þykja víglegri \textbf{konungs} velli að sjá en eg er\\
\hline
730&konungur&nkee&svo mikið sem þar var í ábyrgð er konungurinn var sjálfur því að öllum þótti vís von að hann mundi látast fyrir þeim og fá minna hlut í \textbf{konungs} skiptum ef svo yrði sem líklegt mundi þykja fyrir sakir æsku konungs þeirra en harðfengi berserkjanna\\
\hline
731&konungur&nkfe&Hún var allra kvenna fegurst og best að sér orðin um það allt er henni var ósjálfrátt en það er mál manna að henni hafi allt verið illa gefið það er henni var \textbf{konunga} \\
\hline
732&konungur&nkee&fór á fund Sveins konungs og fékk þar góðar viðtökur og töluðu þeir löngum einmæli og kom það upp að lyktum að Finnur gekk til handa Sveini konungi og gerðist hans maður en Sveinn konungur gaf Finni jarldóm og Halland til yfirsóknar og var hann þar til \textbf{konungs} fyrir Norðmönnum\\
\hline
733&konungur&nkee&En er menn komu til kaupa við þá þá vildi engi kaupa \textbf{konungs} \\
\hline
734&konungur&nkee&Móðir Flosa var Ingunn dóttir \textbf{konungs} af Espihóli\\
\hline
735&konungur&nkee&að koma til fundar við Ólaf \textbf{konungs} \\
\hline
736&konungur&nkeeg&báðu fá sér lík konungsins \textbf{konungsins} \\
\hline
737&konungur&nkee&Sigurður var þá til konungs tekinn er hann var þrettán vetra \textbf{konungs} fjórtán en Eysteinn var vetri eldri\\
\hline
738&konungur&nkee&og svo hirðsveitinni og handgengnum mönnum konungs \textbf{konungs} \\
\hline
739&konungur&nkee&« að konungurinn eigi vald \textbf{konungs} að drepa Óttar þegar hann vill þó hann kveði kvæði þetta fyrst og hlýðum vér vel kvæðinu því að oss er gott að heyra lof konungs vors\\
\hline
740&konungur&nkee&En er að þeim \textbf{konungs} setið þá var það af ráðið að þau ráð tókust og fékk Þórður Ísríðar\\
\hline
741&konungur&nkee&« Þú verður sjálfur \textbf{konungs} því að segja hvað þér þykir líklegast hvar eg hafi mest um lönd farið\\
\hline
742&konungur&nkee&» Síðan tók biskup söx og skar af hári \textbf{konungs} og svo að taka af kömpunum\\
\hline
743&konungur&nkee&Em eg ekki víðförull en mér er mikil forvitni \textbf{konungs} að sjá tvo konunga senn í einum stað\\
\hline
744&konungur&nkee&Þá réð fyrir Englandi Haraldur Guðinason \textbf{konungs} \\
\hline
745&konungur&nkee&En í annan stað er þar til að taka \textbf{konungs} að þá er Geirviður son Hróðbjarts konungs var átta vetra gamall tók Hróðbjartur konungur sótt og verður það lítil frásaga því að sóttin leiðir svo til lands að konungurinn andast\\
\hline
746&konungur&nkee&Eigi þorir hann að sigla með höfuðin \textbf{konungs} skipi sínu\\
\hline
747&konungur&nkee&« Meir frýr þú mér Gunnhildur grimmleiks en aðrir \textbf{konungs} \\
\hline
748&konungur&nkee&sér náttból og kom þar þá saman allt lið hans og lágu um nóttina úti undir skjöldum \textbf{konungs} \\
\hline
749&konungur&nkee&« En þótt það sé þá geri eg eigi þann mun \textbf{konungs} Magnúss konungs að eg selji þann mann í hendur þér er hann vill skýla láta\\
\hline
750&konungur&nkeeg&Þá var konungur settur á stól þann er stóð á hauginum og hófu menn hann svo einkum til tignar og gáfu honum þá enn af nýju dýrar presentur og dýrkuðu hann sem þeir höfðu framast föng \textbf{konungsins} \\
\hline
751&konungur&nkee&« Ráð þú þá og farið báðir bræður til hirðarinnar ef ykkur líkar það \textbf{konungs} \\
\hline
752&konungur&nkee&Sölvi var síðan víkingur mikill langa hríð og gerði oftlega mikinn skaða \textbf{konungs} ríki Haralds konungs\\
\hline
753&konungur&nkee&« að siglanda mundi þykja þetta veður fyrir Jaðar ef Erlingur Skjálgsson hefði veislu búið fyrir oss \textbf{konungs} Sóla\\
\hline
754&konungur&nkee&En Norðmenn æptu allir senn og sögðu svo að fyrr skyldi hver falla um þveran annan en þeir gengju til griða við enska \textbf{konungs} \\
\hline
755&konungur&nkee&Magnús konungur skyldi fá \textbf{konungs} dóttur Inga konungs\\
\hline
756&konungur&nkee&Nú er \textbf{konungs} því að segja að þeir bræður komu heim og sögðu fall Sigurðar konungs og Klypps bróður síns\\
\hline
757&konungur&nkee&Víkinni eftir andlát \textbf{konungs} yfir allan Noreg\\
\hline
758&konungur&nkee&Þessi ofrausn gerðist þeim brátt að skaða miklum því að Erlings menn sáu bera höggstaði á \textbf{konungs} \\
\hline
759&konungur&nkee&Magnús konungur hafði varðveitt helgan \textbf{konungs} er hann kom í land\\
\hline
760&konungur&nkee&Þá segja þeir honum öll þau merki \textbf{konungs} þeir höfðu vísir orðið\\
\hline
761&konungur&nkee&» Þórður kveðst engum boðum \textbf{konungs} játta mundu\\
\hline
762&konungur&nkee&Eg mun og halda hinu sama um stjórn og umboð af hans hendi sem áður hafði eg af fyrra \textbf{konungs} \\
\hline
763&konungur&nkee&Hann átti þá Sigríði dóttur \textbf{konungs} systurdóttur Haralds konungs\\
\hline
764&konungur&nkee&vænti þín þangað í kveld \textbf{konungs} \\
\hline
765&konungur&nkee&Og er það mál \textbf{konungs} að eigi hafi verri ferð farin verið í annars konungs veldi með miklu liði og líkaði Eiríki konungi illa við Magnús og hans menn og þótti þeir hafa mjög spottað sig er hann hafði komið í þessa ferð\\
\hline
766&konungur&nkee&Nú vil eg að þú heimtir af Ásgrími syni Úlfs hersis því að öngum gef eg upp mína \textbf{konungs} \\
\hline
767&konungur&nkee&En það var eigi af því undarlegt \textbf{konungs} margir menn komu til konungs úr héruðum en því þótti það nýnæmi að þessi var maður svo hár að engi annarra tók betur en í öxl honum\\
\hline
768&konungur&nkee&Bundu þeir Sölvi þá samlag sitt og sendu orð Auðbirni konungi er réð fyrir Firðafylki að hann skyldi koma til \textbf{konungs} við þá\\
\hline
769&konungur&nkee&En \textbf{konungs} því þingi talaði Ástríður og sagði svo\\
\hline
770&konungur&nkfeg&En er hvortveggi herinn sótti mjög til móts við annan þá gerðu lendir menn njósn úr hvorutveggja liði til frænda sinna og vina og fylgdi það orðsending \textbf{konunganna} að menn skyldu gera frið milli konunganna\\
\hline
771&konungur&nkee&Þá hlóðu þeir og seglum og Jarl segir að hann vill bíða Ólafs konungs « og er meiri von að ófriður sé fyrir \textbf{konungs} \\
\hline
772&konungur&nkee&Þaðan fór hann til Noregs og kom á Sognsæ um \textbf{konungs} skeið og spurði þá að um fall Ólafs konungs\\
\hline
773&konungur&nkee&Þá var þetta kveðið \textbf{konungs} \\
\hline
774&konungur&nkee&nefnd \textbf{konungs} \\
\hline
775&konungur&nkee&Um sumarið eftir gerði Grímur hersir \textbf{konungs} eftir gerði Grímur hersir veislu Auðuni jarli Haralds konungs\\
\hline
776&konungur&nkee&Eyjólfur sagði svo að nú mundu menn una að sættast við Gissur er hann hafði nokkurt \textbf{konungs} beisku bitið og bauð öll mál óskoruð í dóm Hákonar konungs\\
\hline
777&konungur&nkee&Konungur segir svo að hann \textbf{konungs} svo að hann vill eigi spilla gersemum sínum svo að leiða fá menn\\
\hline
778&konungur&nkee&En Svíakonungur ætlaði að Ólafur konungur mundi þar bíða frera og þótti Svíakonungi \textbf{konungs} vert um her Ólafs konungs því að hann hafði lítið lið\\
\hline
779&konungur&nkee&En þeir sögðu fall Ólafs konungs Tryggvasonar \textbf{konungs} \\
\hline
780&konungur&nkee&Eg vil gera að dæmum göfugra manna og flýja land \textbf{konungs} \\
\hline
781&konungur&nkee&en norður snerist Áslákur af Finneyju og Erlendur úr Gerði og þeir lendir menn er norður voru \textbf{konungs} þeim\\
\hline
782&konungur&nkee&Eftir andlát Magnúss konungs undu margir lítt þeir sem höfðu verið vinir hans því að allir unnu \textbf{konungs} \\
\hline
783&konungur&nkee&Það var einn dag að til eyjanna komu skip nokkur og lögðu allnær \textbf{konungs} \\
\hline
784&konungur&nkee&En er bóndafólkið kom til \textbf{konungs} þá var þar Magnús til konungs tekinn yfir land allt\\
\hline
785&konungur&nkee&Fann hann þar Rögnvald jarl Brúsason og mjög marga aðra \textbf{konungs} menn er komist höfðu úr orustu\\
\hline
786&konungur&nkee&Gerðu þeir þá sendimenn upp til Hólmgarðs á fund Jarisleifs konungs með þeim orðsendingum að þeir buðu Magnúsi syni Ólafs konungs hins helga að taka við honum og fylgja honum til Noregs og veita honum styrk til \textbf{konungs} að hann næði föðurleifð sinni og halda hann til konungs yfir landi\\
\hline
787&konungur&nkee&En er Ingiríður móðir Inga konungs gekk frá aftansöng þá kom hún að þar \textbf{konungs} var Sigurður skrúðhyrna\\
\hline
788&konungur&nkee&Sá hét \textbf{konungs} \\
\hline
789&konungur&nkee&friðbrotsmenn \textbf{konungs} hirðmanni konungs\\
\hline
790&konungur&nkeeg&Varð Þórir hundur fyrstur til \textbf{konungsins} að halda upp helgi konungsins þeirra ríkismanna er þar höfðu verið í mótstöðuflokki hans\\
\hline
791&konungur&nkee&En Guðröður ljómi fór á fund \textbf{konungs} og bað hann fara með sér til konungs því að Þjóðólfur var ástvinur konungs\\
\hline
792&konungur&nkee&Ölvir hnúfa og Eyvindur lambi \textbf{konungs} \\
\hline
793&konungur&nkee&Hann var hirðmaður Inga konungs og var gamall og hafði mörgum konungum á höndum \textbf{konungs} \\
\hline
794&konungur&nkee&Skulum vér frændur þínir veita þér styrk til þess að þú komir aldrei síðan í slíkt \textbf{konungs} \\
\hline
795&konungur&nkee&móðurföður Magnúss konungs \textbf{konungs} Magnúss konungs\\
\hline
796&konungur&nkee&Flýðu þá margir af Ólafs mönnum en víkingar æptu þá \textbf{konungs} \\
\hline
797&konungur&nkee&tók þar Eyvind og lét drepa en gaf grið flestum mönnum hans og fóru þeir austur til \textbf{konungs} um haustið og komu á fund Ólafs konungs og sögðu honum frá aftöku Eyvindar\\
\hline
798&konungur&nkee&Konungur sefaðist þá og leitaði að við Bjarna og þeir biskup hvers hann mundi að hafa \textbf{konungs} \\
\hline
799&konungur&nkee&að þeir skyldu hafa þvílíkan hlut ríkis af Gunnhildarsonum sem þeir höfðu áður haft af Hákoni \textbf{konungs} \\
\hline
800&konungur&nkee&Svo mun eg gera hvort sem þér viljið nokkurn hug \textbf{konungs} leggja orðsending konungs eða engan\\
\hline
801&konungur&nkee&En ef þú vilt eigi segja mér og farir þó svo héðan til \textbf{konungs} þá mun mér það illa líka\\
\hline
802&konungur&nkee&En þótt sumt þyki heldur örðigt í orðum Inga konungs til Sigurðar konungs bróður síns þá hefir hann mikið til máls síns í marga \textbf{konungs} \\
\hline
803&konungur&nkee&En Þorsteinn fékk sendimenn til \textbf{konungs} konungs að færa honum skatt þann er Egill hafði sótt til Vermalands\\
\hline
804&kostur&nkfe&« Vildi eg \textbf{kosta} \\
\hline
805&land&nhee&Og er Þorbjörn ætlaði að kasta steininum skruppu honum fæturnir og varð honum á hált \textbf{lands} grjótinu svo að hann féll á bak aftur en steinninn fellur ofan á bringspalir honum og verður honum ósvipt við\\
\hline
806&land&nhfe&» Og þessu kaupa \textbf{landa} \\
\hline
807&land&nheeg&þá dreif til \textbf{landsins} allur múgur landsins og þótti þar traust sitt allt\\
\hline
808&land&nhee&Þá mælti bátmaðurinn \textbf{lands} \\
\hline
809&land&nhee&Þeir dysjuðu þá bræður báða þar í eyjunni en tóku síðan höfuð Grettis og báru með sér og allt það sem þar var fémætt í vopnum og \textbf{lands} \\
\hline
810&land&nhee&Í útláti fengu \textbf{lands} veður hvasst og áföll stór og brotnaði ráin\\
\hline
811&land&nhfe&Hann var kallaður Atli grautur \textbf{landa} \\
\hline
812&land&nhee&Þeir minntu konung oft á það og svo það með að Þórólfur hafði rænt konung og \textbf{lands} og farið með hernaði þar innan lands\\
\hline
813&land&nhee&viðbætur úr \textbf{lands} Óttar mælti þá\\
\hline
814&land&nhfe&Flýðu flestir yfir á Katanes til Þorfinns jarls en sumir flýðu úr Orkneyjum til \textbf{landa} en sumir til ýmissa landa\\
\hline
815&land&nhee&Engi maður mátti fara upp í dalinn með hest \textbf{lands} hund því að það var þegar drepið\\
\hline
816&land&nhee&utan Sigurður Torfafóstri \textbf{lands} \\
\hline
817&land&nhee&Og þá fór hún móti Grímkatli syni sínum \textbf{lands} gömlum því að honum dapraðist sundið þá og flutti hann til lands\\
\hline
818&land&nhee&En þegar er hann \textbf{lands} til lands tók hann að herja og brjóta undir sig landsfólk en beiddi sér viðtöku\\
\hline
819&land&nhee&Hann ætlar að vísa oss á illmenni þessi og hyggur að vér munum eigi nenna \textbf{lands} að hverfa þótt hann komi eigi\\
\hline
820&land&nhfe&Hann var \textbf{landa} sumrum í kaupferðum til ýmissa landa\\
\hline
821&land&nhee&snúa við það aftur til \textbf{lands} og riðu heim síðan og þóttust þeir mikla sneypu fengið hafa\\
\hline
822&land&nhee&Þar hafði Ólafur konungur Tryggvason látið efna til kaupstaðar sem fyrr var \textbf{lands} \\
\hline
823&land&nhfe&« götu yfir hraunið út til \textbf{landa} og leggja hagagarð yfir hraunið milli landa vorra og gera byrgi hér fyrir innan hraunið\\
\hline
824&land&nhfe&Ari var \textbf{landa} gamall þá er Ísleifur biskup andaðist\\
\hline
825&land&nhee&reri innan í móti Erlendur sonur Hákons jarls með þremur \textbf{lands} \\
\hline
826&land&nhee&Erlendur tók og fór milli lands og Grímseyjar \textbf{lands} \\
\hline
827&land&nhee&Eigi em eg nú minni valdsmaður í Noregi en þeir voru þá og gaf konungur þeim yfirsókn \textbf{lands} Hallandi\\
\hline
828&land&nhfeg&Brott fór hann þaðan og suður um Reykjanes og vildi ganga upp \textbf{landanna} Víkarsskeiði\\
\hline
829&land&nhfe&Þar var hólmur lítil skammt \textbf{landa} þeim og færðu þangað föng sín sem þeir komust við um nóttina\\
\hline
830&land&nhee&Hann lagðist þá í eitt sker og skreið þar upp á grjótið og lá þar og vænti þá \textbf{lands} annars en hann mundi þar líf láta því að hann var mjög móður og sár en langt til lands\\
\hline
831&land&nhee&» Bóndi kenndi þá og \textbf{lands} þá til lands\\
\hline
832&land&nhee&Ljótur sér nú hvar komið var og fleygir frá sér sverðinu til þeirra bræðra en ætlar að steypa sér útbyrðis og sér þá ei undanbragð \textbf{lands} \\
\hline
833&land&nhee&Gaf þeim vel byri \textbf{lands} \\
\hline
834&land&nhfe&« Fara til þings og krefja höfðingja liðs og gefa þeim fé \textbf{landa} \\
\hline
835&land&nhee&Þá er Þrándur kom til lands með hann þá var sagt liðsmönnum að hann var \textbf{lands} \\
\hline
836&land&nhee&Hann spurði hvert nafn \textbf{lands} var\\
\hline
837&land&nhfe&Fóru þá sendimenn milli \textbf{landa} og báru sættarboð\\
\hline
838&land&nhee&Sæmundur hét félagi \textbf{lands} suðureyskur\\
\hline
839&land&nhfe&Einn dag er \textbf{landa} leið vorið gekk Sveinn konungur ofan á bryggjur og voru menn þá að að búa skip til ýmissa landa\\
\hline
840&land&nhfe&Þorgils kveðst þess eigi mundi lengi bíða að \textbf{landa} \\
\hline
841&land&nhfe&Hann sigldi fyrir vestan Írland og fékk austanveður \textbf{landa} landnyrðinga og rak þá langt vestur í haf og í útsuður svo að þeir vissu ekki til landa\\
\hline
842&land&nhee&Þeir sögðu víg \textbf{lands} og báðu sér skips inn til lands\\
\hline
843&leið&nvee&Már fer nú leiðar sinnar til þess \textbf{leiðar} hann kemur til Jörundar í Oddbjarnareyjar og lætur góðvættlega\\
\hline
844&leið&nvee&« Þormóður skógarmaður vor er á firðinum einn á skipi og skulum vér fara til \textbf{leiðar} við hann\\
\hline
845&leið&nvee&Féll Ásbjörn dauður \textbf{leiðar} stýrinu\\
\hline
846&leið&nvee&Margrætt var um þetta mál hversu eiðar Glúms mundu vera eða fram \textbf{leiðar} \\
\hline
847&leið&nvee&» Þau urðu andorða og segir Bersi að af hennar ráðum mun illt \textbf{leiðar} \\
\hline
848&leið&nvee&» Þetta sumar kom Þorsteinn út sem sagt var og leið framan til \textbf{leiðar} og ríður Þorsteinn og menn hans til leiðar\\
\hline
849&leið&nvee&« Eigi veit eg hvort \textbf{leiðar} ráð mega um það til leiðar koma en ef eg má nekkverju um ráða þá muntu þangað fara í kveld sem eg fer\\
\hline
850&leið&nvee&Líður til \textbf{leiðar} og ríða menn til leiðar en af leið hvarf Glúmur svo að ekki spurðist til hans\\
\hline
851&leið&nvee&Hann kemur á fund ármanns Sveins konungs þess er Áki hét og bað hann vista nakkvarra bæði fyrir sig og fyrir \textbf{leiðar} \\
\hline
852&lið&nhee&Ísleifur Hallsson vildi ofan ganga og berjast við þá og náði eigi fyrir sínum mönnum og fóru þeir Þorvarður og Önundur til Helgastaða með feng \textbf{liðs} \\
\hline
853&lið&nhee&Þau voru þar \textbf{liðs} leiðinni\\
\hline
854&lið&nhee&Þær þrennar tylftir manna skyldu þar dæma um mál \textbf{liðs} \\
\hline
855&lið&nhee&Þorgils snýr þá upp \textbf{liðs} Langamýri til liðs síns\\
\hline
856&lið&nhee&Hann sendir þegar eftir Konáli \textbf{liðs} Einarsstaði að hann komi til liðs við hann\\
\hline
857&lið&nhee&Sturla beiddi að fé væri tekið fyrir frænda hans en því var eigi játað og fóru sakir í \textbf{liðs} \\
\hline
858&lið&nhee&Broddi var þá \textbf{liðs} tvítugsaldri\\
\hline
859&lið&nhee&« Akra-Þórir gaf mér hafra þessa \textbf{liðs} vori til liðs sér er þú hafðir stefnt honum en nú var markað fyrir féránsdóma og á eg hafrana\\
\hline
860&lið&nhee&Það var um dag einn að Ölkofri kom til \textbf{liðs} og gekk fyrir hann og bað sér liðs\\
\hline
861&lið&nhee&» Kom Ástríður svo orðum sínum og liðveislu að fjöldi liðs varð til með Ástríði að fylgja honum til Noregs \textbf{liðs} \\
\hline
862&lið&nhee&en stundum of veturinn var hann \textbf{liðs} Sunnmæri með Arnviði konungi frænda sínum\\
\hline
863&lið&nhee&Hræðumst vér hann nú \textbf{liðs} ekki er hann er einn síns liðs\\
\hline
864&lið&nhee&Danavirki er svo háttað að firðir tveir ganga í landið sínum megin landsins hvor en milli fjarðarbotna höfðu Danir gert borgarvegg mikinn af grjóti og torfi og viðum og grafið díki breitt og djúpt fyrir utan en kastalar fyrir \textbf{liðs} \\
\hline
865&lið&nhee&Þá sneri mannfallinu á hendur Magnúss konungs mönnum og féllu þessir í öndurðri \textbf{liðs} \\
\hline
866&lið&nhee&Hann skipaði allt í héruðum mönnum í \textbf{liðs} og í sýslur\\
\hline
867&lið&nhee&Réð jarl þá til skipa og lét skera upp \textbf{liðs} \\
\hline
868&lið&nhee&Þá spurði Erlingur skakki að Björn systurson hans barðist við þá Halldór og Gregoríus \textbf{liðs} bryggjum inn og honum var liðs þörf\\
\hline
869&lið&nhee&Gekk hann þá milli \textbf{liðs} og fékk þá engi andsvör þótt hann bæði menn liðs\\
\hline
870&lið&nhee&» Bjarni svaraði \textbf{liðs} \\
\hline
871&lið&nhee&Kantarabyrgis og börðust þar allt til \textbf{liðs} er þeir unnu staðinn\\
\hline
872&lið&nhee&Fá mér nú þrjú hundruð \textbf{liðs} og huggast svo að eg mun sjaldan krefja þig héðan frá liðs\\
\hline
873&lið&nhee&« Hvað spyrðu austan \textbf{liðs} Flosa\\
\hline
874&lið&nhee&Þá kom reyður mikil á land staðarins á Möðruvöllum og átti Eyjólfur einn mjög svo allan \textbf{liðs} \\
\hline
875&lið&nhee&Hafði Þórir einn forráð liðs þess og svo aflan þá alla er fengist í \textbf{liðs} \\
\hline
876&lið&nhee&Þar réðu þeir til \textbf{liðs} og létu eftir þriðjung liðs að gæta skipa\\
\hline
877&lið&nhee&Réðust þeir til Egils er aftur vildu fara til \textbf{liðs} en hitt var meiri hluti liðs miklu er fylgdi Arinbirni\\
\hline
878&lið&nhee&Skeggi er skammhöndungur var kallaður og Óspakur \textbf{liðs} \\
\hline
879&lið&nhee&» Og er þá ber þar fyrir dyrnar þá skýtur Höskuldur spjóti og keyrði fyrir brjóst Sölmundi þar sem þeir fóru með hann en þeir bræður hlupu út úr húsinu og til \textbf{liðs} og ríða til liðs síns\\
\hline
880&lið&nhee&» Helgi mælti og tók þá að styttast \textbf{liðs} \\
\hline
881&lið&nhee&Og er þessi mál komu til þings sat Glúmur \textbf{liðs} \\
\hline
882&lið&nhee&En er Sighvatur varð þess var að Knútur konungur býr herferð á hendur Ólafi konungi og hann vissi hversu mikinn styrk Knútur konungur hafði þá kvað \textbf{liðs} \\
\hline
883&lið&nhee&» Lauk Ólafur konungur svo máli sínu að allir menn gerðu góðan róm að og var það ráðs tekið sem hann vildi vera \textbf{liðs} \\
\hline
884&lið&nhee&En Aðalsteinn konungur safnaði herliði að sér og gaf mála þeim mönnum öllum er það vildu hafa til féfangs sér bæði útlenskum og \textbf{liðs} \\
\hline
885&lið&nhee&En þótt hann hafi þrjú hundruð \textbf{liðs} eða fjögur þá er oss það ekki ofurefli liðs ef vér verðum á einu ráði allir\\
\hline
886&lið&nhee&Vér höfum það skip er mest \textbf{liðs} og best skipað í öllum herinum\\
\hline
887&maður&nkfe&» segir Þorgrímur \textbf{manna} \\
\hline
888&maður&nkfe&Hann háttaði svo ferðinni að hann tók veislur uppi í nánd markbyggðinni og stefndi til sín öllum byggðarmönnum og þeim öllum vendilegast er firrst byggðu \textbf{manna} \\
\hline
889&maður&nkfe&Komu nú tíðindi þessi fyrir Odd hvað Blund-Ketill hefir ráðs tekið og tala menn nú um að hann hafi sýnt sig í mótgangi við \textbf{manna} \\
\hline
890&maður&nkfe&En það sýndist honum óráðlegt og mælti til sinna manna \textbf{manna} \\
\hline
891&maður&nkfe&Gekk Halli nú til sætis síns og sýndi Sigurði \textbf{manna} \\
\hline
892&maður&nkfe&Freysteinn fékk frelsi brátt af orðum Þorsteins og gerði Þorkell það vel og liðuglega því að honum var vel í geði til \textbf{manna} því að hann vissi að hann var góðrar ættar og göfgra manna fram í kyn\\
\hline
893&maður&nkfe&bjuggu tvo knörru mikla og höfðu \textbf{manna} hvorum þrjá tigu manna þeirra er liðfærir voru og um fram konur og ungmenni\\
\hline
894&maður&nkfe&Nú mun eg enn ráð þar til leggja \textbf{manna} \\
\hline
895&maður&nkfe&þeir er til \textbf{manna} voru ráðnir með honum\\
\hline
896&maður&nkfe&En er Guðmundur prestur var í brott búinn frá Svínafelli og var kominn til \textbf{manna} þá mælti hann\\
\hline
897&maður&nkfe&En Halldóra batt um sár hans og sat yfir honum til \textbf{manna} er lokið var bardaganum\\
\hline
898&maður&nkee&« Blási herblástur öllu liðinu undir merkin en það lið sem hér er skjóti á skjaldborg og förum síðan undan \textbf{manns} hæli út yfir mýrarnar\\
\hline
899&maður&nkee&Hann nam land á Hafranesi inn til Þernuness og allt hið \textbf{manns} \\
\hline
900&maður&nkfe&Koma þeir þar í stofu \textbf{manna} \\
\hline
901&maður&nkee&» Þar kom að Skeggi hét að biðja konunnar fyrir \textbf{manns} hönd\\
\hline
902&maður&nkfe&Og er erfið var drukkið fæddi húsfreyja \textbf{manna} sveinbarn\\
\hline
903&maður&nkee&» « Eigi ætla eg \textbf{manns} \\
\hline
904&maður&nkfe&» Og áður Einar hafði lokið sögu sinni var Bergljót kona \textbf{manna} komin á þingið með mikla sveit manna\\
\hline
905&maður&nkfe&» Grettir svaraði \textbf{manna} \\
\hline
906&maður&nkfe&Sturla bað hann ekki efast í því að hann ætlaði sér meira hlut en öðrum mönnum \textbf{manna} Íslandi en mér þykir sem þá séu allir yfirkomnir er þú ert því að eg uggi þig einn manna á Íslandi ef eigi fer vel með okkur\\
\hline
907&maður&nkfe&Fannst \textbf{manna} konungsskipinu fjöldi dauðra manna en eigi fannst lík konungs en þó þóttust þeir vita að hann var fallinn\\
\hline
908&maður&nkfe&Nú er að segja frá Þorkeli Eyjólfssyni að hann ríður til \textbf{manna} og kemur þar margt manna\\
\hline
909&maður&nkee&Máttu þá heldur muna að það mun hæfast og sár það er þú sást \textbf{manns} Ásbirni bróðursyni þínum\\
\hline
910&maður&nkfe&Var þar aukið hundrað manna á búi um nóttina \textbf{manna} \\
\hline
911&maður&nkfe&Og lýkur hér \textbf{manna} Þorleifi að segja\\
\hline
912&maður&nkfe&En er bændur urðu varir við ferð hans gerðu þeir orð Þórði á Hítarnesi og báðu hann fyrir bindast að ráða Gretti af en hann fór \textbf{manna} \\
\hline
913&maður&nkfe&« Skip fer þar inn eftir firðinum og kenni eg ferju Guðmundar eða hverjir munu þar vera eða vitið þér nokkuð til hvort Hrólfur gípur er á \textbf{manna} \\
\hline
914&maður&nkfe&Hann sat einn saman og kom fyrr matur hans \textbf{manna} annarra manna\\
\hline
915&maður&nkee&Gaf Hárekur konungi góðar gjafar að skilnaði og gerðist \textbf{manns} maður og tók veislur af konungi og lends manns rétt\\
\hline
916&maður&nkfe&Margt \textbf{manna} manna í Miðfirði\\
\hline
917&maður&nkfe&Það mega allir sjá hve stór lýti \textbf{manna} mér eru fyrir manna augum en miklu eru þó meiri lýti á mínum hag í guðs augliti\\
\hline
918&maður&nkfe&Þaðan fór hún á fund Leifs bróður síns og bað að hann gæfi henni hús þau er hann hafði gera látið á \textbf{manna} \\
\hline
919&maður&nkfe&En ef þú bíður hans þá verður það þinn skaði og \textbf{manna} er þér fylgja\\
\hline
920&maður&nkfe&Björn fór þá hafna á milli og voru síð dags þar er fram gengu \textbf{manna} tvær af Brenneyjum\\
\hline
921&maður&nkfe&Hann reið til \textbf{manna} við þrjá tigu manna\\
\hline
922&maður&nkfe&Nú mun eg lúka upp gerð okkarri \textbf{manna} \\
\hline
923&maður&nkfe&Oddur Þórarinsson fór af Valþjófsstöðum þann sama vetur þegar fyrir jólaföstu og fór allt hið syðra og kom að jólum í Haukadal til Þóris totts og var þar við honum allvel \textbf{manna} \\
\hline
924&maður&nkfe&« Það er \textbf{manna} \\
\hline
925&maður&nkfe&Blund-Ketill var manna \textbf{manna} og best að sér í fornum sið\\
\hline
926&maður&nkee&Hann var ágætlega búinn og undruðu allir menn hans fegurð og \textbf{manns} \\
\hline
927&maður&nkfe&« Hvað \textbf{manna} ertu\\
\hline
928&maður&nkfe&En er að var hugað þá var mark \textbf{manna} á sauðnum\\
\hline
929&maður&nkfe&En hinir finna Glúm og segja honum að þeir fundu þann mann er \textbf{manna} með spotti og kvaðst heita Margur í Mývatnshverfi en Fár í Fiskilækjarhverfi\\
\hline
930&maður&nkee&Þegar hann kom í veldi \textbf{manns} þá höfðu bændur fyrir safnað og múg manns\\
\hline
931&maður&nkfe&Dreif þá til \textbf{manna} margt manna og gekk upp allt sumarbúið\\
\hline
932&maður&nkfe&» Síðan sagði hún honum hvað þær höfðu talað og fjandmæli \textbf{manna} við hann\\
\hline
933&maður&nkfe&Þeir kvöddu þá Styr og Vermund frændur sína til þessar \textbf{manna} og voru þá saman átta tigir manna\\
\hline
934&maður&nkfe&Játaðist allt fólk undir \textbf{manna} við Harald konung og fengu honum gíslar\\
\hline
935&maður&nkfe&Hann gekk upp um nótt á eyna með lið sitt og kom til bæjar Þorvalds á óvart svo að allir menn voru fyrir í \textbf{manna} \\
\hline
936&maður&nkfe&Þeir voru nær þrír tigir \textbf{manna} og riðu flestir en umrenningar gengu\\
\hline
937&maður&nkfe&« Er eigi því mjög til \textbf{manna} haldið\\
\hline
938&maður&nkee&Fóru þeir síðan alla þá leið þar til er þeir komu undir fjallið \textbf{manns} helli Búa\\
\hline
939&maður&nkfe&Einar Halldórsson með sveit manna og var þeim þar vel \textbf{manna} \\
\hline
940&maður&nkfe&Voru þá kærleikar með þeim öllum \textbf{manna} \\
\hline
941&maður&nkfe&En riðu þó báðir samt norður til \textbf{manna} fyrir páska og höfðu þá mikla sveit\\
\hline
942&maður&nkfe&Sveinn Úlfsson var allra manna fríðastur \textbf{Manna} \\
\hline
943&maður&nkfe&En af tali þeirra kom það upp að Styr fastnaði \textbf{manna} goða Ásdísi dóttur sína og tókust þessi ráð um haustið eftir og var það mál manna að hvortveggja þótti vaxa af þessum tengdum\\
\hline
944&maður&nkfe&En þóttust þeir eigi mega hafa hann þar lengur fyrir skaps sakir og fór hann víða of \textbf{manna} \\
\hline
945&maður&nkfe&Töluðu það sumir menn að vel mundi hafa fallið á með þeim Helgu og Sigurði \textbf{manna} þó kom það ekki mjög á loft fyrir alþýðu manna\\
\hline
946&maður&nkfe&Það er nú að segja að þeir Ófeigur og Þorkell söfnuðu að sér mönnum og bjuggust nú til hólmsins og voru þeir saman fjórir tigir manna og hélt Ófeigur skildi fyrir \textbf{manna} \\
\hline
947&maður&nkfe&haltir og blindir eða á annan veg sjúkir en fóru þaðan heilir \textbf{manna} \\
\hline
948&maður&nkfe&dreif þá til \textbf{manna} mart manna\\
\hline
949&maður&nkfe&Og einn morgun jólanna er breytt \textbf{manna} \\
\hline
950&maður&nkee&á er unnið \textbf{manns} Þorgrími\\
\hline
951&maður&nkfe&Spurði hann og að kært var með þeim Víglundi og Ketilríði og meinaði hann það ekki en Þorbjörgu og sonum hennar þótti það mjög illa \textbf{manna} \\
\hline
952&maður&nkfe& \textbf{manna} er árferð tók að versna og sæði manna brugðust\\
\hline
953&maður&nkee&Sonarsonur \textbf{manns} var Þórir helsingur\\
\hline
954&maður&nkfe&Kom þar þá saman \textbf{manna} þriðja hundraði manna\\
\hline
955&maður&nkfe&Hann var \textbf{Manna} hár\\
\hline
956&maður&nkfe&Þórður bróðir \textbf{manna} gekk þar mest í millum manna og kvað Þorgeiri mjög missýnast er hann gekk í mót sonum sínum í orustu\\
\hline
957&maður&nkfe&Glúmur gekk til \textbf{manna} með hundrað manna og náði eigi nær að tjalda en í fjörbaugsgarði\\
\hline
958&maður&nkfe& \textbf{manna} er nam suður Grindavík\\
\hline
959&maður&nkfe&einskis þykir mér vert sverðið hjá \textbf{manna} \\
\hline
960&maður&nkfe&Jafnskylt var öllum mönnum í lögum þeirra að færa dauða menn til graftrar sem nú ef þeir eru \textbf{manna} \\
\hline
961&maður&nkfe&Ef hann skyldi þar \textbf{manna} höfðingja þjóna vildi hann helst Þorgilsi en betur að þjóna öngvum ef hann mætti kyrr sjást\\
\hline
962&maður&nkee&Hefir Hrafn riðið suður til Hítardals og hafði Sturla komið til \textbf{manns} við hann\\
\hline
963&maður&nkfe&Kenndi þess mjög á um marga Upplendinga að illa hafði líkað \textbf{manna} svo sem fyrr var getið\\
\hline
964&maður&nkfe&Hann bjó \textbf{manna} Þorbrandsstöðum\\
\hline
965&maður&nkfe&Órækja vildi fyrir hvern mun \textbf{manna} Stafaholt en þó varð svo að vera sem Snorri vildi\\
\hline
966&maður&nkfe&Voru þá enn vestanmenn flettir á Jökulsárbakka af fylgdarmönnum \textbf{manna} en sumir voru barðir\\
\hline
967&maður&nkfe&Gengu þeir þá fyrir jarl og kvaddi Flosi hann og allir þeir \textbf{manna} \\
\hline
968&maður&nkfe&Safnaði hann þá liði um alla Vestfjörðu og hafði \textbf{manna} fjórða hundraði manna og fóru suður á sveitir\\
\hline
969&maður&nkfe&Víglundar saga Haraldur hinn hárfagri son Hálfdanar svarta var þá einvaldskonungur yfir Noregi er saga þessi \textbf{manna} \\
\hline
970&maður&nkfe&kvað Órækju lítils hafa virt orð sín um griðagjöf við Klæng og fór hann vestur og flestir \textbf{manna} menn\\
\hline
971&maður&nkfe&tiginna \textbf{manna} sonu\\
\hline
972&maður&nkfe&» Þórður kveðst ekki að því fara og sendi Þórhall yfir í Ás til \textbf{manna} er þar bjó að hann sækti Össur og græddi hann\\
\hline
973&maður&nkfe&Biskup var með konungi um veturinn og hlýddi konungur allmjög á \textbf{manna} sagnir\\
\hline
974&maður&nkfe&Þótti þeim \textbf{manna} getið nema Finnboga síðan hann kom út\\
\hline
975&maður&nkfe&Skopti skyldi \textbf{manna} tíðindi öll jarli þá er þeir voru báðir samt en jarl sagði Skopta tíðindi ef hann spurði fyrr\\
\hline
976&maður&nkfe&En er þeir komu til \textbf{manna} mælti hann til sinna manna\\
\hline
977&maður&nkfe&Þá heyrir hann að mælt er \textbf{manna} honum uppi\\
\hline
978&maður&nkfe&Hann átti höfuðból að Hvoli \textbf{manna} \\
\hline
979&maður&nkfe&Kom þá brátt saman lið mikið og réðu þeir til \textbf{manna} og var þá skorað manntal og var nær fimmtán hundruðum manna\\
\hline
980&maður&nkfe&Jarl var farinn á veiðar að ríkra manna sið \textbf{manna} \\
\hline
981&maður&nkfe&Þeir bræður voru hinir mestu höfðingjar \textbf{manna} \\
\hline
982&maður&nkfe&Þá voru þeir komnir ofan \textbf{manna} skálanum á völlinn\\
\hline
983&maður&nkfe&Falla þeir Einar og Sigurður þar og allir förunautar \textbf{manna} og varð Gísli fimm manna bani en Þorkell þriggja\\
\hline
984&maður&nkfe&Og þá verða varðmenn varir við lið Magnúss \textbf{manna} að þeir fóru þá að bænum og hafði Magnús konungur nær sex tigum hundraða manna en Haraldur hafði fimmtán hundruð manna\\
\hline
985&maður&nkfe&Riðu þeir þá skyndilega í Öxnatungur \textbf{manna} \\
\hline
986&maður&nkfe&Hann var bróðir Þorbjarnar og honum líkastur um alla \textbf{manna} \\
\hline
987&maður&nkfe&Nú siglir Karlsefni í haf og kom skipi sínu fyrir norðan land í Skagafjörð og var þar upp sett skip \textbf{manna} um veturinn\\
\hline
988&maður&nkfe&Voru þar fleira konur \textbf{manna} \\
\hline
989&maður&nkfe&En hinir voru allmargir er áður höfðu þegið af honum vingjafir stórar fyrir því að það var satt að segja frá Knúti konungi að hver er á hans fund \textbf{manna} \\
\hline
990&maður&nkfe&Þeir voru \textbf{manna} skipi þrír tigir manna\\
\hline
991&maður&nkeeg&Sá hafði feld \textbf{mannsins} höfði\\
\hline
992&maður&nkfe&Gekk með honum allt búðarlið \textbf{manna} en Karl sat eftir\\
\hline
993&maður&nkee&Síðan kom þar að Sturla lauk upp gerðinni og mælti \textbf{manns} \\
\hline
994&maður&nkfe&» Um morguninn eftir höfðu þeir torfleik hjá búð Þorbrandssona og þar ganga þeir hjá \textbf{manna} \\
\hline
995&maður&nkfe&Nú býst hann til hrossanna og hefir manskæri mikil á linda og hött á höfði og skjöld á \textbf{manna} \\
\hline
996&maður&nkfe&En suður fyrir Erri hitti hann Vindasnekkjur nokkurar og lagði til \textbf{manna} við þá og fékk sigur\\
\hline
997&maður&nkfe&En Þorkell kaupir land \textbf{manna} Barðaströnd það er í Hvammi hét\\
\hline
998&maður&nkee&Lét jarl þá bera út merki sitt og vopnaðist sjálfur og allir \textbf{manns} menn\\
\hline
999&maður&nkfe&Steinar var allra manna \textbf{manna} og rammur að afli\\
\hline
1000&maður&nkfe&Þá fóru þeir út til \textbf{manna} og er þeir komu þar var Arnkell þar fyrir og mart manna með honum\\
\hline
1001&maður&nkfe&Kona ein vermdi vatn í katli til \textbf{manna} að þvo sár manna\\
\hline
1002&maður&nkfe&Þegar honum þykir von að Bolli muni norðan ríða þá safnar hann mönnum og ætlar að sitja fyrir Bolla og vill nú að verði umskipti um mál \textbf{manna} Helga\\
\hline
1003&maður&nkee&Sigldu þeir suður til \textbf{manns} og fóru upp í Þrasvík til göfugs manns er Skeggi hét og voru með honum mjög lengi\\
\hline
1004&maður&nkfe&Þorsteinn hét maður \textbf{manna} \\
\hline
1005&maður&nkee&En er þeir kenndu skipið \textbf{manns} þegar að umhverfis\\
\hline
1006&maður&nkee&« Það hefir hugur minn að þessi blótguð Svía er nú ganga mestar sögur frá og þeir kalla Frey að þar muni vera reyndar Gunnar bróðir þinn því að þau blót verða römmust er lifandi menn eru \textbf{manns} \\
\hline
1007&maður&nkfe&Og hversu margt sem hér var um talað þá fór það fram að þeir festu \textbf{manna} \\
\hline
1008&maður&nkfe&Sneri Gunnar nú aftur til byggðar og undi illa við sína \textbf{manna} \\
\hline
1009&maður&nkee&Þá er synir Magnúss voru til konunga teknir komu utan úr Jórsalaheimi og sumir úr Miklagarði þeir menn er farið höfðu út með Skopta Ögmundarsyni og voru þeir hinir frægstu og kunnu margs konar tíðindi að \textbf{manns} \\
\hline
1010&maður&nkfe&sagði \textbf{manna} mannskaða þeim er hann hafði látið af Færeyjum « en skattur sá er þeir hafa mér heitið\\
\hline
1011&maður&nkfe&Ei bar hann það nafn af því að hann hefði til \textbf{manna} ætt eða eðli\\
\hline
1012&maður&nkfe&miskviðalaust og með skörulegum flutningi \textbf{manna} \\
\hline
1013&maður&nkfe&Berg-Önundur var í sveit konungs og þeir bræður og höfðu þeir sveit \textbf{manna} \\
\hline
1014&maður&nkfe&Egill hafði eitt langskip mikið og þar \textbf{manna} hundrað manna eða vel svo\\
\hline
1015&maður&nkfe&Hafði Þórir einn forráð liðs þess og svo aflan þá alla er fengist í \textbf{manna} \\
\hline
1016&maður&nkee&Fátt hafði hann manna hjá sér utan konu \textbf{manns} \\
\hline
1017&maður&nkfe&Jarl fór með fá menn en er hann kom á heiðina \textbf{manna} norðan Kastalabryggju þá komu móti honum ármenn tveir með sveit manna og tóku hann og settu í fjötur og síðan var hann höggvinn og kalla enskir menn hann helgan\\
\hline
1018&maður&nkfe&Ásbjörn Illugason fór með Oddi og margt röskra \textbf{manna} \\
\hline
1019&maður&nkfe&Fóru þá menn milli þeirra Órækju og var fundur lagður á Söndum og grið til \textbf{manna} \\
\hline
1020&maður&nkeeg&Fyrir því máttu djarflega sækja til \textbf{mannsins} að guð mun þér bera vitni að það er þín eiga\\
\hline
1021&maður&nkfe&Og litlu síðar kom Sturla í hina mestu kærleika við konunginn og hafði konungur hann mjög við ráðagerðir sínar og skipaði honum þann vanda að setja saman sögu Hákonar konungs föður síns eftir sjálfs hans ráði og \textbf{manna} vitrustu manna forsögn\\
\hline
1022&maður&nkfe&» Þorsteinn fór nú heim með Bjarna til Hofs og fylgdi honum allt til dauðadags og þótti nær \textbf{manna} maki vera að drengskap og hreysti\\
\hline
1023&maður&nkfe&Þórður bróðir \textbf{manna} gekk þar mest í millum manna og kvað Þorgeiri mjög missýnast er hann gekk í mót sonum sínum í orustu\\
\hline
1024&maður&nkfe&En hann var sonur \textbf{manna} í Hraunsási\\
\hline
1025&maður&nkfe&Faðir hans var \textbf{manna} \\
\hline
1026&maður&nkfe&Bar Þorsteinn fram orðsending \textbf{manna} bæði um grið og sættir\\
\hline
1027&maður&nkfe&Hafði jarl \textbf{manna} því ríkra manna hátt að Flosi gekk í þá þjónustu sem Helgi Njálsson hafði haft\\
\hline
1028&maður&nkfe&brýst í móti miklu ofurefli og á honum og hans ráðum liggur reiði Danakonungs og Svíakonungs ef hann heldur þessu \textbf{manna} \\
\hline
1029&maður&nkfe&Kominn var Ófeigur til \textbf{manna} með fimm tigu manna\\
\hline
1030&maður&nkfe&Um vorið eftir kom Þorgrímur vestan með þrjá tigi manna að vitja þessa manns og sömdu þeir Ljótólfur svo og Þorgrímur að Ljótólfur galt fé fyrir það hann hafði haldið Skíða um veturinn og kölluðust sáttir og hafði Þorgrímur með sér þrælinn að sinni því að Ljótólfur hafði fámennur heima \textbf{manna} \\
\hline
1031&maður&nkee&Eru og fyrirmenn \textbf{manns} eigi annars manns vinir meiri en þínir\\
\hline
1032&maður&nkfe&Eftir Þjóðólfs sögn er fyrst rituð ævi \textbf{manna} og þar við aukið eftir sögn fróðra manna\\
\hline
1033&maður&nkfe&« Gerast mætti það því að nú er Þorvaldur Oddsson kominn í Norðurtungu og er þar nú \textbf{manna} gistingu\\
\hline
1034&maður&nkfe&Leitið enn síðan til \textbf{manna} þá er þessu lýkur\\
\hline
1035&maður&nkfe&Konungur lét illa yfir þeirra ferð en bauð Gunnsteini með sér að vera og segir það að hann skyldi leiðrétta mál Gunnsteins þá er hann mætti við \textbf{manna} \\
\hline
1036&maður&nkfe&Hallgerður varð fegin Gunnari er hann kom heim en móðir hans lagði fátt \textbf{manna} \\
\hline
1037&maður&nkfe&» Eftir það fóru þeir biskup heim til Lækjamóts og dvöldust þar um \textbf{manna} \\
\hline
1038&maður&nkfe&Þeir riðu í Garpsdal og gerðu þar engar óspektir en þar létu þeir eftir nær \textbf{manna} manna\\
\hline
1039&maður&nkfe&En það var ráðið á laun að Skúta skal fá Þórlaugar dóttur \textbf{manna} og skal Skútu svo heima standa hundraðið að Glúmur gyldi því minni heimanfylgjuna en þó vissi það ekki fyrst alþýða manna\\
\hline
1040&maður&nkfe&Gregoríus hafði tvö skip og vel \textbf{manna} tigu manna er hann fékk allar vistir\\
\hline
1041&maður&nkfe&Ólafur konungur var allra manna glaðastur og leikinn \textbf{manna} \\
\hline
1042&maður&nkfe&Brátt sér það \textbf{manna} Ólafi er hann óx upp að hann mundi verða mikið afbragð annarra manna fyrir vænleiks sakir og kurteisi\\
\hline
1043&maður&nkfe&Í Borgarfirði koma þeir til Þorleifs Oddur Sveinbjarnarson og Ólafur \textbf{manna} Borg með sveit manna\\
\hline
1044&maður&nkfe&Tók hann þar undir sig alla eign \textbf{manna} í Borgarfirði\\
\hline
1045&maður&nkee&Nú biður Þorbergur Austmennina verða í ferð með þeim að rannsaka \textbf{manns} \\
\hline
1046&maður&nkfe&Hann sendi og menn til \textbf{manna} og reið suður á sveitir með fimm hundruð manna\\
\hline
1047&maður&nkfe&Letja vil eg þig og alla frændur mína og vini að fara með ófriði \textbf{manna} annarra manna sveitir\\
\hline
1048&maður&nkfe&Refur stefnir heim á bæ \textbf{manna} og er hann kemur fyrir dyr sér hann ekki úti manna\\
\hline
1049&maður&nkfe&Á þessum tímum byggðist allur Breiðafjörður og þarf hér ekki að segja \textbf{manna} þeirra manna landnámum er eigi koma við þessa sögu\\
\hline
1050&maður&nkee&» segir Þórir \textbf{manns} \\
\hline
1051&maður&nkfe&Þórir langi tók skútu og fór með þrjá tigu \textbf{manna} og er lýsti sáu þeir skútur tvær litlar fara fyrir þeim\\
\hline
1052&maður&nkee&» Signý kvaðst annan veg ætla \textbf{manns} \\
\hline
1053&maður&nkfe&örlæti hans né meinlætum því að nakkvað bar \textbf{manna} til jafnan af tilstilli góðra manna að hann fékk því haldið er hann hafði upp tekið\\
\hline
1054&maður&nkee&Allt sýndist Þorsteini athæfi þessa manns merkilegt og mjög hæversklegt \textbf{manns} \\
\hline
1055&maður&nkfe&« Gerið svo til \textbf{manna} sem hann hafi mér allvel trúað\\
\hline
1056&maður&nkfeg&Þeir Skúfur og Bjarni sakna Þormóðar og þykir þeim eigi örvænt að hann muni vera valdur áverkans því að Skúfur hafði heyrt í Noregi \textbf{mannanna} átekjur um hefnd eftir Þorgeir Hávarsson\\
\hline
1057&maður&nkeeg&« Á hitt er að líta herra hvað í er að virða orð \textbf{mannsins} og dyggð mannsins að ganga á vald þitt\\
\hline
1058&maður&nkfe&En Sigurður tók svo langt kaf í brott að hann var fyrr \textbf{manna} landi uppi en þeir hefðu snúið skipi sínu eftir honum\\
\hline
1059&maður&nkfe&Þórir fór út til \textbf{manna} og er það margra manna sögn að hann hafi eigi aftur komið\\
\hline
1060&maður&nkee&Og eigi veit eg hversu trúlegt yður þykir að eg mundi eigi óbúnari en einnhver yðvar að \textbf{manns} Gísla\\
\hline
1061&maður&nkee&« og er eg son Ásbjarnar dettiáss er margir menn kannast hér við í Noregi en móðurkyn mitt er út á Íslandi en Þorgeir Ljósvetningagoði er móðurbróðir \textbf{manns} \\
\hline
1062&maður&nkfe&veturinn til Svíakonungs og var þar í góðu \textbf{manna} \\
\hline
1063&maður&nkfe&Reið Sighvatur eftir sættina norður heim og dreifði liði sínu \textbf{manna} \\
\hline
1064&maður&nkfe&Magnúss konungs menn ráku flóttann lengi og drápu allt það er þeir \textbf{manna} \\
\hline
1065&maður&nkfe&Gerði þá margur sem vant var að fara til \textbf{manna} við Njál en hann lagði það til mála manna sem ekki þótti líklegt að eyddust sóknir og varð af því þræta mikil er málin máttu eigi lúkast og riðu menn heim af þingi ósáttir\\
\hline
1066&maður&nkfe&» Þeir skipa síðan \textbf{manna} og voru sex hvorir og finna hann síðan þar er hann sat og hóf grautinn\\
\hline
1067&maður&nkfe&Gengu þeir Kolbeins menn þá til \textbf{manna} og rannsökuðu hvað manna þar væri inni\\
\hline
1068&maður&nkfe&Mikils þótti mönnum vert um aflraunir Orms þær sem hann hafði gert og gerði \textbf{manna} \\
\hline
1069&maður&nkfe&Hákon gamli svarar vel og segir að móðir hans skal ráða ferð hans en Ástríður vill fyrir engan mun að sveinninn \textbf{manna} \\
\hline
1070&maður&nkfe&Ólafur konungur gekk eftir árum útbyrðis er menn hans reru á Orminum og hann lék að þremur handsöxum svo að jafnan var eitt á lofti og henti æ \textbf{manna} \\
\hline
1071&maður&nkfe&sögðu þá mega sjá hvað manna Hallfreður var « mun \textbf{manna} svo ætla að bleðja hirðina\\
\hline
1072&maður&nkfe&Um haustið sendi Þorgils mann til \textbf{manna} að leita um sættir en hann var þver í því að taka fé fyrir vígsmálið við Þorgils en um önnur víg kvaðst hann gera mundu eftir skynsamra manna tillögum\\
\hline
1073&maður&nkfe&Nú gerðist svo mikið um drauma \textbf{manna} að hann gerir svo myrkhræddan að hann þorir hvergi einn saman að vera\\
\hline
1074&maður&nkfe&Jafnt hegndi hann \textbf{manna} og óríka\\
\hline
1075&maður&nkfe&Var Sigurður elstur \textbf{manna} og fyrir þeim mest í öllu\\
\hline
1076&maður&nkfe&Í þenna tíma var svo mikill ofsi Sturlu Sighvatssonar að nær öngvir menn hér á landi héldu sér réttum fyrir honum og svo hafa sumir menn hermt orð hans síðan að hann þóttist allt land hafa undir lagt ef hann gæti Gissur yfir \textbf{manna} \\
\hline
1077&maður&nkfe&Magnús konungur spurði til \textbf{manna} uppi á Ré\\
\hline
1078&maður&nkfe&Þá var fallinn Sæmundur húsfreyja og var það ráð \textbf{manna} er eftir voru að gefa upp kastala og sjálfa sig í vald heiðinna manna og var það hið ósnjallasta ráð fyrir því að heiðingjar efndu eigi orð sín\\
\hline
1079&maður&nkfe&Og svo sem Spes og skari \textbf{manna} kemur fram að veisunni var þar fyrir fjölmenni mikið og fjöldi fátækra manna er sér báðu ölmusu því að þetta var almenningsstræti\\
\hline
1080&maður&nkfe&Þrándur var \textbf{manna} mestur og sterkastur og manna fóthvatastur\\
\hline
1081&maður&nkfe&Fór Þorgeir Hjaltason frá Urðum og þeir tólf saman inn til \textbf{manna} við þá Eyjólf og Hrafn\\
\hline
1082&maður&nkfe&og leituðu meðalgöngu \textbf{manna} \\
\hline
1083&maður&nkfe&« Hér horfði til \textbf{manna} Kolli\\
\hline
1084&maður&nkfe&Vali frændi hans frétti hví það sætti er hann var svo óglaður « hvort þykir þér mikið \textbf{manna} \\
\hline
1085&maður&nkfe&Þetta sama haust er Ólafur konungur kom til \textbf{manna} urðu þau tíðindi þar að Sveinn konungur Haraldsson varð bráðdauður um nótt í rekkju sinni og er það sögn enskra manna að Játmundur hinn helgi hafi drepið hann með þeima hætti sem hinn helgi Merkúríus drap Júlíanum níðing\\
\hline
1086&maður&nkfe&Sá maður er nefndur Kolli er þar var \textbf{manna} skipi með þeim\\
\hline
1087&maður&nkfe&réðu þá það að þeir Þórir hundur snerust til \textbf{manna} með Verdælum og hafði hann sex hundruð manna\\
\hline
1088&maður&nkfe&» Egill þakkaði konungi orð sín og beiddist þess að konungur skyldi fá honum sannar jartegnir sínar til \textbf{manna} á Aurland eða annarra lendra manna í Sogni og á Hörðalandi\\
\hline
1089&maður&nkfe&Það var í þann tíma er Magnús konungur frændi hans var \textbf{manna} \\
\hline
1090&maður&nkfe&Og með því að það var konungs boð þá sá hún það að ráði og með henni vinir hennar að heitast Þórólfi ef það væri föður hennar eigi í móti \textbf{manna} \\
\hline
1091&maður&nkfe&Skalla-Grímur sótti fast smiðjuverkið en húskarlar \textbf{manna} vönduðu um og þótti snemma risið\\
\hline
1092&maður&nkfe&Örn var til \textbf{manna} en mestur hluti manna mælti í gegn og kváðu Örn allan villast og sögðu þá ráða eiga er fleiri voru\\
\hline
1093&maður&nkfe&Um kveldið er menn komu til \textbf{manna} talaði drottning við konung að það væri undarlegt « og ei vel til skipt að gefa Halla þá gripi er varla er ótiginna manna eiga fyrir klámyrði sín en þá fá sumir lítið fyrir góða þjónustu\\
\hline
1094&maður&nkee&« Skulum vér víst vináttu vora \textbf{manns} móti leggja\\
\hline
1095&maður&nkfe&Förunautar hans verða við hið vasklegasta og stíga af baki og bregða vopnum \textbf{manna} \\
\hline
1096&maður&nkfe&Veit eg og að þú munt ekki að verr gera til \textbf{manna} þó að eg sé hvergi í nánd og honum ann eg mest manna\\
\hline
1097&maður&nkfe&Og jafnan sýndi Áskell það að hann var fám mönnum líkur sakar \textbf{manna} er hann hafði manna í millum og drengskapar við hvern mann\\
\hline
1098&maður&nkfe&» Skarphéðinn glotti við og var svo búinn að hann var í blám kyrtli og í blárendum brókum og uppháva svarta skúa \textbf{manna} fótum\\
\hline
1099&maður&nkfe&Voru þar tvö skip í \textbf{manna} \\
\hline
1100&maður&nkfe&Brá Sturla þegar við og stefndi til \textbf{manna} mönnum og fór norður með fjölmenni og mikla sveit manna\\
\hline
1101&maður&nkfe&Um vorið litlu fyrir páska áður Oddur kom til Gissurar var fundur stefndur í Laugardal með Gissuri og \textbf{manna} \\
\hline
1102&maður&nkfe&að sitja löngum úti á haugi einum eigi langt \textbf{manna} bænum og svo bar nú að móti er Hallfreður kom\\
\hline
1103&maður&nkfeg&Tóku þá menn hans og drógu hann öfgan milli skipanna til sín og í því fékk hann fjögur \textbf{mannanna} \\
\hline
1104&maður&nkfe&Þá gerir Gunnhildur þegar sendimenn og býr þá vel að vopnum og \textbf{manna} þeir þrjá tigu manna og var þar til forráða ríkur maður\\
\hline
1105&maður&nkfe&Hann var vanur í fyrra lagi í kaupstefnur að koma og leggja lag á varning \textbf{manna} því að hann hafði héraðsstjórn\\
\hline
1106&maður&nkfe&Sonur Halls hinn elsti hét \textbf{manna} \\
\hline
1107&maður&nkfe&Og þar er þeir þóttust skjöldu bera \textbf{manna} baki sér þar báru þeir söðla sína\\
\hline
1108&maður&nkfe&Eitt haust var \textbf{manna} fundur fjölmennur í Skörðum að tala um hreppaskil og ómegðir manna og var því skipt að lögum\\
\hline
1109&maður&nkfe&Og er Gunnlaugur var tólf vetra gamall bað hann föður sinn \textbf{manna} og kvaðst hann vilja fara utan og sjá sið annarra manna\\
\hline
1110&maður&nkfe&En er Ásmundur hafði verið litla hríð með konungi þá undi hann ekki þar og hljópst í brott um nótt og kom aftur til sveitar sinnar og gerði þá enn fleira illt en \textbf{manna} \\
\hline
1111&maður&nkfe&Þar koma þeir bræður \textbf{manna} \\
\hline
1112&maður&nkfe&Skortir þig hvorki til \textbf{manna} né harðfengi að gera þeim þvílíka skömm eða meiri hverjir sem þeir eru\\
\hline
1113&maður&nkfe&En \textbf{manna} þinginu voru málin reifð\\
\hline
1114&maður&nkee&« Mun eg og hafa merki \textbf{manns} mér\\
\hline
1115&maður&nkfe&Hann \textbf{manna} Otradal vestur í Arnarfirði\\
\hline
1116&maður&nkfe&« Eg vildi að þú færir til dóma með mér og veittir mér lið því að þú ert vitur og framkvæmdarmaður \textbf{manna} \\
\hline
1117&maður&nkee&Nú líða þau misseri og er vorar býður Oddur honum þar að vera og þykir \textbf{manns} \\
\hline
1118&maður&nkfe&Svo er \textbf{manna} honum sagt að hann væri manna slægastur\\
\hline
1119&maður&nkfe&Nú ríður Illugi eftir þeim við hundrað manna og nú lýstur \textbf{manna} mjörkva miklum og verða nú aftur að hverfa\\
\hline
1120&maður&nkfe&lágur og þreklegur og kallaður manna minnstur \textbf{manna} sem þá voru\\
\hline
1121&maður&nkfe&Og allir hafa vörn þá ágætt er varð \textbf{manna} Flugumýri\\
\hline
1122&maður&nkfe&Geym þess er fundið er en eg skal leita \textbf{manna} er vantar\\
\hline
1123&maður&nkfe&Þorkell prestur Bergþórsson er naddur var kallaður varðist alldrengilega og féll þar \textbf{manna} húsunum\\
\hline
1124&maður&nkfe&Eigi er það fyrir þá sök að eg sé málsnjallur maður heldur ber mér til handa mikinn vanda oft af \textbf{manna} \\
\hline
1125&maður&nkfe&Gengu menn þá undir það \textbf{manna} \\
\hline
1126&maður&nkfe&Þórður sagði þá kost \textbf{manna} að reyna hvatleik manna og vopnfimi\\
\hline
1127&maður&nkfeg&Nú vil eg að vér verðum samdóma fyrir konungsmönnum að ei brjóti þetta \textbf{mannanna} mér einum\\
\hline
1128&maður&nkfe&Síðan far þú \textbf{manna} Galmaströnd eftir þeim mágum mínum\\
\hline
1129&maður&nkfe&« Heyrði eg \textbf{manna} getið og sjaldan að góðu\\
\hline
1130&maður&nkfe&En Gunnhildur drottning lagði svo miklar mætur á hann að hún hélt engin hans jafningja innan hirðar hvorki í orðum né öðrum \textbf{manna} \\
\hline
1131&maður&nkfe&Mörður Valgarðsson stóð \textbf{manna} Gissuri hvíta mági sínum\\
\hline
1132&maður&nkfe&Þeir námu \textbf{manna} við götuskarðið er þeir komu yfir Vatnsdalsá\\
\hline
1133&maður&nkfe&» Nú tók hann við málinu en gaf honum gjafar og skildust \textbf{manna} \\
\hline
1134&maður&nkfe&Hann andaðist í \textbf{manna} \\
\hline
1135&maður&nkfe&Hann svarar að Egill var við \textbf{manna} skútu einni með þrjá tigi manna « og reru þeir leið sína út til Steinssunds\\
\hline
1136&maður&nkfe&Einar lagði þegar til \textbf{manna} við þá og sigraðist en þeir féllu báðir\\
\hline
1137&maður&nkfe&kvað Ketil vilja mönnum \textbf{manna} gott en Loft kvað hann mæla til manna hvaðvetna gott\\
\hline
1138&maður&nkfe&Voru þá felld seglin \textbf{manna} \\
\hline
1139&maður&nkfe&Það er flestra manna sögn að Magnús prestur og Grunnvíkingar hafi \textbf{manna} gera bréf það er kom til Ásgríms\\
\hline
1140&maður&nkfe&Var það allra manna mál að Haraldur konungur hafði verið umfram aðra menn að speki og \textbf{manna} \\
\hline
1141&maður&nkfe&Þau skip létu út bæði senn \textbf{manna} Gásum\\
\hline
1142&maður&nkfe&Ef þér viljið mér lið veita þá sé eg ekki til vænna en að menn fari og hafi tveir saman hest því að eg veit að Eyjólfur mun \textbf{manna} \\
\hline
1143&maður&nkfe&Eyjólfur Bölverksson var virðingamaður mikill og allra manna lögkænastur svo að hann var hinn þriðji maður mestur lögmaður á \textbf{manna} \\
\hline
1144&maður&nkfe&Ketill tók þegar mál af Símoni og fór með nokkura sveit manna en sagði að þeir kaupmenn skyldu halda skjótt eftir \textbf{manna} og hafið varning með yður\\
\hline
1145&maður&nkfe&Eftir það gengu til frændur Erlings og vinir og báðu hann til vægja og færa við vit en eigi \textbf{manna} \\
\hline
1146&maður&nkfe&Þá bjuggust þeir bræður og Össur með þeim að ríða austur til \textbf{manna} og riðu við sex tigu manna\\
\hline
1147&maður&nkfe&Steinþór svarar \textbf{manna} \\
\hline
1148&maður&nkfe&Mikil ætt er komin \textbf{manna} Þórði hreðu og margir göfgir menn bæði í Noregi og Íslandi\\
\hline
1149&maður&nkfe&Þann dag er biskup sat þar urðu kynlegleikar þeir \textbf{manna} \\
\hline
1150&maður&nkfe&Kjartan fastaði þurrt langaföstu og gerði það að engis manns dæmum hér \textbf{manna} landi því að það er sögn manna að hann hafi fyrstur manna fastað þurrt hér innanlands\\
\hline
1151&maður&nkfe&Voru þá grið sett og fundur lagður út við Fossá út \textbf{manna} Skógum þar beint sem þeir höfðu fundist Sæmundur og Sigurður Ormsson\\
\hline
1152&maður&nkee&» « Eigi varði mig \textbf{manns} fóstri minn\\
\hline
1153&maður&nkfe&áttu þar óðal og \textbf{manna} \\
\hline
1154&maður&nkfe&Nú ganga menn á milli um stund og leita um sættir \textbf{manna} \\
\hline
1155&maður&nkfe&Mætti guð gefa það að eg væri \textbf{manna} öðrum fundi miklu snjallmæltari\\
\hline
1156&maður&nkfe&Knútur jarl var mikill maður vexti og vænn sýnum \textbf{manna} \\
\hline
1157&maður&nkfe&Nollar átti fé lítið en mikla ómegð og hafði það mest til atvinnu er hann \textbf{manna} \\
\hline
1158&maður&nkfe&Þeir fóru degi fyrr en Eyjólfur og fóru Öxnadalsheiði og \textbf{manna} Norðurárdal og áðu í Svínanesi\\
\hline
1159&maður&nkfe&Hann er kvonlaus \textbf{manna} \\
\hline
1160&maður&nkfe&hvort hann vill heldur fara með þeim og brenna inni Þorbjörn og sonu \textbf{manna} eða láta þar líf\\
\hline
1161&maður&nkfe&Sendi Broddi mann í Sæmundarhlíð til Páls Kolbeinssonar en annar var sendur Einari að hann skyldi safna mönnum um \textbf{manna} \\
\hline
1162&maður&nkfe&hans son var \textbf{manna} \\
\hline
1163&maður&nkfe&Og er hann kemur á fund \textbf{manna} þá slær Þorkell við hann kaupi á laun að hann skyldi svo greina frásögn um líflát manna sem hann segði fyrir\\
\hline
1164&maður&nkee&Haraldur \textbf{manns} manns rétt og tólf marka veislur og umfram hálft fylki í Þrándheimi\\
\hline
1165&maður&nkfe&Vér kunnum frá öngvum tíðindum að segja víslega en sögðu \textbf{manna} því hvar þeir Bárður skildust og þar var mikill fjöldi manna fyrir ofan Sleðaás og þangað riðu þeir Bárður til og það þóttumst vér sjá að menn spruttu upp í flokkinum með vopnum og gerðu þá handtekna alla að minnsta kosti\\
\hline
1166&maður&nkfe&Þeir Gissur og Kolbeinn héldu flokkunum vestur um Bláskógaheiði og höfðu \textbf{manna} hundrað manna\\
\hline
1167&maður&nkfe&Þá býður biskup honum staðinn á Völlum til \textbf{manna} en hann vildi það eigi\\
\hline
1168&maður&nkfe&Flosi kvað \textbf{manna} að von « og er yður þetta viðvörun\\
\hline
1169&maður&nkfe&« Ríð þú nú fyrir en eg mun fylgja Þórlaugu \textbf{manna} \\
\hline
1170&maður&nkfe&Kolli sótti Björn fast \textbf{manna} \\
\hline
1171&maður&nkfe&Skip þeirra kom í óbyggð á Grænlandi og týndust menn allir en þess varð svo víst að fjórtán vetrum síðar fannst skip þeirra og sjö menn í hellisskúta \textbf{manna} \\
\hline
1172&maður&nkfe&Hann hafði leiddan inn \textbf{manna} sér hest sinn og réð til þrisvar á bak að hlaupa\\
\hline
1173&maður&nkfe&Um vorið fór Björn að reka geldinga sína neðan af Völlum og upp eftir dalnum þeim megin sem Húsafellsbær er og húskarlar \textbf{manna} með honum og sáu kolreyk í skóginn og heyrðu manna mál\\
\hline
1174&maður&nkfe&Þar hafði verið glaumur og gleði mikil en nú \textbf{manna} það af og gerist hljóðlæti mikið í höllinni\\
\hline
1175&maður&nkfe&Hélt hann til Borgarfjarðar og kom skipinu skammt frá bæ \textbf{manna} \\
\hline
1176&maður&nkfe&En Ketill var \textbf{manna} sterkastur í það mund\\
\hline
1177&maður&nkfe&» og fór með þrjá tigu manna og kom fram í Eystra-Gautlandi \textbf{manna} \\
\hline
1178&maður&nkfe&Og færðu þá steina tuttugu menn þannug sem þeir vildu er engan veg gátu áður hrært hundrað manna og var vegurinn ruddur að miðjum degi svo að fært var bæði mönnum og hrossum með Síðan fór konungur ofan aftur þangað sem vist þeirra var og nú heitir \textbf{manna} \\
\hline
1179&maður&nkfe&Hann hafði skútu \textbf{manna} Hofi er átti Broddi Þorleifsson\\
\hline
1180&maður&nkfe&Ögmundur Guðmundarson var særður til \textbf{manna} og drukknaði í Hvítá er hann fór heim\\
\hline
1181&maður&nkfe&Björn var allbeinn við hann \textbf{manna} kveldið\\
\hline
1182&maður&nkfe&Fátt var manna heima því að Halldór hafði sent menn norður í \textbf{manna} \\
\hline
1183&maður&nkfe&Sumir vorir menn skulu fara til Ingjalds og gera þar slíkt hið \textbf{manna} \\
\hline
1184&maður&nkfe&Börkur og Eyjólfur komu um kveldið með sex tigu manna og er þar \textbf{manna} búi hundrað manna en að Gísla er hálft hundrað manna\\
\hline
1185&maður&nkfe&Nú líður fram vetrinum og þegar á bak jólum býr jarl ferð sína og hefir sex tigu manna \textbf{manna} \\
\hline
1186&maður&nkfe&Þar áttu verkmenn Þorsteins tjald á \textbf{manna} \\
\hline
1187&maður&nkfe&ríða nú síðan suður til þings og segja þar þessi \textbf{manna} \\
\hline
1188&maður&nkee&Og nú þóttist Skúta sjá að þetta mundi verða meir til \textbf{manns} að taka og setti þau ráð til að Þorsteinn skal bjóða Þorgeiri sætt fyrir sekt þess manns er hann hafði fyrir ráðið og lúka þegar fyrir sakar fjölskyldar þeirrar er hann átti að annast\\
\hline
1189&maður&nkfe&Hélt Þorgrímur skipinu til Grænlands og fórst honum \textbf{manna} \\
\hline
1190&maður&nkfe&Ormur Svínfellingur hafði \textbf{manna} tigu manna en Þórarinn bróðir hans fimm tigu manna og voru þeir Kolbeins vinir mestir\\
\hline
1191&maður&nkfe&Gissur jarl hafði eigi færra með því liði er hann tók upp suður en átta hundruð manna \textbf{manna} \\
\hline
1192&maður&nkfe&Þeir Halldór og \textbf{manna} förunautar riðu að Öxnagróf\\
\hline
1193&maður&nkfe&Það er að segja frá för Finns að hann hafði skútu og \textbf{manna} nær þremur tigum manna en er hann var búinn fór hann ferðar sinnar til þess er hann kom á Hálogaland\\
\hline
1194&maður&nkee&« Melt hafið þér það bræður er ei er vænna til en steina þessa er þér hafið ei þorað að hefna Halls bróður \textbf{manns} \\
\hline
1195&maður&nkfe&Sámur svarar \textbf{manna} \\
\hline
1196&maður&nkee&Björn hjó af þessum manni höndina og skaust aftur síðan að baki Kára og fengu þeir honum engan geig \textbf{manns} \\
\hline
1197&maður&nkfe&Hann sendir menn sína norður um land til \textbf{manna} að kveðja lið upp og gaf landráðasök þeim er eigi fóru og fór margt manna norðan og kom suður yfir heiði\\
\hline
1198&maður&nkfe&Þar lá fyrir skip Þorgeirs Hávarssonar og hafði hann sekur orðið um sumarið um víg Þorgils frænda Grettis Ásmundarsonar og um launvígsmál \textbf{manna} að Hrófá\\
\hline
1199&maður&nkfe&Riðu þeir Þorvarður og Þorgils til \textbf{manna} og mundu hafa úr Skagafirði nær sjö tigi manna\\
\hline
1200&maður&nkfe&Stórólfur bjó að Hvoli er síðan var kallaður Stórólfshvoll \textbf{manna} \\
\hline
1201&maður&nkfe&kvað það sakagiftir \textbf{manna} við Gissur og Klæng\\
\hline
1202&maður&nkfe&Hann kastar þá til Þorvalds öxinni og kemur öxin í höfuðið og féll hvortveggi dauður \textbf{manna} \\
\hline
1203&maður&nkfe&og sögðu að Órækja var kominn vestan til \textbf{manna} með átta tigu manna og ætlaði að setjast á Staðarhól\\
\hline
1204&maður&nkfe&» Bárður kvað eigi auðfengna menn til slíks verks « nú skal eg heimta þetta fé saman sem eg eigi sjálfur til handa \textbf{manna} \\
\hline
1205&maður&nkfe&Hann var sköllóttur í skarlatsklæðum og hélt \textbf{manna} hjartskinnsglófum en kona mikil og væn sat við borða\\
\hline
1206&maður&nkfe&Gunnar hafði riðið áður um daginn vestur \textbf{manna} Þórisstaði og er hann var þar kominn spyr Þórir hvaðan hann væri að kominn\\
\hline
1207&maður&nkfe&Hann hafði til \textbf{manna} og þar á þrjá tigu manna\\
\hline
1208&maður&nkfe&Þorkell neytir lítt \textbf{manna} um kveldið og gengur fyrstur manna að sofa\\
\hline
1209&maður&nkfe&bæði norrænir menn og íslenskir \textbf{manna} \\
\hline
1210&maður&nkfe&» Fór Eiríkur heim í Brattahlíð en Leifur réðst til skips og félagar \textbf{manna} með honum\\
\hline
1211&maður&nkfe&En lítið er mér um að fara í sveit \textbf{manna} og að hann eigi meiri afla en eg\\
\hline
1212&maður&nkfe&Nú er að þann veg að vér munum breyta ráðum um ferðir vorar og snúa aftur til héraðs en Álfur son Þórodds jarls skal taka við goðorði \textbf{manna} \\
\hline
1213&maður&nkfe&fer nú á konungs fund og segir honum sem komið \textbf{manna} \\
\hline
1214&maður&nkfe&Og er þeir hættu þökkuðu allir fyrir glímuna þeim og var það dómur þeirra er hjá sátu að þeir væru eigi sterkari tveir en Grettir einn en hvor \textbf{manna} hafði tveggja manna megin þeirra sem gildir voru\\
\hline
1215&maður&nkfe&Voru þeir svo hraustir menn í sér að enginn þeirra vildi í sjóð bera \textbf{manna} \\
\hline
1216&maður&nkfe&stálhúfa \textbf{manna} höfði og öll ryðug\\
\hline
1217&maður&nkfe&Staðfestist Ásmundur þar um hríð og var vel \textbf{manna} \\
\hline
1218&maður&nkfe&» « Það er auðsætt \textbf{manna} \\
\hline
1219&maður&nkfe&Þormóður son Refs fór til Íslands eftir fall Haralds konungs og tók við landi á Kvennabrekku og kvæntist \textbf{manna} Íslandi og er margt göfugra manna frá honum komið\\
\hline
1220&maður&nkfe&Berjast þeir Þórður og Skeggi lengi \textbf{manna} svo að engi skakkar með þeim\\
\hline
1221&maður&nkfe&Og ætlar þú að landsbyggðin megi eigi bera ríki \textbf{manna} hér á landi er svo göfugra manna er\\
\hline
1222&maður&nkfe&Björn reið um sumarið \textbf{manna} Hítarnes með sex tigu manna og stefndi Þórði um vísuna sem hann kallaði lög til standa\\
\hline
1223&maður&nkee&« Hvað sagðir þú við móður þína \textbf{manns} hausti að hún skyldi eigi skipta skap sitt eftir hins versta manns orðum\\
\hline
1224&maður&nkfe&Kom því svo að þeir sem á virkinu voru fundu eigi fyrr en Órækju menn allir voru komnir í húsin og höfðu gengið upp eftir forskála \textbf{manna} laugu\\
\hline
1225&maður&nkfe&Réðst þá til ferðar með honum Gunnsteinn bróðir hans og hafði hann sér \textbf{manna} \\
\hline
1226&maður&nkfe&Húsið var gert að veggjum af timburstokkum stórum en í annan enda hússins var skjaldþili \textbf{manna} \\
\hline
1227&maður&nkfe&en hann fór eftir að leita \textbf{manna} og hitti í einu bóli þrjá tigu manna og drap alla svo að engi komst undan en síðan hitti hann saman fimmtán eða tuttugu\\
\hline
1228&maður&nkfe&En fyrir því að Hálfdan konungur var ofurliði borinn flýði hann til \textbf{manna} og lét mart manna\\
\hline
1229&maður&nkfe&Þar var og \textbf{manna} sumt strákar og stafkarlar og göngukonur\\
\hline
1230&maður&nkfe&« Auðsætt er það \textbf{manna} Ólafi þessum að hann er stórættaður maður\\
\hline
1231&maður&nkfe&Þá réð fyrir Orkneyjum Sigurður jarl Hlöðvisson \textbf{manna} \\
\hline
1232&maður&nkfe&Hann lét og fyrir lið allan farangur sinn og biður hvern mann til farar með sér er komast mætti og biður sér liðs og ríður eftir \textbf{manna} \\
\hline
1233&maður&nkfe&Það var einn þvottdag að Gísl stóð við stræti nokkuð snemma \textbf{manna} og heyrði hann gný mikinn\\
\hline
1234&maður&nkfe&Á sömu leið fór um aðra sendimenn er Hákon konungur sendi austur \textbf{manna} Vermaland að menn voru drepnir en fé kom eigi aftur\\
\hline
1235&maður&nkeeg&Þá hljóp Gunnar yfir gjána en fylgdarmenn \textbf{mannsins} komust eigi yfir og skildi þar með þeim\\
\hline
1236&maður&nkfe&Er hitt bæn mín og vilji að þér konungur farið að heimboði til \textbf{manna} og heyrið þá orð þeirra manna er þú trúir\\
\hline
1237&maður&nkfe&Og einn hvern dag er það sagt að konungur og hans menn færu \textbf{manna} dýrsveiði og svo hirðin en fátt manna var eftir í höllinni\\
\hline
1238&maður&nkfe&fóru vestur til \textbf{manna} eftir liði og fengu þar fjóra tigu manna\\
\hline
1239&maður&nkfe&Nú sjá biskupsmenn hvar fer flokkur \textbf{manna} og snúa þegar í mót þeim\\
\hline
1240&maður&nkfe&að eldur yndisins og logi elskunnar brennur því heitara og sækir því meir brjóst og hjörtu mannanna saman sem fleiri vilja þeim meina og stærri skorður við settar þeirra vandamanna er áður hefir ást og elska saman fallið \textbf{manna} á millum sem nú þessara manna\\
\hline
1241&maður&nkfe&Áslákur \textbf{manna} \\
\hline
1242&maður&nkfe&En Kolbeinn segir að hann var þá búinn til \textbf{manna} er hann vildi fyrir engan mun bregða\\
\hline
1243&maður&nkfe&spurði síðan hvort \textbf{manna} hefði nokkuð skatt heimt um Norðureyjar eða hver greiði þá mundi á vera um silfrið\\
\hline
1244&maður&nkfe&Setti Þórður þá menn út á Stigagnúp að verða \textbf{manna} hið ytra með skipum\\
\hline
1245&maður&nkfe&Hafði hver \textbf{manna} mikla sveit manna\\
\hline
1246&maður&nkfe&Hann bjó \textbf{manna} Hnappsstöðum og var orðamaður mikill\\
\hline
1247&maður&nkfe&Hann skilur og það til að hann skal þá \textbf{manna} misserum land laust láta og gera hinum orð um\\
\hline
1248&maður&nkfe&» « Hér \textbf{manna} þeir\\
\hline
1249&maður&nkfe&Vildi hann að Þorgils kæmi til móts við hann og töluðu um ráðagerðir \textbf{manna} \\
\hline
1250&maður&nkfe&Þeir tjölduðu ágætt herbergi við þinghelgi til \textbf{manna} við Eyjólf\\
\hline
1251&maður&nkfe&Þeir Kjartan höfðu og mikið fjölmenni fyrir \textbf{manna} \\
\hline
1252&maður&nkee&Vildi eg faðir þú vísaðir okkur til víkings nokkurs þess er mér væri nokkur frægð í og biði eg annaðhvort bót eða bana og væri mín síðan getið að \textbf{manns} \\
\hline
1253&maður&nkfe&« að engi \textbf{manna} mundi þetta ráð fundið hafa nema þér nytuð yður hyggnari manna við\\
\hline
1254&maður&nkfe&Hann mælti þá til sinna manna \textbf{manna} \\
\hline
1255&maður&nkfe&Graðungur hans gerði mönnum mart mein þá er hann kom úr \textbf{manna} \\
\hline
1256&maður&nkfe&Einn af þeim var Böðvar Klængsson \textbf{manna} Bjarnargili úr Fljótum er síðan var allra manna knástur\\
\hline
1257&maður&nkfe&svo um veislur og heimboð við vini sína þá hafði hann meira efni um það allt en fyrr \textbf{manna} \\
\hline
1258&maður&nkfe&fagureygur \textbf{manna} snareygur svo að ótti var að sjá í augu honum ef hann var reiður\\
\hline
1259&maður&nkfe&Björn var hinn þriðja vetur með Skalla-Grími en eftir um vorið bjóst hann til \textbf{manna} og sú sveit manna er honum hafði þangað fylgt\\
\hline
1260&maður&nkfe&Var nú ráðin norðurferð \textbf{manna} og skildu að því\\
\hline
1261&maður&nkfe&Þorgils bróðir Kormáks kvað þetta \textbf{manna} \\
\hline
1262&maður&nkfe&En eftir þingið búast þeir til \textbf{manna} og höfðu fjögur skip og þrjá tigu manna á hverju og réðu þeir Einar og Þórarinn og Þórður fyrir skipunum og komu innan að eyjunni í næturelding og sáu reyk yfir húsunum og spurði Einar hvort þeim sýndist svo sem honum að reykurinn væri eigi allblár\\
\hline
1263&maður&nkfe&Bóndi kvað við hátt með miklum skræk og þreif til þjóhnappanna báðum \textbf{manna} \\
\hline
1264&maður&nkfe&« Eg sé \textbf{manna} þér að þú ert hinn mesti íþróttamaður að nokkurum hlut en það mun eg sjá brátt hvað það er\\
\hline
1265&maður&nkfe&En ef þú bíður \textbf{manna} þá verður það þinn skaði og þeirra manna er þér fylgja\\
\hline
1266&maður&nkfe&« því að mig grunar að \textbf{manna} mun von\\
\hline
1267&maður&nkfe&Þeir komu til \textbf{manna} er kallaður var Kakalahóll\\
\hline
1268&maður&nkfe&« \textbf{manna} þykir mikill munur að það er höfðinglegra að sá er yfirmaður skal vera annarra manna sé mikill í flokki\\
\hline
1269&maður&nkfe&Hann fór til \textbf{manna} með sex tigu manna\\
\hline
1270&maður&nkfe&Gengur Sigurður vel fram og þar kemur að hroðin voru öll skip Sigurðar en þrjú af \textbf{manna} \\
\hline
1271&maður&nkfe&því \textbf{manna} það var siður ríkra manna barna í þann tíma að hafa nokkura iðn fyrir hendi\\
\hline
1272&maður&nkfe&Vann Eysteinn konungur Jamtaland með viti en eigi með áhlaupum sem sumir \textbf{manna} langfeðgar\\
\hline
1273&maður&nkfe&« Guðmundur bróðir þinn er \textbf{manna} og vill hitta þig\\
\hline
1274&maður&nkfe&« \textbf{manna} á millum\\
\hline
1275&maður&nkfe&En féránsdómar voru í Hvammi og sóttu þeir heim til \textbf{manna} Þorleifur og Einar Þorgilsson og höfðu nær hundrað manna\\
\hline
1276&maður&nkfe&« Nú skulum vér ganga til móts við þá en eg skal hafa orð fyrir \textbf{manna} \\
\hline
1277&maður&nkfe&Einar kveðst það og vita að það mundi ekki skræfa verið hafa er boga \textbf{manna} hafði fyrir odd dregið\\
\hline
1278&maður&nkfe&Snorri sendi orð Þorvaldi Vatnsfirðingi að hann skyldi ríða til \textbf{manna} með honum\\
\hline
1279&maður&nkfe&Gissur kom er \textbf{manna} leið daginn með fjóra tigi manna\\
\hline
1280&maður&nkfe&Þeir sátu lengi tveir á máli en það kom upp úr hjali þeirra að þeir skyldu hittast í Björgyn allir bræður eftir um \textbf{manna} \\
\hline
1281&maður&nkfe&En föstudaginn lét hann Jón prest lærdjúp ólea sig við fullting djákna sinna \textbf{manna} annarra lærðra manna þar heima\\
\hline
1282&maður&nkfe&Bjó hann eftir föður sinn á Finnbogastöðum og þótti hann \textbf{manna} í sveitum gildastur bóndi og formaður annarra manna\\
\hline
1283&maður&nkfe&Um veturinn \textbf{manna} gerðust menn handgengnir Gissuri jarli þrír tigir manna\\
\hline
1284&maður&nkfe&Og margir aðrir menn mjög ágætir voru \textbf{manna} Orminum þótt vér kunnum eigi einum\\
\hline
1285&maður&nkfe&Er manna hingað von í dag og skuluð þér þá neyta stafanna og berja hrossin undir þeim og reka svo úr túni allt \textbf{manna} \\
\hline
1286&maður&nkfe&Hafði Snorri þá og margt manna \textbf{manna} \\
\hline
1287&maður&nkfe&Þá fór Jón með þeim út í Otradal á fund Sturlu og tók hann allhart á \textbf{manna} \\
\hline
1288&maður&nkfe&Þá var til \textbf{manna} leiddur Hermundur Hermundarson\\
\hline
1289&maður&nkfe&En er hann kom suður um Kjöl reið hann \textbf{manna} liðinu með hundrað manna suður til Keldna og bað Hálfdan veita sér lið með allan sinn afla\\
\hline
1290&maður&nkfe&Sonarsonur \textbf{manna} var Þórir helsingur\\
\hline
1291&maður&nkfe&hverjir fylgdu best föður þínum vestur \textbf{manna} Írlandi efsta sinni\\
\hline
1292&maður&nkee&En eg veit að við erum báðir illmenni því að þú mundir ekki hér kominn \textbf{manns} öðrum mönnum nema þú værir nokkurs manns útlagi\\
\hline
1293&maður&nkfe&háttagóð hversdaglega og kom til kirkju hvern dag áður hún færi til verks síns en eigi var hún glöð eða margmálug \textbf{manna} \\
\hline
1294&maður&nkfe&Böðvar gekk til \textbf{manna} og segir honum hljótt hvað títt var\\
\hline
1295&maður&nkfe&þóttist nær kominn til \textbf{manna} í Skagafirði\\
\hline
1296&maður&nkfe&Bjarni spurði hann um heilsun \textbf{manna} og um búfjárhagi\\
\hline
1297&maður&nkfe&Þá var sagt að þar væru fyrir \textbf{manna} hundruð heiðinna manna og þangað væri von berserks þess er Ótryggur hét og voru allir við hann hræddir\\
\hline
1298&maður&nkfe&Sagði að njósnarmenn \textbf{manna} voru komnir vestan og segja að Hrafn og Sturla mundu koma vestan í Langadal að drottinsdagshelginni við þrjú hundruð manna\\
\hline
1299&maður&nkfe&Konungur bað Leif taka til \textbf{manna} manna ef hann þyrfti skjótleiks við því að þau voru dýrum skjótari\\
\hline
1300&maður&nkfe&» Bóndi þakkar honum en kvað þó mikið í hættu þar er Þorgils var og \textbf{manna} menn\\
\hline
1301&morgunn&nkee&Már sagði að þeir væru sáttir \textbf{morguns} \\
\hline
1302&mál&nhfe&« Hér munum vér nú skiljast heilir og finnast \textbf{mála} þingi og taka þar til óspilltra mála\\
\hline
1303&mál&nhee&» Síðan gekk Ölvir hnúfa til \textbf{máls} og lét kalla Þórólf til máls við sig\\
\hline
1304&mál&nhfe&Er mér sagt að þú eigir sókn og vörn mála þeirra er við eru kenndir \textbf{mála} og Höskuldur\\
\hline
1305&mál&nhee&Ólafur gekk til móts við báða bræður sína um \textbf{máls} \\
\hline
1306&mál&nhee&Hér skulu eigi \textbf{máls} fyrir koma að svo búnu\\
\hline
1307&mál&nhfe&Nú far þú á fund \textbf{mála} og leita við hann þessa mála\\
\hline
1308&mál&nhee&En um sumarið ríður Þorbjörn til þings með menn sína úr \textbf{máls} \\
\hline
1309&mál&nhee&Gekk þá allt lið á land og var sett þing \textbf{máls} \\
\hline
1310&mál&nhee&Nokkuru síðar kom Mörður í Ossabæ og \textbf{máls} Höskuld til máls við sig\\
\hline
1311&mál&nhfe&Vil eg að þú vitir það Höskuldur áður við sláum kaupi þessu \textbf{mála} \\
\hline
1312&mál&nhee&» « Þess vil eg þá biðja þig og svo aðra þá \textbf{máls} vil eg þá biðja þig og svo aðra þá er líklegastir væru að þá hluti mættu til leggja að þá væri sigurinn vænlegri en áður\\
\hline
1313&mál&nhee&þá er konungur var klæddur var hann fámálugur og ókátur og hræddust vinir hans að þá mundi enn að honum komið \textbf{máls} \\
\hline
1314&mál&nhfe&» « Satt mun það vera \textbf{mála} \\
\hline
1315&mál&nheeg&« Ef eg á svo mikið vald á þér sem eg ætla þá legg eg það á við þig að þú megir engri munúð fram koma við þá konu er þú ætlar þér \textbf{málsins} Íslandi að eiga en fremja skalt þú mega við aðrar konur vilja þinn\\
\hline
1316&mál&nhfe&Arnór Tumason lagði það til að Loftur skyldi standa í þeim sporum þá er handsöl færu fram sem Sigurður mágur hans stóð þá er þeir lögðu þar virðing sína fyrir þeim \textbf{mála} \\
\hline
1317&mál&nhee&Eftir það reið Egill heim til Borgar og er hann kom heim þá gekk hann þegar til \textbf{máls} er hann var vanur að sofa í\\
\hline
1318&mál&nhfe&Það sumar bjóst Þorvarður til \textbf{mála} í Eyjafirði\\
\hline
1319&mál&nhee&» Að Lögbergi var \textbf{máls} rómur að því að honum mæltist vel og skörulega\\
\hline
1320&mál&nhfe&« Til \textbf{mála} skal eg honum þjóna lengur\\
\hline
1321&mál&nhfe&« Fá Halldóri mála sinn skíran því að verður er hann að \textbf{mála} \\
\hline
1322&mál&nhee&En ef nokkur hlutur gerist sá í lögvörn þeirra er eg þurfi til sóknar að hafa þá kýs eg sókn undir \textbf{máls} \\
\hline
1323&mál&nhfe&Síðan fór hann til spekings eins og sagði honum drauminn en hann réð svo að hann mundi fara suður í lönd og verða \textbf{mála} riddari\\
\hline
1324&mál&nhfe&þeir er \textbf{mála} mála vildu ganga\\
\hline
1325&mál&nhfe&« Vil eg taka við Helga syni þínum og geyma sem eg kann en eg vil hafa vináttu þína í mót og fylgi til þess að eg nái réttu af \textbf{mála} \\
\hline
1326&mál&nhfe&Og einn dag um veturinn gekk Þórður að Birni og bað hann drekka með sér « erum við nú þar komnir að vist að okkur samir eigi annað en vel sé með okkur og það eitt missætti hefir hér í millum verið að lítils er virðanda og því látum nú vel vera héðan \textbf{mála} \\
\hline
1327&mál&nhee&En er til tók lag þeirra Ara og Úlfheiðar lét hún koma í hendur honum fimmtán hundruð þriggja alna aura til forráða og hafði hún þá eftir gullhring og marga gripi \textbf{máls} \\
\hline
1328&mál&nhfe&Og ætlar þú að landsbyggðin megi eigi bera ríki þess manns hér á landi er svo göfugra \textbf{mála} er\\
\hline
1329&mál&nhfe&Og munu menn það mæla að þínu máli sé framar komið þó að á \textbf{mála} \\
\hline
1330&mál&nhee&Þórólfur gerði sér títt við Björn og var honum \textbf{máls} \\
\hline
1331&mál&nhfe&Aðalsteinn konungur var vel kristinn \textbf{mála} \\
\hline
1332&mál&nhee&Höfðu þeir bræður synir Þorgils sigrað björninn er þeir höfðu gengið frá \textbf{máls} \\
\hline
1333&mál&nhee&Var hann þá \textbf{máls} gamall\\
\hline
1334&mál&nhee& \textbf{máls} Nú er þar til máls að taka er fyrr var frá horfið að þá er þeir Guðmundur biskup og Hrafn Sveinbjarnarson komu út og höfðu einn vetur verið í Noregi fór Hrafn vestur í Arnarfjörð á Eyri til bús síns\\
\hline
1335&mál&nheeg&Bjarni sagði að honum þótti lítilmannlegt að flýja bú sín og fór hann af því heim en þó vildi Glúmur að hann hefði eigi heim farið \textbf{málsins} því méli\\
\hline
1336&mál&nhee&Fór konungur þá ferðar sinnar og tók veislur þar er fyrir voru \textbf{máls} \\
\hline
1337&mál&nhfe&Hann fann þar Pál biskup úr Hamri og voru þeir allir samt í för út í Róma og veitti biskup \textbf{mála} vel föruneyti og var hinn mesti flutningsmaður allra hans mála er þeir komu til páfafundar\\
\hline
1338&mál&nhee&« Fyrir það mun ganga sem eg sé aðili þess máls því að eg hefi \textbf{máls} umboð til þessa máls\\
\hline
1339&mál&nhee&Eftir það eiga Norðlendingar stefnu milli sín og ræða þetta \textbf{máls} \\
\hline
1340&mál&nhee&Nú er þar til máls að taka að Hallgerður vex \textbf{máls} \\
\hline
1341&mál&nhfe&« Varst þú \textbf{mála} Þingskálaþingi um haustið\\
\hline
1342&mál&nhfe&að taka við vináttu Knúts konungs og Hákonar jarls og gerast þeirra maður og selja til \textbf{mála} trú þína og taka hér mála þinn\\
\hline
1343&móðir&nvee&Nú þar um hugsandi hefi eg fundið það \textbf{móður} allir hafa og halda\\
\hline
1344&móðir&nvee&faðir \textbf{móður} \\
\hline
1345&móðir&nvee&faðir \textbf{móður} \\
\hline
1346&móðir&nvee&En er synir Vísburs voru tólf vetra og þrettán fóru þeir á fund \textbf{móður} og heimtu mund móður sinnar en hann vildi eigi gjalda\\
\hline
1347&móðir&nvee&» Arnór skildi góðfýsi móður sinnar og tók vel ásakan hennar \textbf{móður} \\
\hline
1348&móðir&nvee&Leiðólfur var faðir \textbf{móður} \\
\hline
1349&móðir&nvee&hana átti \textbf{móður} undir Felli\\
\hline
1350&móðir&nvee&En Þorgeir stendur þá milli þeirra og er hann til samnings og nú fær Þorsteinn eigi hefnt \textbf{móður} og skiljast þeir nú að því\\
\hline
1351&móðir&nvee&faðir \textbf{móður} \\
\hline
1352&móðir&nvee&faðir \textbf{móður} \\
\hline
1353&móðir&nvee&Svo segja fróðir menn að Haraldur hinn hárfagri hafi verið allra manna fríðastur sýnum og sterkastur og \textbf{móður} \\
\hline
1354&móðir&nvee&faðir \textbf{móður} \\
\hline
1355&móðir&nvee&móðir \textbf{móður} \\
\hline
1356&móðir&nvee&kristnir menn kirkjur sækja \textbf{móður} \\
\hline
1357&móðir&nvee&faðir \textbf{móður} \\
\hline
1358&móðir&nvee&Hann tók nú við barninu og leist vel á og nennti eigi að kasta á ána og sneri nú ferðinni og lét barnið niður í garðshliði \textbf{móður} Signýjarstöðum og þótti von að brátt mundi finnast\\
\hline
1359&móðir&nvee&faðir \textbf{móður} \\
\hline
1360&móðir&nvee&Auðun skökull var faðir \textbf{móður} mosháls\\
\hline
1361&móðir&nvee&móðir \textbf{móður} \\
\hline
1362&móðir&nvee&móðir \textbf{móður} \\
\hline
1363&móðir&nvee&faðir \textbf{móður} \\
\hline
1364&móðir&nvee&Voru nú sett fullkomin grið milli þeirra Sæmundar og Ög \textbf{móður} og veittar fullar tryggðir af góðvilja og ráðum Brands ábóta og Steinunnar húsfreyju og Álfheiðar móður Sæmundar og margra annarra góðra manna tillögu\\
\hline
1365&móðir&nvee&« Förum við austur yfir vatn og ofan til Eyvindarár að hitta Gróu frændkonu \textbf{móður} því að mér leiðist fálæti móður minnar\\
\hline
1366&móðir&nvee&föður Magnúss goða og \textbf{móður} er Þorvaldur Gissurarson átti\\
\hline
1367&móðir&nvee&faðir \textbf{móður} \\
\hline
1368&móðir&nvee&Ljótur Hallsson var faðir \textbf{móður} \\
\hline
1369&móðir&nvee&móðir \textbf{móður} \\
\hline
1370&móðir&nvee&Hann þiggur það illa \textbf{móður} \\
\hline
1371&móðir&nvee&faðir \textbf{móður} \\
\hline
1372&móðir&nvee&faðir \textbf{móður} \\
\hline
1373&móðir&nvee&er Styr vó \textbf{móður} \\
\hline
1374&móðir&nvee&Högni og Grani \textbf{móður} \\
\hline
1375&móðir&nvee&son \textbf{móður} frá Kiðjabergi\\
\hline
1376&móðir&nvee&Hrómundur var faðir \textbf{móður} \\
\hline
1377&móðir&nvee&Þeim þótti Gunnar vera kátur og kveða í hauginum \textbf{móður} \\
\hline
1378&móðir&nvee&Þá er Ásbjörn var riðinn til \textbf{móður} um sumarið hafði Skíði tekið í brott meyna með ráði Þorgerðar móður hennar\\
\hline
1379&nótt&nvee&Varð hann var við að þar voru gestir og svo hvert erindi \textbf{nætur} var\\
\hline
1380&nótt&nvee&Sáu þeir þá brátt að jarls lið \textbf{nætur} \\
\hline
1381&nótt&nvfe&» En við þessi orð hans treystist engi að kalla á \textbf{nótta} \\
\hline
1382&nótt&nvee&Sá þá enginn maður nauðung \textbf{nætur} henni í það sinn\\
\hline
1383&nótt&nvee&Síðan reri jarl þannug til og veitti þar harða atgöngu svo að \textbf{nætur} létu þá enn undan síga\\
\hline
1384&nótt&nvee&Enginn þeirra Helga \textbf{nætur} þiggur þar greiða\\
\hline
1385&nótt&nveeg&Vaka þeir nú allir með vopnum það \textbf{næturinnar} var næturinnar\\
\hline
1386&nótt&nvee&Þorgnýr fagnar honum vel og bað hann ganga til \textbf{nætur} er hann var vanur að sitja\\
\hline
1387&nótt&nvee&Og um kveldið eftir náttverð mælti Sturla við Guðnýju húsfreyju að slá skyldi hringleik og fór til alþýða \textbf{nætur} og svo gestir\\
\hline
1388&orð&nhfe&« er tekur til Vésteins og svo mun mér þykja nokkura stund og meira ann eg honum en Þorkeli bónda mínum þótt við megum aldrei \textbf{orða} \\
\hline
1389&orð&nhfe&Og er hann var kominn í rekkju þá kemur \textbf{orða} Ásgerður og lyftir klæðum og ætlar niður að leggjast\\
\hline
1390&orð&nhfe&Hví skyldir þú eigi hyggja fyrir því áður þú hétir þeirri ferð að þú hefir ekki ríki til þess að mæla í mót Ólafi \textbf{orða} \\
\hline
1391&orð&nhfe&Þorgils kvaðst það eitt mundu vinna til \textbf{orða} sér að honum væri hvorki að síðan skömm né klæki\\
\hline
1392&orð&nhfe&» Konungur svarar \textbf{orða} máli og tók seint til orða\\
\hline
1393&orð&nhfe&Grettir kastaði sér á bak aftur ofan í vatnið og sökk sem steinn \textbf{orða} \\
\hline
1394&orð&nhfe&Og er hann kom þar tók Sighvatur \textbf{orða} orða\\
\hline
1395&orð&nhfe&þeir koma þar og hittir Guðmundur sauðamann \textbf{orða} er Oddur hét\\
\hline
1396&orð&nhee&Vildi eg Ljótur frændi að við bæðum okkur \textbf{orðs} að skilja menn þó að okkur sé það til orðs lagið af nokkurum mönnum\\
\hline
1397&orð&nhfe&Og þótti Lofti húskarlar Kolskeggs hafa höggið skóg sinn og beiddi þar bóta fyrir en Björn Þorvaldsson vildi engu bæta láta og taldi Loft ljúga allt til um \textbf{orða} \\
\hline
1398&orð&nhfe&Þar var Þorgeir hófleysa og konur nokkurar \textbf{orða} \\
\hline
1399&orð&nhfe&að því að mér þykir \textbf{orða} \\
\hline
1400&orð&nhfe&Flosi mælti til \textbf{orða} úr Mörk\\
\hline
1401&orð&nhfe&Frétti Þorbjörn það skjótt og ríður til skips og finnur sonu \textbf{orða} \\
\hline
1402&orð&nhee&« Eg ætla það fyrir \textbf{orðs} sakar vel fallið en þó fyrir orðs sakar annarra manna er á hefir leikið mun eigi af því verða\\
\hline
1403&orð&nhfe&Piltar tveir léku \textbf{orða} gólfinu\\
\hline
1404&orð&nhfe&Hverfa þeir nú frá til \textbf{orða} og leita þar Gísla og finna hann eigi sem von var\\
\hline
1405&orð&nhee&Það fyrst að vera kann \textbf{orðs} nokkurar skæðar tungur taki svo til orðs að eg renni frá þér fyrir hugleysi ef eg ríð í braut\\
\hline
1406&orð&nhfe&að hann þóttist koma til \textbf{orða} og þótti hann fagna sér vel\\
\hline
1407&orð&nhfe&Ganga þeir nú til \textbf{orða} og leita Gísla og finna hann eigi\\
\hline
1408&orð&nhfe&« að sá Þorsteinn komi nokkurn tíma \textbf{orða} Heiðarskóg\\
\hline
1409&orð&nhfe&« Fór nokkuð fjarri \textbf{Orða} sem eg gat til\\
\hline
1410&orð&nhfe&Dyraverðir svöruðu og kváðu það engan sið að ókunnir menn gengju þá inn er konungur \textbf{orða} yfir borðum\\
\hline
1411&orð&nhfe&Hann lést ætla út til \textbf{orða} að kaupa sér þarfindi\\
\hline
1412&orð&nhfe&að þau breiddu \textbf{orða} léreft og þurrkuðu er vot höfðu orðið\\
\hline
1413&orð&nhfe&Þorgerður undraði þetta mjög og hugði að fóstra \textbf{orða} mundi svo gömul að hún mundi eigi barn mega eiga\\
\hline
1414&orð&nhfe&Vildi eg til hafa \textbf{orða} þitt orða fullting og framkvæmd því að þeir eru hér flestir menn að mikils munu virða þín orð\\
\hline
1415&orð&nhfe&Þar kom þó um síðir að Bjarni tekur svo til orða \textbf{orða} \\
\hline
1416&orð&nhfe&En er þeir voru skammt \textbf{orða} ströndu komnir þá tók Þórólfur til orða\\
\hline
1417&orð&nhfe&En er hann var klæddur lét hann kalla til sín spekinga yfir dómum með honum og réðu um vandamál en það var eigi vandalaust því að konungi líkaði illa ef dómum var hallað frá réttu en eigi hlýddi að mæla á móti \textbf{orða} \\
\hline
1418&orð&nhfe&Nú mun eg ekki hafa hér um mörg orð því \textbf{orða} ekki er von að þið skipist af framhvöt orða ef þið íhugið ekki við slíkar bendingar og áminningar\\
\hline
1419&orð&nhfe&Húskarlar voru úti að Hofi og gengu inn og sögðu Bjarna að þeir Þorvaldur voru heim komnir og sögðu þá eigi erindlaust farið hafa \textbf{orða} \\
\hline
1420&orð&nhfe&En í miðjum flotanum lá konungsskipið og þar næst Sigurðar skip en á annað borð konungsskipinu lá Nikulás en þar næst Eindriði \textbf{orða} \\
\hline
1421&ráð&nhee&lítill og frár \textbf{ráðs} fæti og einarður\\
\hline
1422&ráð&nhee&Þá leitaði hann þess ráðs að fyglarar hans tóku smáfugla \textbf{ráðs} \\
\hline
1423&ráð&nhfe&Skal eg nú reyna vitsmuni þína \textbf{ráða} \\
\hline
1424&ráð&nhee&Flosi nam \textbf{ráðs} og mælti\\
\hline
1425&ráð&nhee&Skúta gekk til \textbf{ráðs} þá er þeir Glúmur skildu og reið með hlíðinni og sér nú mannareiðina og veit að það má honum eigi endast að finna þá\\
\hline
1426&ráð&nhee&Flosi spurði hann þá \textbf{ráðs} hvað honum þætti þá líkast\\
\hline
1427&ráð&nhfe&Hann seilist til birkirafts eins og kippir brott úr \textbf{ráða} \\
\hline
1428&ráð&nhfe&Er mér og kunnug öll göng og \textbf{ráða} en gæfuvant er til slíkra ráða\\
\hline
1429&ráð&nhee&« Þú skalt fara sendiför mína til \textbf{ráðs} og mæla þessum orðum við hann að þú þykist þurftugur að hann sé forstjórnarmaður þíns ráðs\\
\hline
1430&ráð&nhfe&Það var enn eitt kveld um veturinn að Þórður talaði til Bjarnar og voru þeir þá drukknir báðir og þó Björn \textbf{ráða} \\
\hline
1431&ráð&nhee&En er Ólafur hafði verið um vetur \textbf{ráðs} Íslandi og er vor kom þá ræða þeir feðgar um ráðagerðir sínar\\
\hline
1432&ráð&nhee&Yngva nafn var lengi síðan haft í hans ætt fyrir tignarnafn og Ynglingar voru síðan kallaðir \textbf{ráðs} ættmenn\\
\hline
1433&ráð&nhfe&Og er hann kemur heim ræddi hann um við föður sinn að hann skyldi taka við stýrimanninum og kvaðst ætla góðan dreng vera og mikils verðan og tjáði málið fyrir honum vel um \textbf{ráða} \\
\hline
1434&ráð&nhee&Eg hefi eigi þurft \textbf{ráðs} hér til en nú þykir mér þess ráðs þurfa er svo ber til\\
\hline
1435&ráð&nhee&Var það þá ráðs tekið að þeir sendu Hjalta \textbf{ráðs} upp á þing og hleyptu þeir upp þinginu og flettu vestanmenn vopnum og klæðum og hrossum og gekk Guðmundur Þórðarson af þinginu er mest var fyrir vestanmönnum\\
\hline
1436&ráð&nhee& \textbf{ráðs} getur Bjarni Gullbrárskáld í kvæði því er hann orti um Kálf Árnason\\
\hline
1437&ráð&nhee&Eða hvað skal nú til ráða taka Ásgrímur \textbf{ráðs} \\
\hline
1438&ráð&nhfe&« Það er eg mun fara heim fyrst en síðan mun eg fara upp til Grjótár og segja þeim tíðindin og láta illa yfir \textbf{ráða} \\
\hline
1439&ráð&nhfe&Og svo sem Þorbjörn öngull var þrotinn að ráðagerðum leitar hann þangað til trausts sem flestum þótti ólíklegast en það var til \textbf{ráða} og spurði hvað þar væri til ráða að taka hjá henni\\
\hline
1440&ráð&nhee&Hann sagði honum víg Klængs með þeim atburðum sem verið höfðu og hann sagði að Órækju var vestan von með miklu liði á hendur \textbf{ráðs} \\
\hline
1441&ráð&nhee&Og einn dag að kveldi sjá þeir eld í kleifunum fyrir sunnan ána \textbf{ráðs} \\
\hline
1442&ráð&nhfe&Kann vera að oss takist annað \textbf{ráða} betur til en nú\\
\hline
1443&ráð&nhee&En er \textbf{ráðs} Teitur óx upp þá var honum ráðs leitað\\
\hline
1444&ráð&nhfe&Og er hann kom í sundið sér hann fjölda \textbf{ráða} í sundinu\\
\hline
1445&ráð&nhfe&» Magnús sonur Haralds konungs stýrði þá \textbf{ráða} \\
\hline
1446&ráð&nhee&Svo gerði Bragi frændi minn þá er hann varð fyrir reiði Bjarnar Svíakonungs að hann orti drápu tvítuga um hann eina nótt og þá þar fyrir höfuð \textbf{ráðs} \\
\hline
1447&ráð&nhee&Og það mundi eg vilja að þú gæfir til þess önnur þrjú hundruð \textbf{ráðs} að við hefðum aldrei fundist og muntu taka svívirðing fyrir mannskaða\\
\hline
1448&ráð&nhfe&En tveir eru kostir fyrir höndum \textbf{ráða} \\
\hline
1449&ráð&nhee&Hann fór til \textbf{ráðs} við Brand biskup og leitaði ráðs við hann\\
\hline
1450&ráð&nhee&að hann lést varla þola mega vansa þann og ámæli er leiddi af málum \textbf{ráðs} og leitaði ráðs við Guðmund prest\\
\hline
1451&ráð&nhee&« Sérðu frændi mörg spjót koma upp \textbf{ráðs} hólunum og menn með vopnum\\
\hline
1452&ráð&nhee&kvað Vigdísi engar sakir hafa fundið Þórði þær er sannar væru og til \textbf{ráðs} mættu metast « og var Þórður eigi að verr menntur þótt hann leitaði sér nokkurs ráðs að koma þeim manni af sér er settur var á fé hans og svo var sökum horfinn sem hrísla eini\\
\hline
1453&ráð&nhee&Jarlinn átti festarmey þar \textbf{ráðs} Englandi og fór hann þess ráðs að vitja og ætlaði brullaup sitt að gera í Noregi en aflaði til á Englandi þeirra fanga er honum þóttu torfengst í Noregi\\
\hline
1454&ráð&nhee&Hann bað þá Grím og Gunnar fara til Þóris « og segið honum þessi tíðindi » og biðja hann \textbf{ráðs} \\
\hline
1455&ráð&nhee&Þar varð fyrir Kár heimamaður \textbf{ráðs} frá Hofi og stóð þegar í gegnum hann\\
\hline
1456&ráð&nhee&Fór hann þá enn til \textbf{ráðs} og fann Grím Þórhallsson og leitaði hann þá ráðs hversu hann skyldi þá breyta\\
\hline
1457&ráð&nhee&Þetta þykir frændum \textbf{ráðs} óvirðing er hann bregður þessum ráðahag og leita sér ráðs\\
\hline
1458&ráð&nhee&Fór þá Haraldur út til Englands á fund \textbf{ráðs} og kom ekki til Vallands síðan að vitja ráðs þessa\\
\hline
1459&sinn&fevee&Páll reið til \textbf{sinnar} því að hann átti Reykhyltingagoðorð og voru viðsjár og varðhöld með flokkum\\
\hline
1460&sinn&fehee&Þorleifur seldi sinn hlut \textbf{síns} og fór síðan til bús síns eftir það\\
\hline
1461&sinn&fekfe&Nú veistu að það er vandi þinn að fara á hendur þingmönnum þínum norður um sveitir á vorið með þrjá tigu manna og setjast að eins bónda sjö \textbf{sinna} \\
\hline
1462&sinn&fehee&Gengu þeir mjög fótspor móður sinnar um slíka hluti en Hólmkeli var það leitt og gat þó ekki að \textbf{síns} \\
\hline
1463&sinn&fekee&Það var eitt sinn í tali þeirra bræðra að Eyvindur kvaðst heyra gott af Íslandi sagt og fýsti bróður sinn Ketil til \textbf{síns} með sér eftir andlát föður síns\\
\hline
1464&sinn&fekfe&Hugðist hann konung mundu mýkja mega \textbf{sinna} þeirri stundu og koma sínu máli í betri vingan við konung með fortölum sinna manna\\
\hline
1465&sinn&fekfe&Sveinn flýði þá upp á Skáni og allt lið hans það er undan komst en Magnús konungur og hans lið ráku flóttann langt á land upp og varð þá lítil viðurtaka af Sveins mönnum eða \textbf{sinna} \\
\hline
1466&sinn&fekfe&Síðan gekk Sturla til þeirra Þórðar og Snorra og bað þá ganga til \textbf{sinna} hvað sem í gerðist\\
\hline
1467&sinn&fekee&En er fjölmennið kom sá Öngull að hann gat ekki að gert \textbf{síns} \\
\hline
1468&sinn&fekee&Unnur gekk til búðar föður síns \textbf{síns} \\
\hline
1469&sinn&fehee&Þeir deildu og um hvalmál nokkur og færðu það til \textbf{síns} og var hvortveggi hinn mesti fulltingsmaður síns máls\\
\hline
1470&sinn&fekee&Til hofsins skyldu allir menn tolla gjalda og vera skyldir hofgoðanum til \textbf{síns} sem nú eru þingmenn höfðingjum en goði skyldi hofi upp halda af sjálfs síns kostnaði\\
\hline
1471&sinn&fevee&fer heim til móður sinnar \textbf{sinnar} \\
\hline
1472&sinn&fehee&» Oddur kemur út hingað og fer til bús síns \textbf{síns} \\
\hline
1473&sinn&fekee&Arinbjörn spurði hver kona sú væri er hann orti mansöng um « hefir þú fólgið nafn hennar í vísu \textbf{síns} \\
\hline
1474&sinn&fehfe&En meðan Þórður hafði þetta að starfa þá höfðu Kolbeins menn komið stafnljám á skip \textbf{sinna} og dregið það fram milli skipa sinna\\
\hline
1475&sinn&fekee&Gunnar gerði ýmist að hann hjó eða skaut og hafði margur maður \textbf{síns} fyrir honum\\
\hline
1476&sinn&fehfe&Eldur var og mikill \textbf{sinna} gólfinu\\
\hline
1477&sinn&fekee&En Álfur hroði sonur hans kom þá gangandi í \textbf{síns} \\
\hline
1478&sinn&fekee&« Víða hefi eg nú sveimað síðan og er mér boðið til \textbf{síns} og vildi eg að þú færir með mér\\
\hline
1479&sinn&fekfe&Fór Þorsteinn suður um Dali og alla leið til \textbf{sinna} er hann kom til búa sinna\\
\hline
1480&sinn&fekee&En er hann kom til \textbf{síns} var þeim skipað í gestaskála og veitt þeim hið stórmannlegasta\\
\hline
1481&sinn&fevee&Gangið nú heim og verið kátir af því að þið munuð þess við þurfa ef þið skuluð deila við Hrafnkel að bera ykkur vel upp um hríð en segið þið öngum manni að við höfum liðveislu \textbf{sinnar} \\
\hline
1482&sinn&fehee&» Hún svaraði \textbf{síns} \\
\hline
1483&sinn&fekee&Hersteinn son Bergs prests vó að \textbf{síns} \\
\hline
1484&sinn&fehfe&» En stundu síðar sáu þeir og kenndu skip Sigvalda jarls og viku þau þannug að \textbf{sinna} \\
\hline
1485&sinn&fekfe&Meira var Lambi virður af mönnum en faðir \textbf{sinna} fyrir sakir móðurfrænda sinna\\
\hline
1486&sinn&fekee&Tók því engi \textbf{síns} atgeirinum\\
\hline
1487&sinn&fehee&Alls drápu þeir nær hundrað manna og tóku þar ógrynni fjár og komu aftur um vorið við svo \textbf{síns} \\
\hline
1488&sinn&fevee&En það var þó til \textbf{sinnar} tekið að hann iðraðist og gaf hálfan hluta eigu sinnar fátækjum mönnum en annan helming frændum mannsins\\
\hline
1489&sinn&fevee&Honum kom nú og í hug eggjan \textbf{sinnar} að þrótt og djarfleik mundi til þurfa að vinna slíkt afrek eða önnur en frami og fagurlegir peningar mundu í móti koma og hann mundi þá þykja betur gengið hafa en sitja við eldstó móður sinnar\\
\hline
1490&sinn&fevee&Féll Ásbjörn dauður \textbf{sinnar} stýrinu\\
\hline
1491&sinn&fekfe&» Síðan setti hann sjóðinn á nasir \textbf{sinna} svo fast að brotnuðu tvær tennur úr höfðinu og stóðu blóðbogar úr andlitinu og gekk Þorkell burt við svo búið en Karl fór til sinna manna\\
\hline
1492&sinn&fekee&Tekur hann við \textbf{síns} þegar að hann sér jarteignir Þorbjarnar vinar síns\\
\hline
1493&sinn&fekfe&Var þar hin skörulegasta veisla og hvíldu þau í einni hvílu \textbf{sinna} \\
\hline
1494&sinn&fekee&og Aldís Sigmundardóttir \textbf{síns} Húsafelli fór heim til föður síns\\
\hline
1495&sinn&fehfe&Var nú drukkið brúðhlaup \textbf{sinna} með hinni mestu sæmd og prýði\\
\hline
1496&sinn&fehee& \textbf{síns} getur Bjarni skáld\\
\hline
1497&sinn&fekee&Segir hann að þá væri \textbf{síns} og sonar síns en hann kveðst vera hennar arfi og tók hann allt féið undir sig\\
\hline
1498&sinn&fekfe&Eilífur hafði þrjá tigu manna sinna sveitunga \textbf{sinna} \\
\hline
1499&sinn&fekee&Og láttu eigi þá fæð sem á millum \textbf{síns} hefir verið svo ríkt ganga að þú virðir meira en það að hann er úr því héraði sem þú ert\\
\hline
1500&sinn&fehee&Annað sumar eftir fór Karlsefni til \textbf{síns} og Guðríður með honum og fór hann heim til bús síns í Reynines\\
\hline
1501&sinn&fevfe&Tveir menn féllu af Karlsefni en fjórir \textbf{sinna} Skrælingjum en þó urðu þeir Karlsefni ofurliði bornir\\
\hline
1502&sinn&fehee&Narfi tók ekki því síður til \textbf{síns} og mælti svo\\
\hline
1503&sinn&fevee&Þá tóku þeir á Kóreksstöðum við ómegð \textbf{sinnar} en misstu landsleigu sinnar við hann og sátu fyrir öllum vandkvæðum\\
\hline
1504&sinn&fekee&En um vorið eftir páska skipar Otkatla lönd sín og tók þá til \textbf{síns} er verið hafði um sumarið og síðan fór hún af Helgastöðum með allt sitt og inn til Möðrufells til föður síns og er úr þessi sögu\\
\hline
1505&sinn&fekfe&« Er eigi því mjög til \textbf{sinna} haldið\\
\hline
1506&sinn&fekfe&Hann var vin \textbf{sinna} en lítill vin flestra frænda sinna nema Sig-hvats\\
\hline
1507&sinn&fekfe&« Hví sýnist þér það \textbf{sinna} að vísa honum af hendi\\
\hline
1508&sinn&fehfe&Til þeirrar ferðar réðst Sigurður konungur með \textbf{sinna} \\
\hline
1509&sinn&fehfe&Ingimundur undi hvergi \textbf{sinna} \\
\hline
1510&sinn&fevee&En skammt mun til áður vér munum \textbf{sinnar} vísir verða\\
\hline
1511&sinn&fekee&son Skalla-Gríms « vildum vér þess biðja konungur að þú minntist þess er frændur hans hafa vel til þín gert en létir hann eigi gjalda þess er faðir \textbf{síns} gerði þótt hann hefndi bróður síns\\
\hline
1512&sinn&fekee&Eg þóttist gera vel við þig og hugði eg til ölmusu af þér en eg hefi af þér heitingar og hrakning en ekki til \textbf{síns} » og lét sem honum kæmi í allt skap\\
\hline
1513&sinn&fekee&Var þar og mart \textbf{síns} og fór það vel fram\\
\hline
1514&sinn&fekfe&Það var sagt Birni bukk að Víkverjar börðu \textbf{sinna} á bryggjum niðri\\
\hline
1515&sinn&fekfe&en var þó \textbf{sinna} og svo auðigur að hann vissi varla aura sinna tal\\
\hline
1516&sinn&fehfe&gerðu orð á að þeir mundu eigi leggja drag undir ofmetnað \textbf{sinna} og það að þeir mundu ganga þar örna sinna sem annars staðar á mannfundum á grasi þótt þeir væru svo stolts að þeir gerðu lönd sín helgari en aðrar jarðir í Breiðafirði\\
\hline
1517&sinn&fevee&Ólafur fór nú leiðar sinnar en þær konurnar bundu sár Más \textbf{sinnar} \\
\hline
1518&sinn&fekfe&hugþekkur og ástúðigur öllum liðsmönnum og eigi ómaklega því að þá enn heiðinn sýndi hann réttlæti um fram hátt annarra heiðinna manna svo að hlutskipti það allt er hann fékk í hernaði veitti hann þurföndum og til útlausnar herteknum mönnum og hjálpaði mörgum þeim er meinstaddir \textbf{sinna} \\
\hline
1519&sinn&fekee&bæði í leikum og í öðrum \textbf{síns} og gerðist fæð á með þeim frændum\\
\hline
1520&sinn&fekee&Hann mælti ekki orð og því var hann kallaður ómáli og ekki maður \textbf{síns} \\
\hline
1521&sinn&fehfe&Var \textbf{sinna} því þingi hinn mesti helgistaður en eigi var mönnum þar bannað að ganga örna sinna\\
\hline
1522&sinn&fevee&Sýndist mönnum þann veg helst sem hann mundi leikinn því að hann fór \textbf{sinnar} sér og talaði við sjálfan sig og fór svo fram um hríð\\
\hline
1523&sinn&fekee&Þorkell kvað Hall nú missa \textbf{síns} bróður síns og sagði hann þá eigi mundu þurfa að ganga á hólm við Þorstein « ef hann væri hér\\
\hline
1524&sinn&fekee&» Björn fór með eftir ráðum Eysteins og rak \textbf{síns} geldinga til Háls frænda síns á laun um nóttina\\
\hline
1525&sinn&fekee&og var það \textbf{síns} mikið fé svo að af því silfri lét Guttormur gera róðu eftir vexti sínum eða stafnbúa síns og er það líkneski sjö álna hátt\\
\hline
1526&sinn&fevee&» segir hann og greip þær upp \textbf{sinnar} og setti hina yngri í kné móður sinnar og bar þær svo á vinstra armlegg sér en hafði lausa hina hægri hönd og óð svo út á vaðið\\
\hline
1527&sinn&fekee&Hallfreður spyr \textbf{síns} hann beiddist\\
\hline
1528&sinn&fekee&» Þá lofuðu allir guð almáttkan fyrir þetta háleita stórmerki sem verðugt var er hann hafði varðveitt \textbf{síns} líf fyrir bænir og verðleika síns ástvinar\\
\hline
1529&sinn&fehee&Skyldi brullaup það vera \textbf{síns} Höskuldsstöðum\\
\hline
1530&sinn&fekfe&Það var siðvani \textbf{sinna} að hann lét hvern dag ganga í haga til sauða sinna og lét telja\\
\hline
1531&sinn&fekee&lést og eigi vita það nema safnaður væri \textbf{síns} Rangárvöllum og væri sú ráðagerð að þeim sé ætlað að verða í klofanum en Gissur fari eftir oss ef vér förum suður\\
\hline
1532&sinn&fehfe&» Er það og sögn manna að flestir konungar hafi það varast \textbf{sinna} \\
\hline
1533&sinn&fekee&Sat Grettir í Fagraskógafjalli svo einn vetur að honum voru öngvar aðfarir \textbf{síns} en þó misstu þá margir síns fyrir honum og fengu ekki að gert því að hann hafði gott vígi en átti jafnan vingott við þá sem næstir honum voru\\
\hline
1534&sinn&fehfe&Svo og eigi síður sýndi óvinur \textbf{sinna} opinberlega í slíkum hlutum og mörgum öðrum þeim er í frásagnir eru færðir hversu nauðigur hann lét laust sitt ránfengi og þann lýð er hann hafði áður allan tíma haldið hertekinn í villuböndum sinna bölvaðra skurðgoða þá er hann hvessti með slíkum áhlaupum sína grimmdarfulla reiði á þeim sem hann hafði vald yfir sem hann vissi nálgast sína skömm og maklegan skaða síns herfangs\\
\hline
1535&sinn&fevee&þóttust \textbf{sinnar} nær komnir að ráða fyrir kosti frændkonu sinnar og leist þeim allvænlega stofnað\\
\hline
1536&sinn&fekfe&Þá tekur Ingjaldur til \textbf{sinna} og mælti til förunauta sinna\\
\hline
1537&sinn&fehfe&Leið nú á sumarið og kom sá \textbf{sinna} sem ákveðinn var að brullaupið skyldi vera\\
\hline
1538&sinn&fekee&Egill gerði sendimenn til \textbf{síns} á Aurland\\
\hline
1539&sinn&fehee&En er konungurinn kom heim til ríkis síns þá varð hann frægur mjög víða um lönd af sínu þrekvirki og ágætum sigri og urðu \textbf{síns} vinir konungsins og frændur honum fegnir er hann kom heim með göfuglegum sigri og þóttust menn hann nálega úr helju heimt hafa\\
\hline
1540&sinn&fekfe&Þeir spyrja hvað er hann hét \textbf{sinna} \\
\hline
1541&sinn&fekfe&En er \textbf{sinna} leið sumarið sendi Þórður Andrésson mann með bréfi sunnan til þeirra Brandssona frænda sinna þann er Kraki hét\\
\hline
1542&sinn&fekee&Þorvaldur hittir Þórarin \textbf{síns} \\
\hline
1543&sinn&fekfe&» Síðan könnuðust þau við og vissi Ástríður skyn \textbf{sinna} honum\\
\hline
1544&sinn&fevee&Vildi Kolbjörn hefna \textbf{sinnar} á Gesti\\
\hline
1545&sinn&fekee&« Illa erum vér við komnir að þola ágang yðvarn en \textbf{síns} » og kvað vísu\\
\hline
1546&sinn&fehee&Er það ekki mikið land en grasloðin og \textbf{síns} skógarrunnar\\
\hline
1547&sinn&fekfe&Neytir Þórður þá vel saxins er Gamli konungur hafði gefið honum og varð margs manns bani áður en hann komst \textbf{sinna} \\
\hline
1548&sinn&fehee&stingur \textbf{síns} honum spjótskafti sínu\\
\hline
1549&sinn&fehee&fóru þá aftur er \textbf{síns} leið veturinn upp á mörkina\\
\hline
1550&sinn&fekee&fór þá til \textbf{síns} á fund Sveins konungs frænda síns\\
\hline
1551&sinn&fekfe&Kjartan þakkar konungi gjöfina og gengur til \textbf{sinna} og sýnir þeim skikkjuna\\
\hline
1552&sinn&fekfe&En hvort veldur þú því er vér megum eigi sækja til \textbf{sinna} í Almannagjá\\
\hline
1553&sinn&fekee&» Nú þykir mönnum um þetta \textbf{síns} vert\\
\hline
1554&sinn&fevee&Með henni fæddist upp son hennar er Kolfinnur \textbf{sinnar} \\
\hline
1555&sinn&fekfe&Skipuðu þá Sigfússynir til búa sinna og dvöldust \textbf{sinna} um daginn en um kveldið riðu þeir vestur yfir Þjórsá og sváfu þar um nóttina\\
\hline
1556&sinn&fevee&Hann mælir við Loðinn \textbf{sinnar} \\
\hline
1557&sinn&fehee&En er hann var búinn ferðar sinnar þá fór hann með liðinu fyrst suður \textbf{síns} Mæri og heimti þar saman lengi í Hereyjum og beið liðs síns\\
\hline
1558&sinn&fekee&Þá hljóp Þorbjörn mót Gretti og hjó til hans en hann brá við buklara hinni vinstri hendi og bar af sér en hann hjó fram saxinu og klauf skjöldinn af Þorbirni og kom saxið í höfuðið honum svo hart að í heilanum stóð og féll hann af þessu dauður \textbf{síns} \\
\hline
1559&sinn&fekee&Sigurður og Gyrður Kolbeinssynir \textbf{síns} \\
\hline
1560&sinn&fekee&Guðmundur var þá andaður og kom Eyjólfur þá í mót þeim og bauð til sín og síðan fór hver til \textbf{síns} er allir voru sýknir\\
\hline
1561&sinn&fekfe&Arinbjörn hersir var með Haraldi Eiríkssyni og gerðist ráðgjafi \textbf{sinna} og hafði af honum veislur stórlega miklar\\
\hline
1562&sinn&fevee&« og skal þín ráð \textbf{sinnar} \\
\hline
1563&sinn&fekfe&segja það að hann skyldi hafa í \textbf{sinna} ríki það veldi er honum þætti sér sæmilegt en báðu hann eigi fara á vald fjandmanna sinna með svo lítinn liðskost sem hann hafði þar\\
\hline
1564&sinn&fekee&Snýr hann þá til hests síns \textbf{síns} steig í stigreipið og ætlaði á bak og gekk eigi því að hann var þrotinn mjög af mæði\\
\hline
1565&sinn&fekfe&« Það er þó að segja að mikla gersemi \textbf{sinna} eg þar sem þú ert\\
\hline
1566&sinn&fehfe&Dregur upp skjótt á himininn \textbf{sinna} tekur að drífa\\
\hline
1567&sinn&fehfe&Og nú fara þeir á hans fund og bera upp fyrir \textbf{sinna} \\
\hline
1568&sinn&fekfe&En eftir messudaginn fóru þeir til \textbf{sinna} og gerði Kolbeinn menn undan norður til sveita til vina sinna og stefndi þeim til móts við sig á fjallið\\
\hline
1569&sinn&fekee&« Auðséð er mér það að sundurleit er skilning ykkur biskups og hans og eigi síður skil eg það að með kappi miklu fylgja hvorir sínu \textbf{síns} \\
\hline
1570&sinn&fehfe&En er þetta spurði Magnús \textbf{sinna} hvað Sveinn hafðist að þá fór hann til skipa sinna og hafði með sér Norðmanna lið það er þá var í Danmörk en sumt Dana lið\\
\hline
1571&sinn&fekee&» Þá voru þeir fluttir til \textbf{síns} og kölluðust göngumenn og koma þar er síldaferja ein var og spurðu hver fyrir réði\\
\hline
1572&sinn&fehfe&fara nú til \textbf{sinna} og fjölmenna hvorirtveggju\\
\hline
1573&sinn&fevfe&Skuluð þér þessi ráðagerð leyna því að líf vort \textbf{sinna} liggur við\\
\hline
1574&sinn&fekee&Sturla og Þorgils báðu þá vakra að vera \textbf{síns} sem við þyrfti og halda sér upp vel\\
\hline
1575&sinn&fehee&» Þráinn var skamma stund í hafi og kom til \textbf{síns} og fór heim til bús síns\\
\hline
1576&sinn&fekee&Þórir hljóp af baki hestinum en hesturinn ærðist og hljóp út á sjóinn \textbf{síns} \\
\hline
1577&sinn&fevee&Álfur tók þakksamlega við gjöfinni « og má hér gera mér af loðkápu » og bað Egil þar koma til \textbf{sinnar} er hann færi aftur\\
\hline
1578&sinn&fehfe&» Síðan skildu þeir og fór konungurinn \textbf{sinna} en Stúfur fór sinna erinda\\
\hline
1579&sinn&fekee&En hann segir til nafns síns hið sanna og til kyns og hugði hann að það mundi heldur víðara fara þar að hann mundi \textbf{síns} njóta en gjalda eigi föður síns því að Vésteinn var hinn vinsælasti maður í kaupförum\\
\hline
1580&sinn&fehee&Nú sigla þeir þaðan í haf landnyrðingsveður og voru úti tvö dægur áður þeir sáu land og sigldu að landi og komu að ey einni er lá norður af landinu og gengu þar upp og sáust um í góðu veðri og fundu það að dögg var \textbf{síns} grasinu og varð þeim það fyrir að þeir tóku höndum sínum í döggina og brugðu í munn sér og þóttust ekki jafnsætt kennt hafa sem það var\\
\hline
1581&sinn&fekee&Nú er þar til máls að taka að Hneitir bóndi úr Ávík fer til fundar við Hafliða \textbf{síns} \\
\hline
1582&sinn&fekee&og \textbf{síns} allir vel yfir því er hann kom heim\\
\hline
1583&sinn&fehee&Gestur leit til mannsins er gekk \textbf{síns} honum\\
\hline
1584&sinn&fevee&« Erum vér \textbf{sinnar} við komin því að hann vitjar hverja nátt sængur sinnar\\
\hline
1585&sinn&fehfe&Ríður nú hver heim til \textbf{sinna} er þetta boð hafa sótt\\
\hline
1586&sinn&fevee&Og koma þeir aftur til fundar við Börk og una allilla við sína för og þykjast fengið hafa mikinn mannskaða með svívirðing en komið engu \textbf{sinnar} \\
\hline
1587&sinn&fehee&Þá mælti húsfreyja \textbf{síns} \\
\hline
1588&sinn&fehee&Konungur bað þá koma til \textbf{síns} eftir um daginn\\
\hline
1589&sinn&fekee&Ætla eg sjaldan að kveðja þig til ferðar með mér og óþökk skaltu af mér hafa fyrir þessa \textbf{síns} \\
\hline
1590&sinn&fehfe&Helga dóttir bónda var þá \textbf{sinna} fótum og heil meina sinna\\
\hline
1591&sinn&fekfe&Gekk hann þá af skipum en lét menn fara með \textbf{sinna} hið ytra\\
\hline
1592&sinn&fehee&» Síðan skiljast þeir bræður og fer Þorkell til fundar við þá \textbf{síns} \\
\hline
1593&sinn&fehee&Lambi reiddist mjög við orðtak \textbf{síns} og kvað þá kunna sig ógjörla er þeir veittu honum átölur « því að eg hefi dregið yður undan dauða\\
\hline
1594&sinn&fekee&Kom hann þá til Hákonar \textbf{síns} \\
\hline
1595&sinn&fekee&Þetta kveld hið sama hafði smalamaður \textbf{síns} fundið Höskuld dauðan og fór heim og sagði Hróðnýju víg sonar síns\\
\hline
1596&sinn&fekfe&nær tveimur hundruðum sinna manna \textbf{sinna} \\
\hline
1597&sinn&fekee&Svo er sagt að Helga Granadóttir hljóp nú \textbf{síns} búi Háls bónda síns og heim til föður síns og hitti ekki Áskel\\
\hline
1598&sinn&fekee&Hann fór og utan á fund Eysteins konungs og naut þar mjög \textbf{síns} mörgum mönnum bróður síns\\
\hline
1599&sinn&fehee&Nú var þar Karlsefni fyrir og hafði albúið skip sitt til hafs og beið byrjar og er það mál manna að eigi mundi auðgara skip gengið hafa af Grænlandi en það er hann \textbf{síns} \\
\hline
1600&sinn&fekfe&Oddur ómagaskáld frændi \textbf{sinna} skyldi fylgja hesti þeirra frænda sinna um daginn\\
\hline
1601&sinn&fekfe&Þá kom að Hallur \textbf{sinna} Sandfelli við tíunda mann og fór þegar til með Glúmi og þeir tíu sem með honum voru\\
\hline
1602&sinn&fevee&Þeir bræður bjuggust til ferðar \textbf{sinnar} sem veglegast\\
\hline
1603&sinn&fevee&En Geir bóndi veik þeim svörum til dóttur \textbf{sinnar} en hún kvað vilja hér hlíta föður síns ráði\\
\hline
1604&sinn&fekfe&og hafði fyrst áratal til \textbf{sinna} er kristni kom á Ísland\\
\hline
1605&sinn&fekee&Sjórinn var að fallinn svo að hesturinn var nær \textbf{síns} sundi undir honum\\
\hline
1606&sinn&fekee&til Íslands og komu skipi sínu á Borðeyri í \textbf{síns} \\
\hline
1607&sinn&fevee&Snorri goði þreif til handarinnar Eyjólfi og fletti upp af erminni og sér að hann hafði gullhring mikinn á \textbf{sinnar} \\
\hline
1608&sinn&fekee&» Finnbogi kveðst ekki mundu það óttast við þá sem til \textbf{síns} eru\\
\hline
1609&sinn&fehee&« \textbf{síns} get eg að nú sé orðið það erindi\\
\hline
1610&sinn&fekee&En þá nótt \textbf{síns} \\
\hline
1611&sinn&fekee&Synir \textbf{síns} fara til Gests frænda síns og skora á hann að hann komi þeim utan með ráðum sínum og Gunnhildi móður þeirra og Auði er Gísli hafði átta og Guðríði Ingjaldsdóttur og Geirmundi bróður hennar\\
\hline
1612&sinn&fekee&» Tók hann þá jaxl úr pússi sínum er hann hafði höggvið úr Þráni og kastaði til Gunnars og kom í augað svo að þegar lá úti á \textbf{síns} \\
\hline
1613&sinn&fevee&Hann þiggur það illa \textbf{sinnar} \\
\hline
1614&sinn&fekfe&Þórður gerði þá bert að hann ætlaði vestur til \textbf{sinna} með flokk þann er hann fengi og reka harma sinna\\
\hline
1615&sinn&fevfe&og komu út það sumar í Dýrafjörð og sigldu þann sama dag í Haukadalsós sem Gísli hafði áður inn siglt \textbf{sinna} byrðingi sínum\\
\hline
1616&sinn&fehfe&Tekur hann þá til \textbf{sinna} og kippir honum upp enda var þá hvönnin nær öll upp tognuð\\
\hline
1617&sinn&fekfe&» Ingimar sagði \textbf{sinna} \\
\hline
1618&sinn&fehee&» Sveinungur gengur til rúms síns \textbf{síns} tekur ofan sax eitt mikið\\
\hline
1619&sinn&fekee&Allir stakkgarðsmenn viku sínu máli undir forsjá \textbf{síns} til sætta en Þórdís tók við búi í Vatnsfirði að ráði föður síns\\
\hline
1620&sinn&fekee&annaðhvort að fara á fund konungs og leggja allt á hans vald eða stefna af landi brott \textbf{síns} \\
\hline
1621&sinn&fevee&Kallaði Sturla Snorra í ölbúð og sagði honum þar vígið og bauð honum sína liðveislu slíka sem hann \textbf{sinnar} \\
\hline
1622&sinn&fekee&tveim vetrum eftir andlát föður síns \textbf{síns} \\
\hline
1623&sinn&fekfe&brýtur spjótið \textbf{sinna} skafti og hefir fyrir staf\\
\hline
1624&sinn&fekfe&Þetta sumar er þeir Gissur komu út sendi Þórður Sighvatsson út til \textbf{sinna} grön og Ara Ingimundarson til vina sinna\\
\hline
1625&sinn&fekfe&Ætla eg mig og ekki \textbf{sinna} að dyljast fyrir þér\\
\hline
1626&sinn&fehee&Hann kvaddi búa til \textbf{síns} og höfðu þeir Hrútur ætlað að veita honum atgöngu en treystust eigi\\
\hline
1627&sinn&fekfe&Nú finnast þeir \textbf{sinna} þinginu\\
\hline
1628&sinn&fekfe&Hygg að nú \textbf{sinna} langt frændum þínum ganga neðan kveðjurnar við þig\\
\hline
1629&sinn&fevee&Var hann jafnan utanlands vel virður af meira \textbf{sinnar} mönnum sakir menntanar sinnar\\
\hline
1630&sinn&fehfe&Litlu síðar stóð hann upp og gekk til \textbf{sinna} og sagði honum til vandkvæða sinna og bað hann flytja sig út og gerði hann svo og léði skip og flutti hann út og þakkaði Grettir honum fyrir þenna drengskap\\
\hline
1631&sinn&fevee&sagðist verið hafa í víking og oft einn \textbf{sinnar} brott komist\\
\hline
1632&sinn&fekee&Þá fór Eyjólfur á Helgastaði og máttu þeir eigi sættast því að hvorirtveggju kölluðust allt eiga það er þeir deildu um og urðu engi miðlunarmál með þeim því að hvorigir vildu láta né eitt af sínu máli og varð það síðan að stefnuför og stefndi Eyjólfur Önundi um afneyslu \textbf{síns} og kallaði síns neytt vera\\
\hline
1633&sinn&fekee&Griðkona gerði honum Þóri \textbf{síns} og lét segja honum víg Einars sonar síns og brá Þórir skjótt við og fór norður til Vopnafjarðar með tvo húskarla sína og fór á skipi yfir fljót og til Hofs\\
\hline
1634&sinn&fekfe&Fer Þormóður þá til \textbf{sinna} en Þorgeir ætlaði að fara suður á Reykjanes til frænda sinna\\
\hline
1635&sinn&fevee&Gunnlaugur þakkar honum boðið og kveðst þó vilja fara fyrst út til \textbf{sinnar} á vit festarmeyjar sinnar\\
\hline
1636&sinn&fekee&Nú hleypur upp allur flokkur \textbf{síns} og þykir biskupunum nú undarlega við bregða\\
\hline
1637&sinn&fevee&« ef atgeirinn er eigi \textbf{sinnar} lofti\\
\hline
1638&sinn&fekee&« Hefir mér og gott eina til gengið þótt eg byði hingað Finnunni \textbf{síns} \\
\hline
1639&sinn&fekfe&og eruð þér orðnir langt frá yðrum ættmönnum er mikils eru verðir og ei mundu þeir þvílíka skömm eða hneisu setið hafa sem þér hafið þolað um hríð og \textbf{sinna} ámæli fyrir haft\\
\hline
1640&sinn&fevee&Það er að segja frá för Finns að hann hafði skútu og á nær þremur tigum \textbf{sinnar} en er hann var búinn fór hann ferðar sinnar til þess er hann kom á Hálogaland\\
\hline
1641&sinn&fehee&Grettir reið heim til \textbf{síns} en Barði til bús síns\\
\hline
1642&sinn&fehfe&Guðrún unni Bolla mest \textbf{sinna} sinna\\
\hline
1643&sinn&fekfe&Síðan krækti hann fingrinum í augað svo að úti lá \textbf{sinna} kinninni\\
\hline
1644&sinn&fekfe&hafði veður komið mikið \textbf{sinna} \\
\hline
1645&sinn&fekee&Þar var og mikil ætt \textbf{síns} um Foldina\\
\hline
1646&sinn&fehee&Eftir það fór þar fram gleði og skemmtan \textbf{síns} \\
\hline
1647&sinn&fekee&En Þorgils var fyrir miðri fylkingu hjá Sturlu því að þar var mest þörf og barðist ágæta vel og þótti \textbf{síns} á að sækja en þar er hann var fyrir\\
\hline
1648&sinn&fehee&Þórður brá skjótt við \textbf{síns} \\
\hline
1649&sinn&fekee&Og þá reið Þorsteinn til heimboðs vestur í Hjarðarholt til \textbf{síns} pá mágs síns Höskuldssonar er þá þótti vera með mestri virðingu allra höfðingja vestur þar\\
\hline
1650&sinn&fekee&Var það þá samþykki kennimanna og að því brúðlaupi var Brandur ábóti \textbf{síns} \\
\hline
1651&sinn&fevee&Þetta sér Vestar og hljóp heiman og til vopna þeirra og tekur eitt sverð og bregður og hleypur upp að \textbf{sinnar} \\
\hline
1652&sinn&fekee&Þiðrandi fór landa í \textbf{síns} hafði aldur til\\
\hline
1653&sinn&fevee&» segir húskarl og rís upp og heldur seint og gengur til hurðar og sér mann úti fyrir dyrum í náttmyrkri og heilsar honum \textbf{sinnar} \\
\hline
1654&sinn&fekfe&Þorvarður Þorgeirsson var þá á Víðimýri með Kolbeini Tumasyni og er hann frá þessi tíðindi þá lét hann söðla sér hest og reið hann það \textbf{sinna} nótt er hann mátti eigi á dag uns er hann kom á Draflastaði og hafði hann eigi verið snerri en þá\\
\hline
1655&sinn&fehfe&Er það og satt að mikill munur er hvorum \textbf{sinna} verður betur til hjóna sinna\\
\hline
1656&sinn&fekfe&En jafnan er þeir skiptu liði sínu þá fylgdi konungi Norðmanna lið en Dagur fór þá í annan stað með sitt lið en Svíar í þriðja stað með sínu \textbf{sinna} \\
\hline
1657&sinn&fehfe&Síðan fer hann til \textbf{sinna} og hafði Geirmundur geymt hlutverka sinna því að loka var engi fyrir hurðum\\
\hline
1658&sinn&fekfe&kvaðst heldur vilja taka það á sig að gefa honum annað augað \textbf{sinna} \\
\hline
1659&sinn&fevee&hann gekk mót henni með húskarla sína \textbf{sinnar} veglyndi systur sinnar\\
\hline
1660&sinn&fehfe&« Fór nokkuð fjarri \textbf{sinna} sem eg gat til\\
\hline
1661&sinn&fekfe&Þann draum réð fóstra hennar enn til barngetnaðar þeirra á milli og mundi vera dóttir og lifa eftir ætt stór er henni sýndist limamargt tréið « en þar er þér þótti það bera blóma mikinn mun merkja siðaskipti það er koma mun og mun hennar afkvæmi hafa þá trú sem þá er boðin og mun sú \textbf{sinna} \\
\hline
1662&sinn&fekee&Öndóttur kvaðst halda mundu fénu til handa Þrándi dóttursyni \textbf{síns} \\
\hline
1663&sinn&fevfe&Þá svipaðist Þorsteinn um og sá ei fleiri menn upp standa af liði sínu en tólf og eftir það reru þeir til \textbf{sinna} og ætluðu til herbúða sinna\\
\hline
1664&sinn&fekee&Nú kom Hneitir heim og rak þegar Má í burt og kvað margt illt af honum standa \textbf{síns} \\
\hline
1665&sinn&fehfe&Hví skyldir þú eigi hyggja fyrir því áður þú hétir þeirri ferð að þú hefir ekki ríki til þess að mæla í mót Ólafi \textbf{sinna} \\
\hline
1666&skip&nhee&Karl gekk til \textbf{skips} og sagði henni\\
\hline
1667&skip&nhee&Eftir það koma þeir Kormákur og beiða Þórveigu skips \textbf{skips} \\
\hline
1668&skip&nhfe&Það mun mestri giftu stýra \textbf{skipa} \\
\hline
1669&skip&nhee&Hann kom fram í Leku og gengu þeir heim til \textbf{skips} gættu skips\\
\hline
1670&skip&nhee&Þess er getið að skip kom út í þetta mund norður og það ætlaði að fara \textbf{skips} \\
\hline
1671&skip&nhee&Þenna hinn sama morgun gekk Kári í borgina \textbf{skips} \\
\hline
1672&skip&nhfe&Í annan fylkingararm var Búi digri og Sigurður bróðir hans með tuttugu \textbf{skipa} \\
\hline
1673&skip&nhfe&og annan Jón prest smyril \textbf{skipa} prest smyril\\
\hline
1674&skip&nhee&Þá er Björn hafði verið fimm vetur með Skúla frænda sínum bar það til tíðinda að skip kom í \textbf{skips} \\
\hline
1675&skip&nhee&« Eg vil \textbf{skips} kunnigt gera að eg á einn skógarmann er heitir Hrafn og vil eg vara yður við að þér flytjið hann eigi um Íslandshaf þótt hann sé yður boðinn\\
\hline
1676&skip&nhee&Geirmundur fer til skips og kannast brátt við Ólaf því að hann \textbf{skips} \\
\hline
1677&skip&nhee&Þorsteinn fagri var eina nótt \textbf{skips} Hofi í það sinni\\
\hline
1678&skip&nheeg&Síðan búast þeir og búa báða bátana og höfðu tuttugu menn \textbf{skipsins} hvorum\\
\hline
1679&skip&nhfeg&En er þeir Erlingur heyrðu ópið hljópu þeir til vopna \textbf{skipanna} og stefndu síðan ofan til skipanna\\
\hline
1680&skip&nhee&» « Eigi muntu þess vís verða þenna \textbf{skips} \\
\hline
1681&skip&nhee&Hann færði alla vöru \textbf{skips} í Bulungarhöfn til skips og sagði Þorsteini tíðindin\\
\hline
1682&skip&nhee&lét Grímkel maklegan \textbf{skips} frá sér\\
\hline
1683&skip&nhfeg&Ríða þeir brott til \textbf{skipanna} en jarlsmenn gengu til báts síns og róa út til jarlsskipsins\\
\hline
1684&skip&nhee&Vémundur biður að son hennar fari með honum norður \textbf{skips} Sléttu til skips\\
\hline
1685&skip&nheeg&Skip það var gert eftir vexti Orms hins langa og vandað að öllu sem \textbf{skipsins} \\
\hline
1686&skip&nhee&reið heiman og til sels Steinólfs og hitti mann þann í Grjótárdal er þar bjó og Eiríkur hét og gaf honum til kníf og belti að hann segði honum þá er skógarmenn færu til skips þeir er hjá Steinólfi \textbf{skips} \\
\hline
1687&skip&nhee&Fór þá Ketill með tuttugu menn en \textbf{skips} aðra tuttugu eftir að gæta skips\\
\hline
1688&skip&nhfe&gera það bert að þeir munu her þeim stefna norður til Þrándheims á hendur Hákoni \textbf{skipa} \\
\hline
1689&skip&nhee&En það var allskonar fé bæði frítt og ófrítt og var því játað \textbf{skips} \\
\hline
1690&skip&nhee&Nú er menn fóru til skips um vorið þá mælti Illugi við \textbf{skips} \\
\hline
1691&skip&nhee&Sá maður var til skips kominn er \textbf{skips} hét\\
\hline
1692&skip&nhee&Annan dag eftir reið Björn suður í Hraunhöfn til skips og tók sér þar þegar far um sumarið og urðu heldur \textbf{skips} \\
\hline
1693&skip&nhee&Þeir Þórir fóru til \textbf{skips} og komu út í Dögurðarnes\\
\hline
1694&skip&nhfe&Þá spurðu þeir að Hákon var í Víkinni og fór Ingi konungur og Gregoríus austur og höfðu mjög mart skipa \textbf{skipa} \\
\hline
1695&skip&nhee&Þorkell kom brátt til \textbf{skips} var landfast orðið og keypti að stýrimönnum og að hásetum þá hluti er hann þurfti að hafa\\
\hline
1696&skip&nhee&Réðst hann þá til ferðar með konungi og var skrásettur í \textbf{skips} \\
\hline
1697&skip&nheeg&Þá gekk Grís til tals við Ljótólf frænda sinn og \textbf{skipsins} \\
\hline
1698&skip&nhee&Eftir um vorið bjóst Gísli \textbf{skips} skips og bauð á því mestan varnað að nokkur hlutur færi suður með fjalli\\
\hline
1699&skip&nhfeg&Þórir barðist alldjarflega og féll \textbf{skipanna} skipi sínu með mikilli hreysti\\
\hline
1700&skip&nhfe&« Það hefi eg heyrt yður mæla að renna skyldi \textbf{skipa} skipum og á land upp en aldrei það að renna til skipa svo að hver hlypi frá öðrum\\
\hline
1701&skip&nhee&Eyvindur fór með Þórði en setti \textbf{skips} að hann vildi eigi við Þórð skilja meðan hann var eigi sáttur um vígaferli sín\\
\hline
1702&skip&nhee&Hann lagði oftlega fé til höfuðs honum sem öðrum \textbf{skips} \\
\hline
1703&skip&nhee&Þorleifur fór heim til \textbf{skips} en Austmenn vistuðust\\
\hline
1704&skip&nhee&Og þegar er sár Geirs voru bundin sté Hörður á skip með tólfta mann og fór þegar inn til Brynjudals og kveðst enn vilja reyna \textbf{skips} \\
\hline
1705&skip&nhfe&bað þá láta upp \textbf{skipa} og syngja tíðir\\
\hline
1706&skip&nhee&Bauð hann mér um það þá er eg fór austan \textbf{skips} \\
\hline
1707&skip&nhee&Geitir fór til skips og hitti Þórarin og \textbf{skips} Hofs\\
\hline
1708&skip&nhee&Þess er við að geta að höfði sá gekk einum megin hjá sundunum er Hofshöfði heitir og skyldi þar hittast lið konungsins allt við \textbf{skips} \\
\hline
1709&skip&nhee&Flosi sagði vera ærið gott gömlum og \textbf{skips} og sté á skip og lét í haf og hefir til þess skips aldrei spurst síðan\\
\hline
1710&skip&nhee&« Kenna þykist eg að frásögn þeirra skipið að Þórður mun eiga og er maklegur fundur \textbf{skips} \\
\hline
1711&skip&nhfe&Nikulás konungur sendi orð Sigurði konungi Jórsalafara og bað hann veita sér lið og styrk allan af sínu ríki og fara með Nikulási konungi austur fyrir Svíaveldi til Smálanda að kristna þar fólk því að þeir er þar byggðu héldu ekki kristni þótt sumir hefðu við kristni \textbf{skipa} \\
\hline
1712&skip&nhee&Það bar til tíðinda um sumarið að skip kom út í Dýrafirði og áttu bræður \textbf{skips} \\
\hline
1713&skip&nhfe&Ingi konungur stóð upp og sagði mönnum frá hvað hann hafði spurt hvernug bræður \textbf{skipa} vildu við hann skipa og bað sér liðs en alþýða manna gerði góðan róm að máli hans og létust honum vilja fylgja\\
\hline
1714&skip&nhee&Og þá er öll föng Gunnars voru til skips komin og skip var mjög búið þá ríður Gunnar til Bergþórshvols og á aðra bæi að finna menn og þakkaði liðveislu öllum þeim er honum höfðu lið \textbf{skips} \\
\hline
1715&skip&nhee&En er boði var lokið þá reið Ólafur til \textbf{skips} og hitti Örn stýrimann og tók sér þar fari\\
\hline
1716&skip&nhee&Það sumar brá Ari til utanferðar en Stað seldi hann í hendur Þórði Sturlusyni og gifti honum Helgu dóttur \textbf{skips} \\
\hline
1717&skip&nhee&» Gekk Haukur síðan út og allir \textbf{skips} menn og fara leið sína til skips og halda í haf er þeir eru að því búnir og komu aftur til Noregs á fund Haralds konungs og líkaði honum nú vel því að það er mál manna að sá væri ótignari er öðrum fóstraði barn\\
\hline
1718&skip&nhee&Ríður Gunnar heim til \textbf{skips} en Kolskeggur ríður til skips og fer utan\\
\hline
1719&skip&nhfe&Og er hann kom \textbf{skipa} sundið sér hann fjölda skipa í sundinu\\
\hline
1720&skip&nhfeg&Voru þar og gyllt höfuð \textbf{skipanna} en seglin bæði voru stöfuð öll með blá og rauðu og grænu\\
\hline
1721&skip&nhfe&Um haustið hélt hann austan til \textbf{skipa} í þann tíma er leystist Eyrafloti\\
\hline
1722&skip&nhee&En er þeir fóru í brott valdi Sigríður vinum sínum gjafir \textbf{skips} \\
\hline
1723&skip&nhfe&liði sínu úr Noregi \textbf{skipa} \\
\hline
1724&skip&nhee&Og er þau tíðindi komu til Noregs var Þorlákur biskup kominn \textbf{skips} vígslu til skips og fór það sumar til Íslands og sagði þessi tíðindi út\\
\hline
1725&skip&nhee&Þá mælti Eiríkur \textbf{skips} \\
\hline
1726&skip&nhee&Fór Þorgeir heim á Reykjahóla en Þormóður \textbf{skips} Laugaból\\
\hline
1727&skip&nhee&Þorleifur fór nú til manna sinna og svaf af um nóttina og um morguninn rís hann upp og fer í kaupstaðinn og fréttist fyrir um góða kaupunauta og kaupslagar við þá um \textbf{skips} \\
\hline
1728&skip&nhee&Þeir sögðu víg \textbf{skips} og báðu sér skips inn til lands\\
\hline
1729&skip&nheeg&« Minna mun fram koma um stórræði yður og er hitt líkara að vér verðum allir drepnir sakir þess að menn munu eigi þola oss svo mikinn ójöfnuð sem vér \textbf{skipsins} \\
\hline
1730&skip&nhee&» Og við þetta skilja þeir \textbf{skips} \\
\hline
1731&skip&nhfeg&En eftir þingið búast þeir til féránsdóms og höfðu fjögur skip og þrjá tigu manna á hverju og réðu þeir Einar og Þórarinn og Þórður fyrir skipunum og komu innan að eyjunni í næturelding og sáu reyk yfir húsunum og spurði Einar hvort þeim sýndist svo sem honum að reykurinn væri eigi \textbf{skipanna} \\
\hline
1732&skip&nhee&Sagði Þórólfur föður sínum hvað til tíðinda hafði orðið í förum \textbf{skips} um sumarið\\
\hline
1733&skip&nhfe&En Þórður bað þá Hauk prest Auðunarson vita við Böðvar hvern veg honum væri gefið um tilskipan \textbf{skipa} því að hann er arfi minn\\
\hline
1734&skip&nhee&Og er engi annar \textbf{skips} ger en verða á burtu sem skjótast\\
\hline
1735&skip&nhee&Hann hugsar nú efni \textbf{skips} og þykir eigi verða mjög með öllu fylgt ef Þórólfur skal sleppa\\
\hline
1736&skip&nhee&Þeir fóru til skips tólf saman \textbf{skips} \\
\hline
1737&skip&nhfe&Þórir lagði sverði til hundsins og veitti honum sár mikið en jafnskjótt fló kesja konungsins undir hönd Þóri svo að út stóð um aðra \textbf{skipa} \\
\hline
1738&skip&nhee&En fyrir mismuna okkarrar tignar skaltu þiggja gullhring þenna » og dró af hendi \textbf{skips} \\
\hline
1739&skip&nhfeg&En þá gengur einnhver \textbf{skipanna} inn í stofuna og bað Eyvind Finnsson ganga út með sér skjótt\\
\hline
1740&skip&nhfe&En er skipin voru búin þá lögðust \textbf{skipa} sunnanveður hvöss\\
\hline
1741&skip&nhee&Nú ræð eg að þér búið ferð yðra sem fljótast aftur \textbf{skips} leið en þér skuluð færa mig á höfða þann er mér þótti byggilegast vera\\
\hline
1742&skip&nheeg&Þá mælti Þórður blígur að þeir skyldu á milli bols og höfuðs ganga allra Þorbrandssona en Steinþór kvaðst eigi vilja vega að liggjöndum \textbf{skipsins} \\
\hline
1743&skip&nhee&Svo er sagt að varð \textbf{skips} var Þorbergur að Mývatni eða enn heldur Arnarvatni og fer hann þegar til skips og bauð Austmönnum til sín\\
\hline
1744&skip&nhfe&Árni kvaðst ætla að hann mundi nokkuð gott ráð fyrir \textbf{skipa} \\
\hline
1745&skip&nhfe&Guðmundur var bæði ríkur og fjölmennur \textbf{skipa} \\
\hline
1746&skip&nhee&að eg mun því heita þér að þið Þóroddur skuluð eigi hafa skapraun af fundum okkrum Þuríðar hina næstu \textbf{skips} \\
\hline
1747&skip&nhfe&Þá kallar Steinólfur á sína menn og biður þá halda til skipa og láta þau gæta \textbf{skipa} \\
\hline
1748&skip&nhee&Þá mælti Finnur \textbf{skips} \\
\hline
1749&skip&nhfeg&Vandill þreif upp stafnljá og \textbf{skipanna} á meðal skipanna og í skipið Gunnars og dró þegar að sér\\
\hline
1750&skip&nhee&En eftir voru buklarar þeirra þrír og sótti þá Þóroddur prestur og mælti til vel og voru honum í hendur \textbf{skips} \\
\hline
1751&skip&nhee&» Keypti Glúmur honum skip að norrænum mönnum og bjó Ögmundur ferð sína og mikinn fjárhlut er faðir hans fékk \textbf{skips} \\
\hline
1752&sonur&nkee&Synir þeirra voru þeir Játmundur og Játvarður hinn \textbf{sonar} \\
\hline
1753&sonur&nkee&Þá kom kona jarls í stofuna og leit þá er komnir voru og sá að vera mundu útlendir \textbf{sonar} \\
\hline
1754&sonur&nkee&Kveld-Úlfur spurði fall Þórólfs \textbf{sonar} síns\\
\hline
1755&sonur&nkee&Eftir það gengu menn til tryggða og veittu frændur Lýtings Ámunda \textbf{sonar} \\
\hline
1756&sonur&nkee&dóttir Þórðar sonar \textbf{sonar} Þjóðrekssonar\\
\hline
1757&sonur&nkee&Tungu-Steinn hét maður er bjó \textbf{sonar} Steinsstöðum\\
\hline
1758&sonur&nkee&Annan dag eftir gekk Ófeigur yfir brú og hittir frændur sína Skarðamenn og biður að þeir gangi með honum til Lögbergs og svo gera \textbf{sonar} \\
\hline
1759&sonur&nkee&Segir hann að þá væri \textbf{sonar} og sonar síns en hann kveðst vera hennar arfi og tók hann allt féið undir sig\\
\hline
1760&sonur&nkee&Þóttist hann þó ærnu eiga að svara \textbf{sonar} alþingi að eigi væri þessi mál að kæra\\
\hline
1761&sonur&nkee&Þar er nú til máls að taka að þau Hávarður og Bjargey spyrja þessi \textbf{sonar} \\
\hline
1762&sonur&nkee&dóttir Ólafs \textbf{sonar} \\
\hline
1763&sonur&nkee&Mun það nú og vænst til \textbf{sonar} að þú gangir í sonar stað ef þú vilt með mér vera því að hamingjumót er á þér\\
\hline
1764&sonur&nkee&Hélst það allt um \textbf{sonar} ævi og Játvarðar sonar hans en Aðalsteinn kom ungur til ríkis og þótti af honum minni ógn standa\\
\hline
1765&sonur&nkee&dóttir Danps \textbf{sonar} \\
\hline
1766&sonur&nkee&« Vel lýgur sá er með vitnum lýgur og vænti eg að eigi fáir þú vitni til að eg hafi né eina manns konu \textbf{sonar} \\
\hline
1767&sonur&nkee&« Ekki kvíði eg því að eg geti eigi haldið mér réttum fyrir Hrúti og sonum hans og mun eg eigi fyrir því af landi \textbf{sonar} \\
\hline
1768&sonur&nkee&Árni óreiða var þá að að fylkja liði Snorra \textbf{sonar} norrænu og tókst það heldur ófimlega því að hann var eigi vanur því starfi\\
\hline
1769&sonur&nkee&Hann var sonur Gautreks konungs hins \textbf{sonar} \\
\hline
1770&sonur&nkee&Máttu og sjá hversu trúlyndur þessi maður hefir verið í sínum heitum þar sem hann fór hingað í ófriðarstað \textbf{sonar} eignum sínum í hendur oss\\
\hline
1771&spjót&nhee&En er lýsti stóðu þeir upp \textbf{spjóts} \\
\hline
1772&spjót&nheeg&Ótryggur lagði spjóti til \textbf{spjótsins} en Eyjólfur var í skarlatskyrtli rauðum\\
\hline
1773&sumar&nhee&« Eg vil afla mér \textbf{sumars} því langt er sumars eftir\\
\hline
1774&sumar&nhee&Fór hann til \textbf{sumars} og var með honum í góðu haldi\\
\hline
1775&sumar&nhee&» og fer þetta sem Þórarinn gat að þeim leiðist \textbf{sumars} fjöllum úti og fara heim\\
\hline
1776&sumar&nhee&Liðu þessir dagar til sumars og sumars dag \textbf{sumars} mælti Fríður til Búa\\
\hline
1777&sumar&nhee&En Steinunn húsfreyja föðursystir hans bað hann eigi fara en Guðmundur reið þó til \textbf{sumars} og með honum Egill skyrhnakkur\\
\hline
1778&sumar&nhee&Og er hann kom þar voru öll skip gengin til \textbf{sumars} og var það síð sumars\\
\hline
1779&sumar&nhee&Jarlinn kvað vera illa \textbf{sumars} í landi « og mun vera lítil útsigling en þó skalt þú hafa mjöl og við í skip þitt sem þú vilt\\
\hline
1780&sumar&nhee&Ingimundarsynir voru heima um veturinn og sátu á hinn óæðra bekk og fóru til engra leika eða þings og voru mjög \textbf{sumars} \\
\hline
1781&sveinn&nkee-s&En Magnús konungur gekk upp \textbf{Sveins} Fjóni\\
\hline
1782&sveinn&nkee-s&Var skip \textbf{Sveins} hroðið með stöfnum og öll skip Sveins önnur voru hroðin\\
\hline
1783&sveinn&nkee-s&En Órækja var þá á Seljaeyri er þeir bjuggust utan Andrés son Hrafns lögmanns af Katanesi og Andrés son \textbf{Sveins} en hann var son Sveins Ásleifarsonar\\
\hline
1784&sveinn&nkee-s&gerði þá svo harða atgöngu að Sveins menn hrukku fyrir og hrauð Magnús konungur það skip og fjöldi féll hans manna og mart fékk \textbf{Sveins} \\
\hline
1785&sveinn&nkee-s&Var þá sannspurt andlát Sveins jarls \textbf{Sveins} \\
\hline
1786&sveinn&nkee-s&Ólafur Noregskonungur fékk \textbf{Sveins} dóttur Sveins Danakonungs en Ólafur Danakonungur Sveinsson fékk Ingigerðar dóttur Haralds konungs\\
\hline
1787&sveinn&nkee&Höskuldur lagði ást \textbf{sveins} sveininn\\
\hline
1788&sveinn&nkee-s&Í þeirri hríð hrauðst skip Sveins framan um stafninn og \textbf{Sveins} \\
\hline
1789&sveinn&nkee-s&Þá var á Jaðri austur \textbf{Sveins} Sóla Áslákur Erlingsson\\
\hline
1790&sveinn&nkee-s&Hinn efra hlut nætur brast meginflóttinn \textbf{Sveins} Dönum því að þá hafði Haraldur konungur upp gengið með sína sveit á skip Sveins konungs\\
\hline
1791&sveinn&nkee-s&Og er þeir Sveinn sjá för þeirra þá riðu þeir í móti þeim og hittust við sjóinn gegnt \textbf{Sveins} \\
\hline
1792&sveinn&nkee-s&En hann neytti svo þeirrar frumtignar að hann kjöri ríkra manna sonu eða þá hluti aðra er þeim var mest eftirsjá að er látið höfðu en hans félögum þætti minnst fyrir að gefa upp og sendi síðan þeim er átt \textbf{Sveins} \\
\hline
1793&sveinn&nkee-s&Halldóra dóttir Sveins Helgasonar \textbf{Sveins} \\
\hline
1794&sveinn&nkee-s&En er hann var nær kominn \textbf{Sveins} þá sendi hann Þóri bróður sinn til Sveins Úlfssonar að hann skyldi veita hjálp Þóri\\
\hline
1795&sveinn&nkee-s&Jarl dæmdi það að Sveinn konungur skyldi fá Gunnhildar dóttur \textbf{Sveins} en Búrisláfur konungur skyldi fá Þyri Haraldsdóttur systur Sveins konungs en hvortveggi þeirra skyldi halda ríkinu og skyldi vera friður milli landa\\
\hline
1796&sveinn&nkee-s&Og er þeir riðu á Haukadalsá hljóp áin ofan svo ákaflega að þá rak af baki er \textbf{Sveins} ánni voru\\
\hline
1797&sveinn&nkee-s&En er orðsending Sveins konungs kom til \textbf{Sveins} inn á Eggju\\
\hline
1798&sveinn&nkee-s&hann Sigvalda jarl til Vindlands að njósna um ferð \textbf{Sveins} og gildra svo til að fundur þeirra Sveins konungs mætti verða og Ólafs konungs\\
\hline
1799&sveinn&nkee-s&Sigvaldi jarl lagðist þá ferð eigi undir höfuð og fer á fund Sveins Danakonungs og ber þetta mál fyrir hann og kemur \textbf{Sveins} svo fortölum sínum að Sveinn konungur fær í hendur honum Þyri systur sína og fylgdu henni konur nokkurar og fósturfaðir hennar er nefndur er Össur Agason\\
\hline
1800&sveinn&nkee&Mun yður fjarri fara bræðrum að þér munuð þar til hefnda leita sem ofurefli er fyrir er þér getið eigi launað sín tillög slíkum mannfýlum sem Þorkell \textbf{sveins} \\
\hline
1801&sveinn&nkee-s&kominn her Danakonungs og Eiríkur jarl hafði þá og búinn sinn her og þeir höfðingjarnir mundu þá koma austur undir Vindland og þeir höfðu ákveðið að þeir mundu bíða Ólafs konungs við ey þá er Svöld \textbf{Sveins} \\
\hline
1802&sveinn&nkee-s&Systir Málmfríðar var Ingilborg er átti Knútur lávarður sonur Eiríks góða \textbf{Sveins} \\
\hline
1803&sveinn&nkee-s&greiddu þegar atróðurinn \textbf{Sveins} \\
\hline
1804&sverð&nheeg&Ætla eg þá marga er \textbf{sverðsins} mundu til ljá að svíkja hann ef þeir hefðu mátt\\
\hline
1805&sök&nvfe&Þorvaldur mælti \textbf{saka} \\
\hline
1806&sök&nvfe&þótti sem konungur væri heiftrækur um smærri hluti en þá er Kálfur hafði gert til saka við Harald konung \textbf{saka} \\
\hline
1807&sök&nvfe&Þá er Sturla kom á holtið fyrir ofan garðinn sendi Þórður Þorvaldsson mann til Þorkels prests og bað hann koma til \textbf{saka} við sig\\
\hline
1808&sök&nvfe&Gróa tók sverðið Brynjubít og spretti friðböndum og fékk \textbf{saka} \\
\hline
1809&sök&nvfe&Það var eitt sumar er hann kom út í Eyjafirði að Arngrímur bauð honum eigi til \textbf{saka} og mælti ekki við hann þótt þeir sæjust og fann það til saka að hann hefði fleira talað við Þórdísi konu hans en skaplegt væri\\
\hline
1810&sök&nvfe&Ert þú maður sannorður og kominn nær frétt og \textbf{saka} því trúa öllu er þú segir mér frá hvað til saka hefir orðið með þeim\\
\hline
1811&tíðindi&nhfe&Helga hét dóttir hans og var fríð kona sjónum og skörungur \textbf{tíðinda} \\
\hline
1812&tíðindi&nhfe&Finnbogi spurði hví hann færi eigi « eða sérð \textbf{tíðinda} nokkuð til tíðinda\\
\hline
1813&tíðindi&nhfe&Gunnar reið til búðar Rangæinga og var þar með frændum \textbf{tíðinda} \\
\hline
1814&tíðindi&nhfe&» Þeir sögðu slíkt \textbf{tíðinda} þeim þótti vera\\
\hline
1815&tíðindi&nhfe&» Þorsteinn svarar \textbf{tíðinda} \\
\hline
1816&tíðindi&nhfe&Kjartan kemur fyrir Hvítadal og heimti vaðmál \textbf{tíðinda} sem hann hét\\
\hline
1817&tíðindi&nhfe&Hann gekk til \textbf{tíðinda} og spurði þá tíðinda\\
\hline
1818&tíðindi&nhfe&Már Jörundarson færði bú sitt af Grund \textbf{tíðinda} Mársstaði\\
\hline
1819&tíðindi&nhfe&Hann hitti Ólaf konung er hann var í vesturvíking og gerðist hans maður og fylgdi honum \textbf{tíðinda} \\
\hline
1820&tíðindi&nhfe&Bæjarmaðurinn mælti \textbf{tíðinda} \\
\hline
1821&tíðindi&nhfe&Þorleifur Steinólfsson hélt enn \textbf{tíðinda} bónorðinu við Ketilríði en hún tók ekki fljótt\\
\hline
1822&tíðindi&nhfe&Þeir segja að hvortveggi Úlfur er fallinn og dauðir voru hálfur þriðji tugur \textbf{tíðinda} « en fimm einir komust undan með lífi og þó þeir allir sárir og barðir\\
\hline
1823&tíðindi&nhfe&Sá maður var \textbf{tíðinda} vist með Ljótólfi er Þórður fangari hét\\
\hline
1824&tíðindi&nhfe&En nokkrum dögum fyrir jól kom Gnípa ei til þeirra og þóttust þeir ei vita hvað af henni mundi \textbf{tíðinda} \\
\hline
1825&tíðindi&nhfe&Þorgrímur sat eftir \textbf{tíðinda} stólinum og beið ef nokkuð ryddist í búðardurunum\\
\hline
1826&tíðindi&nhfe&Loftur fóstri hans sat í búi \textbf{tíðinda} \\
\hline
1827&tíðindi&nhfe&Hún gengur til \textbf{tíðinda} og heilsar Þorgísli og öllum þeim og spurði þá tíðinda\\
\hline
1828&tíðindi&nhfe&En þá er sveinninn var tvævetur þá var hann almæltur og rann einn saman sem fjögurra vetra gömul \textbf{tíðinda} \\
\hline
1829&tíðindi&nhfe&Honum var boðið þar að vera en hann vildi heim ríða um nóttina og hitti hann Koll \textbf{tíðinda} leið og kveðjast þeir og spurðust tíðinda og spyr Þorvarður hvaðan Kollur væri að kominn en Kollur spyr í móti því hann fari um nætur\\
\hline
1830&tíðindi&nhfe&» Eyjólfur bað hann senda húskarl sinn til \textbf{tíðinda} og vita hvað til tíðinda var\\
\hline
1831&tíðindi&nhfe&Guðmundur reið fjölmennur til \textbf{tíðinda} og varð þar ekki til tíðinda\\
\hline
1832&tíðindi&nhfe&Nú er að segja hvað tíðinda er að selinu að Helgi var þar og þeir menn með honum sem fyrr var \textbf{tíðinda} \\
\hline
1833&tíðindi&nhfe&« Hvað er \textbf{tíðinda} orðið í ferð þinni er þér fær nú hlátrar eða hvað segir þú tíðinda\\
\hline
1834&tíðindi&nhfe&Hann var þá \textbf{tíðinda} er þetta var tíðinda\\
\hline
1835&tíðindi&nhfe&Þorkell hafði og mikinn drykk \textbf{tíðinda} skipi sínu\\
\hline
1836&tíðindi&nhfe&En þótt þú sért svo þrár að þú viljir engis manns ráði hlýða þá muntu lítið veita Gissuri ef þú ert hér drepinn og hlýst það þá af Gissuri sem hann mundi \textbf{tíðinda} \\
\hline
1837&tíðindi&nhfe&En er Þórólfur varð þess vís þá stóð hann upp og gekk til \textbf{tíðinda} við Egil og spurði með hverju móti hann hafði undan komist eða hvað til tíðinda hefði orðið í ferð hans\\
\hline
1838&tíðindi&nhfe&Einn dag ríður Óspakur upp \textbf{tíðinda} Mel og hittir Odd\\
\hline
1839&tíðindi&nhfe&Nokkurir menn voru uppi \textbf{tíðinda} vegginum með Kormáki er Narfi kom\\
\hline
1840&tíðindi&nhfe&ganga að lokrekkju þeirri sem Gísli hvíldi í og kona hans en Þorkell bróðir Gísla gengur upp fyrir í hvílugólfið og sér hvar að skór \textbf{tíðinda} liggja\\
\hline
1841&tíðindi&nhfe&Þeir riðu föstudag allan svo að þeir komu hvergi til \textbf{tíðinda} fyrr en á Valbjarnarvöllum\\
\hline
1842&tíðindi&nhfe&Hann kom þar snemma \textbf{tíðinda} og heimti Guðmund á tal við sig\\
\hline
1843&tíðindi&nhfe&» Vigfús fór til \textbf{tíðinda} og kenndi þar Ögmund frænda sinn og fagnaði honum vel og spurði tíðinda af Íslandi frá föður sínum\\
\hline
1844&tíðindi&nhfe&þar voru þeim fóstruð börn \textbf{tíðinda} \\
\hline
1845&tíðindi&nhfe&Var hann fullþroskaður \textbf{tíðinda} þá er þetta var tíðinda\\
\hline
1846&tíðindi&nhfe&Þeir voru menn \textbf{tíðinda} ungum aldri er þetta var tíðinda\\
\hline
1847&tíðindi&nhfe&Kolfinna fagnar vel Hallfreði og \textbf{tíðinda} tíðinda\\
\hline
1848&tíðindi&nhfe&Nú búast þeir heim \textbf{tíðinda} hauginum\\
\hline
1849&tíðindi&nhfe&» Bárður kvaðst því gjarna játa \textbf{tíðinda} \\
\hline
1850&tíðindi&nhfe& \textbf{tíðinda} skörulegastur að sjá\\
\hline
1851&tíðindi&nhfe&Hann sá þá bræður því að þeir höfðu þá gengið \textbf{tíðinda} hellisdyrunum\\
\hline
1852&tíðindi&nhfe&Önundur úr Meðaldal ræður fyrir búi þeirra Þorkels og Gísla en Saka-Steinn fyrir með Þórdísi á \textbf{tíðinda} \\
\hline
1853&tíðindi&nhfe&Og er Þórður kom \textbf{tíðinda} Staðarhól var það til tíðinda að Sturla stóð úti einn saman og voru menn að kasta steinum í stofuvegginn\\
\hline
1854&tíðindi&nhfe&Hann ríður til búðar Helga og stígur af \textbf{tíðinda} \\
\hline
1855&tíðindi&nhfe&Það haust er berserkirnir komu til \textbf{tíðinda} varð það til tíðinda að Vigfús í Drápuhlíð fór til kolgerðar þangað sem heita Seljabrekkur og með honum þrælar hans þrír\\
\hline
1856&tíðindi&nhfe&Hann býr heiman ferð sína og Helgi fóstri \textbf{tíðinda} með honum\\
\hline
1857&tíðindi&nhfe&Og er Þórður kom á Staðarhól var það til tíðinda að Sturla stóð úti einn saman og voru menn að kasta steinum í \textbf{tíðinda} \\
\hline
1858&vetur&nkfe&Eftir þessi tíðindi færir Þorgerður bú sitt upp \textbf{vetra} Grund og hefur þó annað bú að Upsum\\
\hline
1859&vetur&nkfe&Þeir riðu allir í Hvamm til \textbf{vetra} og bjuggu þar óspaklega heyjum og öðru\\
\hline
1860&vetur&nkeeg&Það er sögn flestra manna að Kjartan hafi þann dag gerst handgenginn Ólafi konungi er hann var færður úr hvítavoðum og þeir Bolli \textbf{vetrarins} \\
\hline
1861&vetur&nkfe&Varð Eiður \textbf{vetra} lögvitrastur svo að hann var af því Laga-Eiður kallaður\\
\hline
1862&vetur&nkfe&Kristín konungsdóttir bjó um lík Inga konungs og var hann lagður í steinvegginn í Hallvarðskirkju utar \textbf{vetra} kór hinum syðra megin\\
\hline
1863&vetur&nkfe&Vetri síðar varð sá atburður að þrælar nokkurir brutu haug til fjár sér en Þorgils kom að þeim og kvað það ekki vera þeirra fé og tók af þeim þrjár merkur en hrakti þá \textbf{vetra} \\
\hline
1864&vetur&nkfe&Guðrún Ásbjarnardóttir sagði og þá að Þorsteinn var faðir að barni \textbf{vetra} því er þá var nokkurra vetra gamalt\\
\hline
1865&vetur&nkfe&örðigur og manna best vígur og hinn hraustasti í öllum mannraunum þegar honum dróst aldur sem \textbf{vetra} mun verða sagt\\
\hline
1866&vetur&nkfe&Arngrímur var tveim vetrum eldri \textbf{vetra} Steinólfur og óxu eigi vinsælli menn upp í Eyjafirði eða allbetur væru að sér gervir og unnust allmikið\\
\hline
1867&vetur&nkee&Nú mun eg koma laugardag til \textbf{vetrar} er fimm vikur eru til vetrar og ef eg kem ei svo þá eruð þér einskis skyldir með mér að fara\\
\hline
1868&vetur&nkfe& \textbf{vetra} að Grís fjögurra vetra forgift og þá greiddi hann fúlgu hans og fengu þeir honum land til ábúðar og ekki skildu þeir um ábyrgð verka hans þaðan í frá og fór hann byggðum til Mela og sat þar um sumarið\\
\hline
1869&vetur&nkfe&Ásbjörn kom heim og var honum sagt \textbf{vetra} þessu\\
\hline
1870&vetur&nkfe&Þá var hann fimmtán vetra er hann vó vígið \textbf{vetra} \\
\hline
1871&vetur&nkfe&Dóttir bónda var á \textbf{vetra} \\
\hline
1872&vetur&nkfe&Þar var á skipi kona hans og son er Hrafnkell \textbf{vetra} \\
\hline
1873&vetur&nkfe&Hann bað \textbf{vetra} á alþingi þá er hún var fimmtán vetra gömul\\
\hline
1874&vetur&nkfe&Og hvert sinn er Þóroddur kom á stöðul gekk Glæsir að honum og daunsnaði um hann og sleikti um klæði hans en Þóroddur klappaði um \textbf{vetra} \\
\hline
1875&vetur&nkfe&Oddur mælti \textbf{vetra} \\
\hline
1876&vetur&nkfe&En \textbf{vetra} Egill kom heim lét Skalla-Grímur sér fátt um finnast en Bera kvað Egil vera víkingsefni og kvað það mundu fyrir liggja þegar hann hefði aldur til að honum væru fengin herskip\\
\hline
1877&vetur&nkfe&Guðbrandur hét sonur \textbf{vetra} \\
\hline
1878&vetur&nkfe&þá er hann var \textbf{vetra} Stiklarstöðum í orustu með Ólafi konungi bróður sínum\\
\hline
1879&vetur&nkfe&Síðan ríða þau vestur til \textbf{vetra} og koma heim til Borgar\\
\hline
1880&vetur&nkfe&« Vertu eigi \textbf{vetra} seytján vetra jafn mikill og sterkur sem þú ert\\
\hline
1881&vetur&nkfe&Þá var Hörður þrjátigi vetra \textbf{vetra} aldri\\
\hline
1882&vetur&nkfe&« Ekki óttast eg Þorbjörn á meðan eg hefi ekki gert til \textbf{vetra} við hann\\
\hline
1883&vetur&nkee&« Til \textbf{vetrar} verður hætt jafnan\\
\hline
1884&vetur&nkfe&Hann var þá fimm vetra er Sturla andaðist en Sighvatur þrettán \textbf{vetra} \\
\hline
1885&vetur&nkfe&Hann Auðun lagði mestan hluta fjár þess er var fyrir móður sína áður hann stigi á skip og var kveðið \textbf{vetra} þriggja vetra björg\\
\hline
1886&vetur&nkfe&Barði \textbf{vetra} þá átján vetra gamall\\
\hline
1887&vetur&nkfe&Þá er brenna var á Flugumýri var liðið \textbf{vetra} Önundarbrennu fjórum vetrum fátt í sex tigi vetra en frá Þorvaldsbrennu hálfur þriðji tugur vetra\\
\hline
1888&vetur&nkee&En jarl lét geyma hans svo að honum urðu engi færi á \textbf{vetrar} \\
\hline
1889&vetur&nkfe&Hann kom þar nærri hádegi og drap á dyr \textbf{vetra} \\
\hline
1890&vetur&nkfe&Ólafur bróðir \textbf{vetra} var þá tólf vetra gamall\\
\hline
1891&vetur&nkfe&Þar er nú til að taka að Guðmundur son \textbf{vetra} óx upp í Múla með Eyjólfi þar til er hann var níu vetra gamall\\
\hline
1892&vetur&nkfe&En til \textbf{vetra} vil eg bjóða yður Njálssonum og Kára og því heita að þér skuluð eigi gjafalaust í braut fara\\
\hline
1893&vetur&nkfe&og verður í öngvan máta \textbf{vetra} eftirbátur\\
\hline
1894&vetur&nkfe&Festu þeir þetta með því að Sturla lét Torfa prest ríða með goðorð sitt og beggja þeirra til þings og sýndu í því samband \textbf{vetra} \\
\hline
1895&vetur&nkee&Hann gekk út \textbf{vetrar} þá er tólf vikur voru til vetrar\\
\hline
1896&vetur&nkfe&Hann bjó í Dal \textbf{vetra} Markarfljót\\
\hline
1897&vetur&nkfe&Hrafnkell goði gaf honum hundrað \textbf{vetra} til að hann reyndi eftir hvar Grímur væri niður kominn\\
\hline
1898&vetur&nkfe&« Þér vil eg gefa hring þenna er Illugi gaf mér því að eg ann þér mest \textbf{vetra} en þú mun þessa gjöf eftir mig dauðan því að eg veit að þú munt lifa lengur en eg\\
\hline
1899&vetur&nkfe&Hann lifði og hálfan fjórða tug vetra hér í \textbf{vetra} \\
\hline
1900&vetur&nkfe&Þá var liðið frá falli \textbf{vetra} hálfur fjórði tugur vetra\\
\hline
1901&vetur&nkfe& \textbf{vetra} \\
\hline
1902&vetur&nkfe&skrúfhár og dapureygður og \textbf{vetra} best vígur\\
\hline
1903&vetur&nkfe&Hann hafði selför fram í Hrútafjarðardali og lét \textbf{vetra} vinna öndverð sumur\\
\hline
1904&vetur&nkfe&Hann var hyrndur vel og \textbf{vetra} fríðastur að sjá\\
\hline
1905&vetur&nkfe&Freysteinn hinn fagri bjó í Sandvík \textbf{vetra} Barðsnesi og átti Viðfjörð og Hellisfjörð og var kallaður landnámsmaður\\
\hline
1906&vetur&nkfe&En þau Ásbjörn og Oddbjörg áttu fjórar dætur \textbf{vetra} komust öngvar úr barnæsku\\
\hline
1907&vetur&nkfe&Hallur andaðist níu vetrum síðar en Ísleifur biskup \textbf{vetra} \\
\hline
1908&vetur&nkfe&Svo var hann vænn maður að eigi fékkst \textbf{vetra} jafningi\\
\hline
1909&vetur&nkfe&Honum unni Haraldur konungur mest sona sinna og virti hann \textbf{vetra} \\
\hline
1910&vetur&nkee&Síðan gekk hann inn og til \textbf{vetrar} og fékk langt óvit og rétti þó við úr því\\
\hline
1911&vetur&nkfe&Hann gekk í túnum \textbf{vetra} sumrum og drakk mjólk bæði vetur og sumar\\
\hline
1912&vetur&nkeeg&Er hann þar það er eftir var vetrarins \textbf{vetrarins} \\
\hline
1913&vetur&nkfe&Hann var þá \textbf{vetra} vetra og fór og með Þorsteini og voru þeir fimm saman og riðu út til Foss og þar yfir Langá\\
\hline
1914&vetur&nkeeg&» Þeir bjuggust þegar á fund jarls og völdu honum góðar gjafir og færðu \textbf{vetrarins} \\
\hline
1915&vetur&nkfe&» Þá var Helgi sex vetra gamall en Grímur fjögra vetra \textbf{vetra} \\
\hline
1916&vetur&nkeeg&Fóru menn nú heim og var kyrrt það sem eftir var vetrarins \textbf{vetrarins} \\
\hline
1917&vetur&nkfe&þá var liðið frá upphafi \textbf{vetra} sex þúsundir vetra og sjö tigir og þrír vetur\\
\hline
1918&vetur&nkfe&fimm síðan er Sigurður kom til \textbf{vetra} en sjö áður\\
\hline
1919&vetur&nkfe&Og þá fór hún móti Grímkatli syni sínum átta vetra gömlum því að honum dapraðist sundið þá og flutti hann til lands \textbf{vetra} \\
\hline
1920&vetur&nkee&Það var drottinsdaginn er fimm vikur voru til vetrar \textbf{vetrar} \\
\hline
1921&vetur&nkfe&Gísla \textbf{vetra} þetta vor eftir er Snorri var sextán vetra gamall gerði hann bú að Helgafelli og bjó þar tuttugu og þrjá vetur áður kristni var í lög tekin á Íslandi en þaðan frá bjó hann átta vetur að Helgafelli\\
\hline
1922&vetur&nkee&Skal sækja \textbf{vetrar} veislu á Höskuldsstaði þá er tíu vikur eru til vetrar\\
\hline
1923&vetur&nkee&Og er það mitt ráð að hver maður ríði heim af þingi og sjái um bú sitt í sumar meðan töður manna eru \textbf{vetrar} \\
\hline
1924&vinur&nkee&Gaf konungi eigi að sigla þann dag \textbf{vinar} \\
\hline
1925&vinur&nkee&og skal neyta góðs \textbf{vinar} Eyvindar vinar míns » og brá saxi fyrir brjóst sér og drap sig\\
\hline
1926&vinur&nkee&» Hann sendi þá norður \textbf{vinar} Hálogaland til Úlfs vinar síns og sagði þar gott fjár að afla í skreiðfiski\\
\hline
1927&vinur&nkfe&Gissur sendi menn suður um heiði \textbf{vina} vina sinna\\
\hline
1928&vinur&nkee&Eða hversu skal eg fara \textbf{vinar} að eg komist í eyna\\
\hline
1929&vinur&nkfe&Þá er Sturla hafði senda þá Svarthöfða til Geirshólms reið hann sjálfur vestur á Reykjahóla til \textbf{vina} og gerði þaðan menn sína vestur í fjörðu til Hrafnssona og annarra vina sinna og stefndi vestan liði og lagði þann stefnudag að allir skyldu koma til Sauðafells laugardaginn fyrir Lárentíusmessu\\
\hline
1930&vinur&nkee&Þorvarður skoraði \textbf{vinar} Kálf um skipkaup en hann svarar\\
\hline
1931&vinur&nkee&En nú tekur fjárhagur minn að óhægjast fyrir lausafjár sakir en hefir kallað verið hingað til heldur \textbf{vinar} \\
\hline
1932&vinur&nkfe&Þá sendi Ólafur konungur um haustið Hrana fóstra sinn til \textbf{vina} að eflast þar að liði og sendu Aðalráðssynir hann með jartegnum til vina sinna og frænda en Ólafur konungur fékk honum lausafé mikið að spenja lið undir þá\\
\hline
1933&vinur&nkfe&Símonarsynir \textbf{vina} höfðingja og vina Eysteins konungs og Sigurðar konungs\\
\hline
1934&vinur&nkee&Fór eg og því mest af Íslandi að eg þóttist vita að þú mundir vilja láta hefna hirðmanns þíns og vinar \textbf{vinar} mest af Íslandi að eg þóttist vita að þú mundir vilja láta hefna hirðmanns þíns og vinar\\
\hline
1935&vor&fekee&Mörður greiddi út heimanfylgju \textbf{vors} og reið hún vestur með Hrúti\\
\hline
1936&vor&fekee&« Hvern leiðir þú eftir þér þar \textbf{vors} soninn\\
\hline
1937&vor&fekee&« Það er öllum auðsýnt mönnum að konungur má gera af ráði Óttars það sem hann vill þó eð hann skýrði um kvæði \textbf{vors} \\
\hline
1938&vor&fekee&Þessi ætlan er nú er sett er þvert \textbf{vors} mínu skapi því að eg kalla þetta vera ófæru\\
\hline
1939&vor&fekee&höfðingja vors \textbf{vors} \\
\hline
1940&vor&fekee&Þá er Friðrekur biskup og Þorvaldur komu til Íslands voru liðin \textbf{vors} holdgan vors herra Jesú Kristi níu hundruð ára og eitt ár hins níunda tigar en hundrað tírætt og sex vetur frá upphafi Íslands byggðar\\
\hline
1941&vor&fekee&og hleypti Haraldur honum þó í brott að óvilja \textbf{vors} og fylgdi honum sjálfur\\
\hline
1942&vor&fekee&Tak nú hér hundrað silfurs og lát hann á brott og skal eg svo til stilla að hann sé eigi hér tekinn á þínum varnaði svo að það sé þér lagið til \textbf{vors} en vér munum þó eftir honum leita þó að föður vors sé eigi að hefndara\\
\hline
1943&vor&fekee&« Skulum vér nú \textbf{vors} \\
\hline
1944&vor&fekee&Þegar er hann kom við land \textbf{Vors} hann upp á Vors til Vigfúss\\
\hline
1945&víg&nhee&« Til \textbf{vígs} mælir þú þetta\\
\hline
1946&víg&nheeg&Var sæst á það mál og var ger góð sæmd \textbf{vígsins} \\
\hline
1947&vísa&nvfe&Þeir sáu þá það er þeim þótti því líkast sem tungl tvö full eða törgur stórar og var \textbf{vísna} millum arn ... sn ein mikil\\
\hline
1948&vísa&nvee&Hann var lágur sem \textbf{vísu} og digur\\
\hline
1949&þing&nhee&« fyrir sakir \textbf{þings} og aðdráttar en Grettir vill ekki starfa\\
\hline
1950&þing&nhee&Þá sáu þeir mikinn fjölda \textbf{þings} fara til þings og báru í milli sín mannlíkan mikið\\
\hline
1951&þing&nhee&Þeir fóru við það í brott \textbf{þings} \\
\hline
1952&þing&nhee&Nú líða nokkur misseri \textbf{þings} því og eitthvert sinn reið Ásbjörn til þings með menn sína\\
\hline
1953&þing&nhee&Þorkell lauk málum sínum \textbf{þings} þinginu\\
\hline
1954&þing&nhee&Þorvaldur þakkaði honum boðið og lést vita búrisnu \textbf{þings} en kveðst þó eigi vilja þar mat hafa\\
\hline
1955&þing&nhee&» Og fór oft heldur lítt með þeim bræðrum \textbf{þings} \\
\hline
1956&þing&nhee&Þeir sögðu að þess beiddu þeir að hann riði heim í Odda og hefði þar setu en þeir létust mundu hafa aðra í Skarði og bíða svo \textbf{þings} en fjölmenna síðan til þings og vita hvorir þá yrðu aflameiri\\
\hline
1957&þing&nhee&Og er vorar safna þeir að sér mönnum og fara suður til \textbf{þings} og koma í Norðurtungu og stefna Arngrími til þings í Þingnes og Hænsna-Þóri\\
\hline
1958&þing&nhee&Órækja Snorrason gerði bú í Deildartungu um vorið og Filippus mágur \textbf{þings} með honum\\
\hline
1959&þing&nhee&lét þess von ef vinir \textbf{þings} kæmu til þings að Skafti mundi eigi jafnstórlega láta\\
\hline
1960&þing&nhee&Hann átti þriðjung í goðorði við Þorgeir og synir hans hinn þriðja \textbf{þings} \\
\hline
1961&þing&nhee&» Gunnar var til \textbf{þings} fár og ekki illa\\
\hline
1962&þing&nhee&Þeir \textbf{þings} menn og þeir Órækja skildu í Tjaldanesi og reið Sturla til þings en þeir Órækja fóru til Stafaholts og dvöldust þar litla hríð áður þeir fóru suður\\
\hline
1963&þing&nhee&En var þó ráð höfðingja að auka eigi vandræði í héraðinu og láta bíða \textbf{þings} og fóru hvorirtveggju til þings of sumarið og voru áttar stefnur að málum\\
\hline
1964&þing&nhee&» Guðrún kvaðst um það mundu engu heita og þótti sinn veg hvoru þeirra og skildu með \textbf{þings} \\
\hline
1965&þing&nhee&» sagði hún \textbf{þings} \\
\hline
1966&þing&nhee&Gunnar reið til \textbf{þings} og Kolskeggur með honum\\
\hline
1967&þing&nhee&Það var um sumarið er menn bjuggust til þings þá beiddi Egill Grím að ríða til þings \textbf{þings} honum\\
\hline
1968&þing&nhee&Síðan skaltu hlaupast í brott og til Skútu og biðja hann ásjá og skaltu fíflast á Sigríði fóstru \textbf{þings} og gefa henni margt en tala aldrei um Skútu við hana\\
\hline
1969&þing&nhee&Njáll beið \textbf{þings} eina nótt því að hann ætlaði að hann skyldi riðið hafa til þings með honum\\
\hline
1970&þing&nhee&« fyrir því að önga ógn býð eg þér að sinni en þú veist til hvers þú hefir \textbf{þings} \\
\hline
1971&þing&nhee&En það var þar \textbf{þings} að búinn einn andast úr kvöðinni en Oddur kveður annan í staðinn\\
\hline
1972&þing&nhee&Það var einn dag í þingbrekku \textbf{þings} Snorri goði spurði Þorstein hvort hann hefði þangað búið mál mörg til þings\\
\hline
1973&þing&nhee&Líður nú vetur sjá og er sumar kemur þá líður framan til þings \textbf{þings} \\
\hline
1974&þing&nhee&Það er þá enn einn dag að Bjargey gengur til \textbf{þings} við Hávarð\\
\hline
1975&þing&nhee&Gerði þá margur sem vant var að fara til fundar við Njál en hann lagði það til mála manna sem ekki þótti líklegt að eyddust sóknir og varð af því þræta mikil er málin máttu eigi lúkast og riðu menn heim af þingi \textbf{þings} \\
\hline
1976&þing&nhee&Kolskeggur bróðir \textbf{þings} fýsti hann að ríða til þings « mun þar vaxa sæmd þín við því að margur mun þar að þér víkja\\
\hline
1977&þing&nhee&Ábóti hét að leggja til samnings með þeim en bað Ögmund eigi halda vini sína til rangra hluta með ofkappi því að þess er von að Sæmundur vilji það eigi \textbf{þings} \\
\hline
1978&þing&nhee&Heitir þar síðan \textbf{þings} Hávarðsstöðum\\
\hline
1979&þing&nhee&» Hallvarður var og þar kominn og bauð að ríða til \textbf{þings} með þeim\\
\hline
1980&þing&nhee&Nú eru menn \textbf{þings} komnir til þings og er leitað um sættir milli þeirra höfðingjanna og segir Þorgils að hann vill þessu máli eigi með kappi fylgja og kvaðst meira hafa gert fyrir úrlausna sakir og bænastað frændanna\\
\hline
1981&þing&nhee&Ögmundur ríður þá brott af Síðu og vissu menn óglöggt um ferðir hans hvert hann mundi snúið \textbf{þings} \\
\hline
1982&þing&nhee&» Grímur fór á burt eftir þetta \textbf{þings} \\
\hline
1983&þing&nhee&Vatnsdælir fjölmenntu mjög og svo hvorirtveggju \textbf{þings} \\
\hline
1984&þing&nheeg&Grettir hjó hann banahögg og því komst sá á fætur er honum hafði fylgt og fór þegar á fund jarls og sagði honum þessi \textbf{þingsins} \\
\hline
1985&þing&nhee&Gunnar reið til þings og \textbf{þings} og synir hans og Sigfússynir\\
\hline
1986&þing&nhee&En er konungurinn kom heim til ríkis síns þá varð hann frægur mjög víða um lönd af sínu þrekvirki og ágætum sigri og urðu allir vinir konungsins og frændur honum fegnir er hann kom heim með göfuglegum sigri og þóttust menn hann nálega úr helju heimt \textbf{þings} \\
\hline
1987&þing&nheeg&að honum létti brátt \textbf{þingsins} er þeir riðu til þingsins\\
\hline
1988&þing&nhee&« Til þess munum vér ráða að bera þetta mál \textbf{þings} \\
\hline
1989&þing&nheeg&en borgarmenn sóttu \textbf{þingsins} þingsins\\
\hline
1990&þing&nhee&Hann lét kirkju gera \textbf{þings} bæ sínum og hélt vel trú sína\\
\hline
1991&þing&nheeg&Þá stefndi Geitir Brodd-Helga um fé Höllu til \textbf{þingsins} og fjölmennti hvorutveggi mjög til þingsins og varð Helgi fjölmennari en Geitir hafði mannval betra\\
\hline
1992&þing&nhee&« Biðja vil \textbf{þings} þig mágur að þú ríðir til þings með mér með alla þingmenn þína\\
\hline
1993&þing&nhee&En þá er \textbf{þings} ekkja orðin bjó hún í Hvammi í Skorradal\\
\hline
1994&þing&nhee&kvað hann vera að telja silfur \textbf{þings} \\
\hline
1995&þing&nhee&Fór Ólafur konungur þannug með allmikið fjölmenni er hann hafði haft austan úr landi og svo það lið er þá hafði komið til hans á Rogalandi og \textbf{þings} \\
\hline
1996&þing&nhee&Gunnar reið til þings og Njáll og synir \textbf{þings} og Sigfússynir\\
\hline
1997&þing&nhee&Björn vill það eigi og koma til þings og sættust þar á málið og hlaut Björn að \textbf{þings} þrjár merkur silfurs fyrir níðreising og vísu\\
\hline
1998&þing&nhee&Um morguninn er menn komu út í Kirkjubæ sáu menn þar \textbf{þings} mikinn\\
\hline
1999&þing&nhee&« Hví skal nú fjölmennari fara \textbf{þings} \\
\hline
2000&þing&nhee&nýtir menn og ei jafnmiklir sem ættin þeirra og báðir kvongaðir og menn frá þeim \textbf{þings} \\
\hline
2001&þing&nhee&Það sumar eftir reið Snorri til alþingis eftir vanda en þeir riðu ekki til \textbf{þings} Þórður og Sturla\\
\hline
\end{longtable}
\end{document}
